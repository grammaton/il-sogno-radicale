%-------------------------------------------------------------------------------
% \section*{del soffio sconosciuto}
%-------------------------------------------------------------------------------
\begin{quote}
  \begin{sf}
    \small
    \begin{description}
      \item[Bremonte:] Mi viene in mente, dicevo, una confidenza che X. m’ha
      fatto su quell’epoca eroica di ricerca sonora. \cite{bertoncini07}
    \end{description}
  \end{sf}
\end{quote}

\emph{In che modo la scrittura di Mario Bertoncini partecipa alla comprensione
di ciò che è Bottega, Laboratorio e Opera? In che modo la lettura dei suoi
scritti può svelare un principio di operazione?}

La metodologia dialettica di Bertoncini impone un momento di riflessione
sull'equilibrio \emph{opera-operazione}\footnote{
  Appunti sulla tensione opera-operazione:\\
  \url{https://github.com/grammaton/il-sogno-radicale/blob/bf51bade8fa8ea7be60f7df9d57efe19a7c4a1d9/src/v03/opera-operazione/2025-10-26-opera-operazione-impaginato.pdf}
} messo in scena. La sua scrittura è un
teatro dell'intelletto, un'immaginario e, insieme, uno scenario, quindi uno
spazio scenico e un'ambientazione, dove il pensiero si rappresenta attraverso
persone multiple, in una forma possibile di \emph{sympòsion} emergente. Una
vetta di comunicazione indiretta \cite{ronchi2001} che svetta su una pianura di
informazione.

\emph{Bremonte}: è una \emph{parola-strumento}. {\phonfont *Bre-}, come brezza,
come \emph{breath}, come l'espansione in un gonfiare\footnote{Pokorny IEW 146,
  167-68, 169, 170-71, 171-72. {\phonfont *bhes-} (soffiare, respirare): radice
  imitativa fondamentale per il vocabolario respiratorio nelle lingue indo-europee;
  {\phonfont *bhreu-} (gonfiare, germogliare): radice collegata ai processi di
  espansione e crescita {\phonfont *bhreus-} (bollire, suonare): radice con campo
  semantico che include fenomeni acustici e movimento turbolento.}.
\emph{Bremonte} è il monte che soffia, il monte che respira. La brezza è il vento
catabatico, fresco, ciclico, il vento soffiato dalla montagna sulla valle.
\emph{Bremonte} è, come ogni opera di Bertoncini, un enigma naturale, un momento
di realtà. L'esplorazione umana è mossa dalla volontà di decodifica di enigmi,
la realtà del mondo necessità volontà di soluzione, che poi chiamiamo verità,
verità supposte\ldots~%a creazione di enigmi è tensione, tendenza alla natura.

%L'uso di \emph{X.} per l'io autobiografico rivela un tratto fondamentale: nel
Nel momento stesso in cui l'autore\footnote{%
  La parola autore deriva dal latino \emph{auctor}, che affonda le sue radici
  nel verbo \emph{augēre}, significante “far crescere”, “aumentare”, “dare
  incremento”. L'autore è chi fa crescere qualcosa dal nulla, chi aumenta la
  realtà attraverso la creazione. La parola greca più vicina ad \emph{auctor} è
  \textgreek{αὐξάνω} (\emph{auxánō}), che significa “far crescere”, “aumentare”,
  “accrescere”. Questo verbo greco e il latino \emph{augēre} (da cui \emph{auctor})
  derivano entrambi dalla medesima radice indoeuropea: Pokorny IEW 84-85
  {\phonfont *au̯eg-, u̯ōg-, aug-, ug-} (aumentare, incrementare, amplificare).
} cerca di catturare la propria esperienza
attraverso la scrittura, diventa \emph{altro} da sé. «X.» non è
semplicemente un espediente stilistico, ma l'incarnazione del «neutro» \cite{blanchot82} — quella dimensione anonima in cui l'autore
si dissolve nell'atto stesso della creazione. «X.» materializza questa dismissione: l'io più intimo diventa l'incognita
più radicale perché la scrittura trasforma il soggetto scrivente in oggetto scritto.
«X.» richiama la \textgreek{χ} (\emph{chi}), lettera che la tradizione algebrica
ha eletto a simbolo dell'incognito, di ciò che deve essere cercato e trovato.
Nell'atto autobiografico questa incognita non trova soluzione: la scrittura,
anziché rivelare il sé, ne svela l'irraggiungibilità.


% «X.» materializza questa
% dismissione: l'io più intimo diventa l'incognita più radicale perché la scrittura
% trasforma il soggetto scrivente in oggetto scritto. «X.» come \textgreek{χ}
% (\emph{chi}), “cosa sconosciuta”, mostra come l'atto autobiografico, anziché
% rivelare il sé, ne sveli l'irraggiungibilità.


%Bertoncini non racconta di sé attraverso \emph{X.}, ma piuttosto \emph{diventa X.}.

%-------------------------------------------------------------------------------
%\section*{fascino}
%-------------------------------------------------------------------------------

\begin{quote}
  \begin{sf}
    \small
    \begin{description}
      \item[Bremonte:] %Non è esatto.
      Un impulso determinante per il mio amico – quasi traumatico, a detta sua!
    \end{description}
  \end{sf}
\end{quote}

Traumatico è ciò che irrompe nell'ordinario flusso dell'esperienza, creando una
discontinuità che il soggetto non riesce immediatamente a integrare. Il
\textgreek{τραῦμα} (\emph{traûma}, “ferita” o “lesione”)
è un evento che eccede la capacità di elaborazione simbolica, lasciando una
sorta di “corpo estraneo” nell'economia psichica. Il ferire, che è un colpire in
modo da rompere la continuità (di cui la ferita è lesione traumatica
caratterizzata da soluzione di continuità), è un penetrare, un attraversare
spezzando, una superficie continua. Verbalizzare la ferita, significa verbalizzare quella dimensione dell'esperienza
estetica che fora, che penetra la comprensione concettuale e determina una
trasformazione del soggetto, sotto il fitto dei colpi del fascino.

«Perché il fascino?» \cite{blanchot82}

\begin{quote}
  \begin{sf}
    \small
    Vedere presuppone la distanza, la decisione separatrice, la facoltà di non
    essere in contatto e di evitare la confusione del contatto. Vedere significa
    che tale separazione è, tuttavia, divenuta incontro. Ma cosa succede quando
    quel che si vede, per quanto distante, sembra toccarvi con un contatto
    sconvolgente, cosa succede quando il modo di vedere è una sorta di tocco,
    quando vedere è un contatto a distanza? Quando quel che viene visto si
    impone allo sguardo, come se lo guardo fosse colto, toccato, messo in
    contatto con l’apparenza? Non si tratta di un contatto attivo, ciò che resta
    ancora dell’iniziativa e dell’azione in un contatto vero e proprio, ma lo
    sguardo è trascinato, come assorbito in un movimento immobile e in un fondo
    senza profondità. Quel che ci viene dato in un contatto a distanza è
    l’immagine, e il fascino è la passione dell’immagine.
  \end{sf}
\end{quote}

L'impulso determinante, quell'incontro con l'\emph{opera-operazione}, quasi traumatico,
descrive la dimensione del fascino come evento che destituisce il soggetto dalla
sua posizione ordinaria, dalla sua posizione di controllo. L'inconsapevolezza
dell'evento fondativo dispone che l'esperienza preceda la coscienza.%\footnote{Fascino, Blanchot, APT: fascino(esperienza,coscienza).}.

\begin{quote}
  \begin{sf}
    \small
    \begin{description}
      \item[Bremonte:] %Non è esatto.
      Un impulso determinante per il mio amico – quasi traumatico, a detta sua! –
      occorse durante una visita ad un piccolo studio romano di ricerca
      elettroacustica il cui tecnico, Guido Guiducci, persona in possesso di
      grande inventiva artigianale,% aveva costruito un oggetto simile alla
      % “chitarra abruzzese”, ma dotato d’un numero infinitamente maggiore di
      % corde metalliche sottilissime. Le corde, vicinissime l’una all’altra,
      % avrebbero dovuto avere una funzione analoga a quella dell’\emph{eco a
      % piastra}, di cui Guiducci voleva dotare lo studio, ma che avrebbe
      % rappresentato uno sforzo notevole per le non floride condizioni economiche
      % di esso. Il tentativo fallì; l’oggetto, prima di finire nel mucchio della
      % spazzatura, servì al Guiducci per una curiosa applicazione: soffiando
      % lievemente sulle corde e registrando l’azione per mezzo d’un microfono
      % sensibilissimo, il tecnico produsse casualmente una sequenza di suoni
      % eolici, una fascia microintervallica di grande complessità spettrale,
      [\ldots] e inconsapevolmente diede l’avvio ad una trentennale ricerca che il mio
      amico misterioso – come ama chiamarlo lei – perseguì con costante
      applicazione esplorando un mondo sonoro inedito le cui possibilità egli, a
      tutt’oggi, pensa di non aver ancora esaurito.
    \end{description}
  \end{sf}
\end{quote}

La ricerca che deriva da questo trauma suggerisce come l'enigma non si esaurisca,
una caratteristica essenziale del \emph{Rätselcharakter} \cite{adorno1970}: l'\emph{opera-operazione}
«autentica» contiene possibilità poetiche senza mai consegnarsi continuamente.
Il «mondo sonoro»  inedito indica una prospettiva ontologica, un'apertura verso
possibilità di esistenza precedentemente inaccessibili. Questo risuona con
l'idea \emph{blanchotiana} che l'arte apra spazi di alterità radicale.
Con «le cui possibilità\ldots~di non aver ancora esaurito» Bertoncini mantiene
aperto l'orizzonte dell'inesplorato, preservando quel carattere enigmatico che
continua ad orientare la ricerca.
L'opera esercita una forza di attrazione che precede e condiziona ogni possibile
comprensione. È l'esperienza di essere afferrati, prima ancora di afferrare.

Una volta in operazione, l'opera «autentica» mantiene un nucleo non riducibile
al concetto, un “di più” che resiste alla spiegazione. Questo carattere
enigmatico è condizione poetica dell'autenticità.
Questo \emph{trauma fondativo} del primo incontro con l'enigma sonoro è un momento di
vocazione, una chiamata (\emph{vocatio}) che determina un orientamento
(esistenziale). Il fascino \cite{blanchot82} e l'enigma \cite{adorno1970} convergono
nel descrivere dove la soglia dell'esperienza poetica si fa esperienza di
trasformazione.

%-------------------------------------------------------------------------------
%\section*{menippo}
%-------------------------------------------------------------------------------

\begin{quote}
  \begin{sf}
    \small
    \begin{description}
      \item[Bremonte:] Il volume sonoro sviluppato da un fascio di corde che
      vengano eccitate con la lieve pressione dell’aria (dal fiato o dal vento)
      è minimo. Sia l’esperimento del Guiducci che i successivi tentativi
      compiuti da X. al riguardo, forniscono soltanto un’indicazione sulla
      natura e sulla qualità sonora del fenomeno, senza potere in alcun modo
      renderlo adatto a sostenere il peso d’un discorso musicale\ldots
      \item[Menippo:] A non parlare delle esigenze d’una sala da concerto!
      \item[Bremonte:] Appunto. Egli costruì quindi un sistema basato sul punto
      nevralgico di tutti gli strumenti acustici tradizionali: il ponticello.
      Fabbricò dei ponticelli mobili di vario spessore e di materiali diversi
      (legno, plastica, cartone, metallo)\ldots
      [\ldots] di modo che le corde assolvono acusticamente la doppia funzione
      di \underline{\emph{produzione} del suono} e di \underline{\emph{amplificazione} di esso}, cioè costituiscono in
      pari tempo anche una cassa di risonanza. L’amplificazione elettronica
      avviene dal ponticello, su cui X. adatta dei cristalli microfonici
      piezoelettrici.
      \item[Menippo:] Che io sappia, però, il microfono a cristallo offre una
      risposta di frequenza limitata, nelle regioni gravi addirittura manchevole.
      \item[Bremonte:] Giusto. È appunto questa la ragione dei ripetuti
      tentativi compiuti dal mio amico, della sua lunga ricerca sonora non
      ancora conclusa.
    \end{description}
  \end{sf}
\end{quote}

Avanzando in questa fantasiosa analisi dell'architettura dell'enunciazione
bertonciniana, Bremonte è il “monte che freme” ed evoca l'inquietudine creativa,
Menippo\footnote{%
  Menippo innovò la forma prosimetrica, mescolando prosa e versi,
  per discutere argomenti filosofici seri in spirito di ridicolo e satira. Per
  Bakhtin la menippea è un genere “carnevalizzato” che, combinando serietà
  filosofica e comicità, alto e basso, testa idee attraverso situazioni
  estreme, mantenendo sempre un legame con l'attualità del proprio tempo. Esso
  consente l'attivazione di «scene di scandalo, comportamenti eccentrici,
  discorsi e performance inappropriati, cioè tutti i tipi di violazioni del
  corso generalmente accettato e consueto degli eventi e delle norme stabilite
  di comportamento» e «la preoccupazione per questioni attuali e d'attualità» \cite{bakhtin84}
}%
è l'indagine classica, richiama la satira menippea, genere dialogico che mescola
serio e giocoso, filosofia e ironia: il dialogo per esplorare contraddizioni
filosofiche. Menippo è la tradizione che attraversa i secoli.\footnote{%
  Marshall McLuhan nella lettera a Michael Hornyansky del 3 febbraio 1976
  scrive:

  Caro Professor Hornyansky,
  Uno studente in un seminario ha recentemente citato alcuni suoi commenti
  riguardo ai miei presunti punti di vista sul libro stampato. Ho mandato di
  recente un articolo su questa questione a Newsweek, di cui allego una copia.
  Partendo dagli studi simbolisti, molto tempo fa iniziai a studiare gli effetti
  degli artefatti umani. Il lavoro di Flaubert, per esempio, è interamente
  interessato a questi effetti nel modo in cui essi plasmano la psiche e la
  società. Dato che gli effetti di qualsiasi forma sono sempre subliminali,
  molte persone di lettere sembrano risentirsi dell'idea di essere state
  manipolate inconsciamente dal loro medium preferito. Tuttavia, dato che oggi
  ci sono molti media in competizione con la letteratura per la manipolazione
  della coscienza, diventa obbligatorio distinguere tra amici e nemici del
  libro. Ho scoperto, nel fare questo, che il solo suggerire che il libro abbia
  qualsiasi effetto subliminale viene considerato come essere nemici del libro.
  (Il cosiddetto contenuto del libro è il lettore, come Baudelaire sottolineò
  nel suo “hypocrite lecteur, mon semblable, mon frère.”) Quelle persone che
  pensano che io sia un nemico del libro semplicemente non hanno letto il mio
  lavoro, né riflettuto sul problema. La maggior parte della mia scrittura è
  satira menippea,* che presenta la superficie attuale del mondo in cui viviamo
  come un'immagine ridicola.

  *La Satire Ménippée (rif. al satirico greco Menippo) era un libello politico
  francese che circolava a Parigi negli anni 1590. \cite{mcluhan87}
}»

Questo \emph{pòlemos} articolativo e non distruttivo tra \emph{Bremonte} (la
\emph{parola-strumento} d'invenzione, il “monte che freme”) e Menippo (la
“tradizione che frena”) svela «la solitudine essenziale» in cui Bertoncini
costruisce e abita, che Blanchot impagina come condizione
specifica del lavoro artistico \cite{blanchot82}, e che Engelberg analizza come
discorso interno «tra il Sé e il (Sé come) Altro» \cite{engelberg01}.

Se seguissimo la traccia de «il cosiddetto contenuto del libro è il lettore» ci troveremmo senza mezzi:
 il libro, il testo, il dialogo, l'oggetto sonoro,
il concerto: \emph{cosa sono?} \emph{Quali sono i contenuti delle opere di Bertoncini?} Non resta che l'azzardo della definizione di qualche lettore,
interprete, musicista, dialogante\ldots~la «separazione» dall'opera attivata
dall'incantesimo del fascino si proietta nell possibilita di un enigma che prende
forma nel contenuto desiderato di un destinatario.

%-------------------------------------------------------------------------------
%\section*{l'attesa}
%-------------------------------------------------------------------------------

\begin{quote}
  \begin{sf}
    \small
    \begin{description}
      \item[Bremonte:] Mi viene in mente, dicevo, una confidenza che X. m’ha
      fatto su quell’epoca eroica di ricerca sonora.

      Era l’inverno del 1973; egli si trovava a Berlino ospite del
      Künstlerprogramm del D.A.A.D. Per la prima volta nella sua vita il mio
      amico ebbe l’agio di allestire un vero e proprio laboratorio artigiano,
      adatto alla realizzazione delle idee che già da alcuni anni lo spingevano
      verso un certo tipo di costruttivismo musicale, ma che per ragioni
      pratiche, a Roma, non potevano avere adeguato sviluppo.

      [\ldots]

      % Tornato dunque nel proprio studio dopo la visita romana al Guiducci,
      % ancora sotto l’impressione vivissima di quelle sonorità particolari anche
      % se appena udibili, sonorità che non avevano come modo d’attacco niente in
      % comune con i mezzi di produzione sonora corrente, egli si accinse alla
      % costruzione d’una prima arpa eolia dalla cornice metallica rettangolare di
      % cm. 80 x 60, cui seguì immediatamente una seconda di dimensioni poco
      % maggiori e una serie di \emph{gong eolici} (così li chiamò in un primo
      % tempo e quel nome rimase); cioè un’installazione basata su spirali e barre
      % metalliche di acciaio, di bronzo e di ottone, messe in vibrazione
      % esclusivamente da getti d’aria compressa.

      Ma io non sono in grado di rendere nemmeno approssimativamente la tensione
      che egli riesce a comunicare quando evoca quei momenti. Nella descrizione
      della fase finale nella costruzione della prima arpa eolia, e delle prove
      ripetute d’un sistema d’amplificazione da lui escogitato per rendere
      chiaramente percettibili quelle sonorità ricchissime ma fondamentalmente
      al limite della audibilità, egli è capace di ricreare e di evocare
      nell’immaginazione dell’ascoltatore la stessa atmosfera gravida di attesa
      che circonda il lavoro febbrile di costruzione della campana nel film di
      A. Tarkowskij, ANDREIJ RUBLOV.
    \end{description}
  \end{sf}
\end{quote}

È l'ottavo capitolo \cite{tarkovskij1966}, come un'ottava immagine sacra, di una processione verso un'unità
suggerita. Il momento in cui la campana inizia a suonare risulta un
punto decisivo: vi si cela il senso della creazione artistica. \emph{Bremonte},
ad un passo dal raccontare quell'«epoca eroica di ricerca sonora», fa un
passo di lato: così come nel momento del primo suonare della
campana si salta dall'oggetto al senso (si evita la visione del rintocco), \emph{Bremonte} “taglia, cambia inquadratura”: «non sono in grado». La «tensione
che egli riesce a comunicare quando evoca quei momenti» è dicibile solo mediante
un'allegoria, cifrando il senso in un altro senso, perché l'oggetto allegorico è
già di per sé un'allegoria. Per provare a comprendere cosa egli voglia dire con «la
stessa atmosfera gravida di attesa che circonda il lavoro febbrile di
costruzione della campana» è necessario attraversare tutti i livelli simbolici
intercettati.

\begin{quote}
  Per noi era il simbolo dell'audacia, nel senso che la creazione richiede
  dall'uomo l'offerta completa del proprio essere. Che si voglia volare prima
  che sia diventato possibile, o fondere una campana senza aver imparato a
  farlo, o dipingere un'icona -- tutti questi atti esigono che, per il prezzo della
  sua creazione, l'uomo debba morire, dissolversi nel proprio lavoro, dare se
  stesso interamente. Questo è il significato del prologo -- l'uomo volò e per
  questo sacrificò la sua vita.» \cite{ciment69}
\end{quote}

Questa filosofia suggerisce che la formazione artistica può fornire competenza
tecnica, ma la scintilla creativa essenziale deve essere accesa attraverso la
fede, l'intuizione e la completa dedizione al lavoro stesso.  Questa comprensione
ha implicazioni profonde per la pratica artistica: non l'auto-espressione, la
chiamata più alta dell'artista diventa l'auto-trascendenza, creando condizioni
entro cui le forze inconsce possano operare attraverso il “mestiere disciplinato”.

Forse l'intuizione più radicale di Tarkovskij riguarda il primato della fede
sulla conoscenza tecnica nella creazione artistica autentica. Come afferma in
\emph{Scolpire il tempo}

\begin{quote}
  Perché egli sia consapevole che una sequenza di tali azioni è dovuta e giusta,
  che risiede nella natura stessa delle cose, deve avere fede nell'idea, poiché solo
  la fede interconnette il sistema di immagini [\ldots] Nell'Arte, come nella
  religione, l'intuizione equivale alla convinzione, alla fede. [\ldots] Potremmo
  definirlo come scolpire nel tempo. Proprio come uno scultore prende un pezzo di
  marmo e, interiormente cosciente dei lineamenti del suo pezzo finito, rimuove tutto
  ciò che non ne fa parte -- così il cineasta, da un “pezzo di tempo” fatto di un enorme,
  solido aggregato di fatti viventi, taglia e scarta tutto ciò di cui non ha bisogno.
\end{quote}

È nelle trasduzioni tra tempi cronologici e kairotici che l'asse Bertoncini
Tarkovskij attiva un segno comune: Tarkovskij scolpisce immagini nel tempo, che
solo il tempo può animare; Bertoncini scolpisce il tempo nel vento, che\ldots

\begin{quote}
Poco dopo, nel 1970, studiavo composizione con lui a Roma. Era davvero come
essere apprendista in una bottega del ‘500. \cite{nottoli2019}
\end{quote}

Mario Bertoncini sta all'aria compressa come Prometeo sta al fuoco. Mario era un
uomo di Bottega. Il Laboratorio è il momento di \emph{catastrofe}\footnote{%
  Dal greco \textgreek{καταστροφή} (\emph{katastrophḗ}), composto da
  \textgreek{κατά} (\emph{katá}, "giù, verso il basso") e \textgreek{στροφή}
  (\emph{strophḗ}, "voltare, girare"): propriamente "rivolgimento",
  "capovolgimento". In ambito teatrale indica il momento risolutivo del dramma;
  in matematica, la teoria delle catastrofi di René Thom \cite{thom1972} studia
  i cambiamenti qualitativi improvvisi nei sistemi dinamici.
} in cui egli ha
addomesticato l'aria. Il laboratorio di Mario è il pensiero, come in Schoenberg,
come in Nono.

Le opere di Mario, oggi, dopo Mario, appaiono come \emph{botteghe di suono}:
elementi udibili di un assoluto. Il progetto «bottega» che ha mosso la sua
intera esistenza è presente, oggi, in ogni traccia della sua eredità. È nostro
dovere vivere quei luoghi, dialogare con i suoi personaggi, tenere in ordine i
suoi strumenti, la sua Bottega.

\null\vfill
\begin{verbatim}
Come onorare la musica
ascoltarla senza agire
senza sequenze di commenti
come tacere non di fronte
ma dentro la musica
come lasciare che mi disfi
il cuore e mi scombini il pensiero.
Un paesaggio sonoro
ci affronta ovunque e ci invita:
essere musica è fuga dal rumore
è orecchio sinfonico
verso la voce del silenzio
verso la conta di quello che resta.
\end{verbatim}
