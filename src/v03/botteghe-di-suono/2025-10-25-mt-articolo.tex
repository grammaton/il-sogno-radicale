%!TEX TS-program = xelatex
%!TEX encoding = UTF-8 Unicode

\documentclass{../../../lib/gs}

\usepackage{../../../lib/apf}
\newfontfamily\phonfont{Charis} % font ricco di simboli IPA

% Packages per figure e grafici
\usepackage{pgfplots}
\pgfplotsset{compat=1.18}
\usepackage{tikz}
\usetikzlibrary{decorations.text}
\usepackage{subcaption}

\usepackage{enumitem}

\title{il sogno radicale}
\subtitle{autobiografia di un eretico. appunti revb.\\[1cm] \textbf{\large botteghe di suono}}
\author{Giuseppe Silvi}
\date{\today}

% Definizione delle parole chiave per i metadati PDF
\kuno{filosofia}
\kdue{greco antico}
\ktre{Platone}
\kquattro{etimologia}
\kcinque{XeLaTeX}

% Comandi per la notazione AllPass
\newcommand{\apf}[3]{\textbf{\texttt{#1(#2,#3)}}}
\newcommand{\apfc}[3]{\begin{center}\apf{#1}{#2}{#3}\end{center}}
\newcommand{\apfcn}[4]{\begin{center}\apf{#1}{#2}{#3}\footnote{#4}\end{center}}

\usepackage[
  backend=biber,
  style=authoryear,        % oppure numeric se preferisci numeri
  sorting=nyt,             % ordina per: nome, anno, titolo
  maxbibnames=3,           % max 3 autori, poi "et al."
  maxcitenames=2,          % in citazione max 2 nomi
  giveninits=false,        % nomi completi, non iniziali
  uniquename=false,
  uniquelist=false,
  url=true,                % mostra URL
  doi=false,               % NON mostrare DOI
  isbn=false,              % NON mostrare ISBN
  eprint=false             % NON mostrare eprint
]{biblatex}

% Personalizzazioni aggiuntive
\DeclareFieldFormat{url}{\newline\footnotesize\url{#1}}
\DeclareFieldFormat{howpublished}{\newline\footnotesize#1}
\DeclareFieldFormat{urldate}{accesso: #1}   % formato data accesso

% Rimuovi completamente i campi che non servono
\AtEveryBibitem{%
  \clearfield{issn}%
  \clearfield{isbn}%
  \clearfield{doi}%
  \clearfield{eprint}%
  \clearfield{eprinttype}%
  \clearfield{pages}% rimuovi se non vuoi vedere le pagine
}

% Per rimuovere "In:" prima del nome della rivista
\renewbibmacro{in:}{}

% Rendi \cite equivalente a \parencite
\let\cite\parencite

% Nel preambolo del .tex
\addbibresource{../../bibliografia.bib}

\usepackage{soul}

\begin{document}

\maketitle

%-------------------------------------------------------------------------------
% ARTICLE PROPOSAL: Technology Mediation in Contemporary Opera
% Focus: Mario Bertoncini's Sound Workshops
%-------------------------------------------------------------------------------

%-------------------------------------------------------------------------------
\section{Introduction: The Workshop as Living Laboratory}
%-------------------------------------------------------------------------------

\begin{quote}
  \begin{sf}
    \small
    It was really like being an apprentice in a 16th-century workshop. \cite{nottoli2019}
  \end{sf}
\end{quote}

The preservation of technology-mediated operatic works presents a fundamental 
tension between two temporalities: the \emph{opera} as discrete sign inscribed 
in the past, and the \emph{operazione} as continuous process oriented toward 
the future\footnote{See: 
\url{https://github.com/grammaton/il-sogno-radicale/blob/bf51bade8fa8ea7be60f7df9d57efe19a7c4a1d9/src/v03/opera-operazione/2025-10-26-opera-operazione-impaginato.pdf}}.
This tension becomes particularly acute in the case of Mario Bertoncini's 
aeolian sculptures, where the technological mediation is not merely an 
amplification system but constitutes the very condition of possibility for 
musical experience.

Bertoncini's works exist today as \emph{botteghe di suono} (sound workshops): 
audible elements of an absolute, material traces of an unrealized utopia. The 
term \emph{bottega} deliberately evokes the Renaissance workshop model, where 
knowledge transmission occurred through direct practice, where the boundary 
between master craftsman and artist remained porous, and where tools themselves 
embodied accumulated wisdom. This paper examines how the concept of 
\emph{bottega} can inform contemporary approaches to preserving and 
re-activating technology-mediated operatic repertoire, using LAZZARO's 
2025 performance project as a methodological case study.

The central thesis articulates as follows: preservation of Bertoncini's works 
cannot proceed through museum-like crystallization but requires active 
\emph{maintenance} as interpretative practice. The workshop becomes not merely 
a site of conservation but a \emph{Laboratorium} in the alchemical sense---a 
space where continuous material transformation embodies poetic decision-making.

%-------------------------------------------------------------------------------
\section{The Electroacoustic Chain as Unfinished Text}
%-------------------------------------------------------------------------------

\begin{quote}
  \begin{sf}
    \small
    \begin{description}
      \item[Menippo:] As far as I know, however, the crystal microphone offers 
      a limited frequency response, even deficient in the low regions.
      \item[Bremonte:] Exactly. This is precisely the reason for my friend's 
      repeated attempts, for his long sonic research not yet concluded. 
      \cite{bertoncini07}
    \end{description}
  \end{sf}
\end{quote}

\subsection{Historical Context: The Piezoelectric Problem}

Bertoncini's technical writings reveal a persistent preoccupation with the 
transduction problem: how to render perceptible sounds that exist ``richly 
complex but fundamentally at the threshold of audibility.'' His solution---
adapting piezoelectric crystal microphones to mobile bridges of varying 
thickness and materials (wood, plastic, cardboard, metal)---represented both 
an elegant acoustical solution and an admission of technical limitation.

The piezoelectric pickup system functioned according to a dual principle:
\begin{enumerate}
  \item The strings perform acoustic \emph{production} of sound through aeolian 
  excitation
  \item The same strings simultaneously function as \emph{amplification} 
  resonance chamber
  \item Electronic amplification occurs from the bridge, where the piezoelectric 
  crystals transduce mechanical vibration into electrical signal
\end{enumerate}

However, Menippo's objection identifies the critical weakness: piezoelectric 
transducers exhibit severe frequency response limitations, particularly in low 
frequency regions. This technical constraint determined not merely timbral 
characteristics but fundamental compositional possibilities. The ``unfinished 
research'' that Bertoncini acknowledged becomes, in this light, a hermeneutic 
opening: the work explicitly contains its own incompleteness as aesthetic 
principle.

\subsection{Contemporary Interventions: New Pickup Strategies}

LAZZARO's 2025 reconstruction confronted this incompleteness as both technical 
and philosophical challenge. The preparation for \emph{Ripresa di Venti} 
required developing a contemporary electroacoustic chain while maintaining 
fidelity to Bertoncini's sonic world. Several strategic decisions emerged:

\paragraph{Pickup Selection and Placement}
The team experimented with multiple transduction technologies:
\begin{itemize}
  \item Contact piezoelectric pickups (maintaining historical continuity)
  \item Electromagnetic pickups (addressing low-frequency deficiency)
  \item MEMS accelerometers (capturing structural vibrations)
  \item Condenser microphones in near-field configuration (capturing aerodynamic 
  turbulence)
\end{itemize}

Each technology privileges different aspects of the sonic phenomenon. The 
piezoelectric crystals respond primarily to direct mechanical stress at the 
bridge; electromagnetic pickups sense string motion in magnetic field; MEMS 
devices register structural resonances of the entire frame; condenser 
microphones capture acoustic radiation and air turbulence around the strings.

\paragraph{Signal Path Documentation}
The complete electroacoustic chain for the 2025 performances:

\texttt{
  Bridge vibration → Pickup → Preamplifier → [optional: EQ/dynamics] → 
  Spatial diffusion system
}

Critical parameters requiring documentation:
\begin{itemize}
  \item Pickup impedance matching and loading effects
  \item Preamplifier gain structure and noise floor
  \item Equalization curves applied (if any) and their rationale
  \item Compression/limiting thresholds (if used)
  \item Spatial diffusion strategies (point source vs. distributed)
\end{itemize}

\subsection{Philological Questions: Fidelity and Evolution}

The technical interventions raise profound hermeneutic questions: What 
constitutes ``fidelity'' to a work whose creator explicitly acknowledged its 
incompleteness? Three possible positions emerge:

\begin{enumerate}
  \item \textbf{Archaeological fidelity}: Reconstruct the 1973-2008 
  electroacoustic chain as closely as possible, accepting its limitations as 
  historically authentic
  \item \textbf{Intentional fidelity}: Complete Bertoncini's ``unfinished 
  research'' by solving the technical problems he identified, thereby realizing 
  the work he intended
  \item \textbf{Evolutionary fidelity}: Treat the work as living system that 
  continues to develop, incorporating contemporary technologies while 
  maintaining essential sonic characteristics
\end{enumerate}

LAZZARO's approach navigates between these positions, recognizing that 
Bertoncini himself operated in this liminal space. His continuous modifications 
to bridge materials, pickup placements, and amplification strategies suggest 
that the work exists not as fixed object but as \emph{research program}---a 
set of constraints and possibilities within which each realization constitutes 
a new experiment.

\subsection{The Threshold of Audibility as Aesthetic Principle}

The technical challenge of amplifying barely audible phenomena reveals itself 
as aesthetic necessity rather than mere limitation. The listener's perceptual 
attention becomes heightened precisely because the sounds exist at the edge of 
perception. The electroacoustic chain does not simply make sounds louder; it 
creates a condition of heightened listening, a phenomenological reduction that 
brackets quotidian acoustic experience.

This principle connects to Bertoncini's broader aesthetic project: the 
revelation of natural enigmas through technological mediation. The aeolian 
harps do not produce ``new'' sounds but render audible sonic worlds that always 
existed but remained inaccessible to unmediated perception. Technology becomes 
not prosthesis but \emph{revelation}.

%-------------------------------------------------------------------------------
\section{Maintenance as Interpretative Practice}
%-------------------------------------------------------------------------------

\begin{quote}
  \begin{sf}
    \small
    It is our duty to inhabit those places, to dialogue with his characters, 
    to keep his instruments in order, his Workshop. \cite{silvi2025botteghe}
  \end{sf}
\end{quote}

\subsection{Beyond Museum Conservation}

Traditional approaches to instrument preservation emphasize stasis: climate 
control, minimal handling, preventive conservation. This model, appropriate for 
historical artifacts, proves inadequate for Bertoncini's sculptures. These 
instruments were designed for continuous use, for experimental modification, 
for ongoing dialogue between maker and performer. Their preservation paradoxically 
requires their activation.

The concept of \emph{maintenance} therefore shifts from conservation to 
\emph{cura}---care in the sense of both preservation and cultivation. Each 
intervention in the workshop becomes simultaneously:
\begin{itemize}
  \item Technical necessity (replacing corroded strings, adjusting bridge 
  pressure, recalibrating air nozzles)
  \item Interpretative decision (choice of string material affects harmonic 
  content, bridge placement modifies resonance characteristics)
  \item Transmission of knowledge (documenting why certain adjustments produce 
  specific sonic results)
\end{itemize}

\subsection{Material Interventions: Case Studies}

\paragraph{String Replacement and Tensioning}
The circular aeolian harp contains approximately 1,200 metal strings. Over time:
\begin{itemize}
  \item Oxidation alters mass and flexibility
  \item Plastic deformation reduces tension
  \item Individual strings break, creating gaps in the spectral field
\end{itemize}

String replacement requires decisions about:
\begin{itemize}
  \item Material: steel, bronze, brass? Original specifications vs. contemporary 
  availability
  \item Gauge: maintaining original diameter vs. optimizing for contemporary 
  pickups
  \item Tensioning: uniform tension vs. graduated to create specific spectral 
  distributions
\end{itemize}

Each decision affects not merely timbre but the fundamental microintervallique 
density that Bertoncini sought. The workshop becomes site of continuous 
negotiation between historical information and contemporary sonic imagination.

\paragraph{Bridge Adjustment and Materials}
Bertoncini fabricated mobile bridges ``of various thicknesses and different 
materials (wood, plastic, cardboard, metal).'' The bridges perform multiple 
functions:
\begin{itemize}
  \item Mechanical: transmitting string vibration to resonant body
  \item Acoustic: defining vibrational nodes and anti-nodes
  \item Electronic: serving as pickup mounting surface
\end{itemize}

LAZZARO's preparation involved systematic experimentation:
\begin{itemize}
  \item Documenting existing bridge configurations
  \item Testing alternative materials for specific sonic goals
  \item Developing reproducible bridge fabrication protocols
  \item Creating notation system for bridge placement in scores
\end{itemize}

This research constitutes not deviation from Bertoncini's practice but 
continuation of it. His own ``repeated attempts'' and ``unfinished research'' 
establish precedent for ongoing experimental modification.

\paragraph{Compressed Air System: Pressure, Flow, Nozzle Design}
Mario Bertoncini stands to compressed air as Prometheus to fire. The workshop's 
pneumatic system requires continuous attention:

\begin{itemize}
  \item \textbf{Pressure regulation}: 2 bar (as specified for \emph{Istantanee II}) 
  vs. variable pressure for dynamic control
  \item \textbf{Flow rate}: continuous vs. pulsed air streams
  \item \textbf{Nozzle design}: diameter, angle, distance from strings
  \item \textbf{Temperature effects}: compressed air cooling affects metal 
  resonance
\end{itemize}

The air system embodies the work's conceptual core: the domestication of wind, 
the transformation of wild air into artistic instrument. Each adjustment to 
pressure or nozzle angle represents interpretative choice about the character 
of this domestication---whether the air should whisper or roar, caress or 
assault the strings.

\subsection{Documentation as Maintenance Practice}

Every intervention in the workshop generates documentation:
\begin{itemize}
  \item \textbf{Photographic}: before/after states, detail shots of 
  modifications
  \item \textbf{Schematic}: technical drawings of bridge designs, air system 
  configurations
  \item \textbf{Audio}: recordings of sonic results from different setups
  \item \textbf{Written}: rationale for decisions, problems encountered, 
  solutions discovered
\end{itemize}

This documentation serves multiple futures:
\begin{enumerate}
  \item \textbf{Immediate}: Supporting current performance preparation
  \item \textbf{Medium-term}: Enabling future restorations and performances
  \item \textbf{Long-term}: Providing archaeological evidence for 22nd-century 
  scholars/performers
\end{enumerate}

The workshop thus becomes not static repository but \emph{living archive}, 
accumulating layers of interpretative knowledge embedded in material 
modifications and their documentation.

%-------------------------------------------------------------------------------
\section{Notating the Wind: The Problem of the Score}
%-------------------------------------------------------------------------------

\begin{quote}
  \begin{sf}
    \small
    The composer had constructed the perfect music stand on which air, for the 
    first time, could read itself. \cite{silvi2025lazzaro}
  \end{sf}
\end{quote}

\subsection{Three Paradigms: Istantanee I, II, III}

The three \emph{Istantanee} (Snapshots) present an evolutionary trajectory in 
notation strategy:

\paragraph{Istantanee I (1995): The Notation of Absence}
Trebnitz, summer 1995. A circular frame of 1,200 metal strings exposed to wind. 
The score consists of:
\begin{itemize}
  \item Instrument specifications (frame dimensions, string count, material)
  \item Site selection (park of Trebnitz castle)
  \item Date and duration (unspecified---``until the wind stops'')
  \item Recording protocol (chronicler notes: ``14:27 a dog barked in the 
  distance, 14:31 a flock of birds crossed the sky, 14:33 a beetle struck the 
  string bed'')
\end{itemize}

The work's ``notation'' thus consists of \emph{conditions} rather than 
prescriptions. The performer is the wind itself; the score merely establishes 
the possibility-space within which wind can manifest its voices. This represents 
an extreme position: notation as pure potentiality, the instrument as 
``library of all possible voices of the unannotatable wind.''

\paragraph{Istantanee II (2006): The Notation of Memory}
A decade later, the same circular harp awaits in a concert hall. Now a human 
``performer'' interrogates the harp using compressed air at 2 bar, fans, and 
breath. The score specifies:
\begin{itemize}
  \item Proportional temporal structures based on golden ratio calculations 
  derived from the 1995 Trebnitz wind duration
  \item Reference to Ovid's Echo myth: ``the voice remains; they say the bones 
  took the form of stone''
  \item Performance instructions regarding air pressure, nozzle angles, 
  approximate durations
\end{itemize}

The score functions as \emph{mnemonic device}---it recalls what the wind has 
already spoken. The performer reads not prescriptive commands but traces, 
archaeological evidence of past sonic events. The notation mediates between 
the absent wind (1995) and the present performer (2006).

\paragraph{Istantanee III (2008): The Notation of Metamorphosis}
Rome, 2008. The harp, the performer, real-time electroacoustic elaboration. 
The score now contains:
\begin{itemize}
  \item Performance instructions from Istantanee II
  \item Signal processing algorithms and parameter ranges
  \item Temporal coordination between acoustic and electronic layers
  \item Spatial diffusion strategies
\end{itemize}

The notation becomes explicitly multi-layered: acoustic score + electronic 
score + spatial score. Yet this complexity maintains connection to the original 
condition: all layers ultimately respond to the aeolian principle, the wind 
reading itself through technological mediation.

\subsection{LAZZARO's Notation Strategies}

\emph{Ripresa di Venti} (2025) required developing notation adequate to:
\begin{itemize}
  \item Five performers
  \item Three historical aeolian sculptures (Kathedrale, Circular Harp, large 
  Spiral)
  \item Diverse air sources (compressed air, fans, breath)
  \item Spatial distribution of instruments and performers
  \item Approximately 20-minute duration
\end{itemize}

The notation system developed combines multiple representational strategies:

\paragraph{Temporal Structure}
Rather than metric notation (measures, beats), the score employs:
\begin{itemize}
  \item \textbf{Duration blocks}: Flexible time segments with approximate 
  durations
  \item \textbf{Cue points}: Synchronization moments triggered by acoustic 
  events rather than clock time
  \item \textbf{Density curves}: Graphical representation of overall activity 
  level
\end{itemize}

This approach acknowledges the fundamental non-metric character of aeolian 
phenomena. Wind does not organize itself in regular pulsation; the notation 
should not impose metric regularity alien to the material.

\paragraph{Action Notation}
For each performer/instrument combination, the score specifies:
\begin{itemize}
  \item \textbf{Air source}: compressed air at X bar, fan at Y setting, breath, 
  natural wind (if outdoor)
  \item \textbf{Point of application}: Which region of string bed, which bars 
  of metal spiral
  \item \textbf{Gesture quality}: Sustained, pulsed, sweeping, localized
  \item \textbf{Approximate intensity}: From pppp (barely touching) to ffff 
  (maximum pressure)
\end{itemize}

Critically, the notation represents \emph{actions} rather than \emph{results}. 
The actual sounds produced depend on countless variables: string tension, 
ambient humidity, previous performance history of the instrument, microscopic 
variations in air pressure. The score creates conditions for sonic emergence 
rather than determining specific outcomes.

\paragraph{Spatial Notation}
The three sculptures occupy specific positions in performance space. The score 
includes:
\begin{itemize}
  \item Ground plan showing instrument and performer locations
  \item Movement paths for mobile performers
  \item Acoustic zones (where each instrument is primarily audible)
  \item Listening positions (where audience focus should be directed)
\end{itemize}

Spatial distribution becomes compositional parameter: the \emph{Kathedrale} 
produces dense harmonic fields, the \emph{Circular Harp} creates microintervallique 
continuums, the \emph{Spiral} generates inharmonic metallic resonances. Their 
positioning determines possibilities for acoustic interference patterns, 
spatial counterpoint, timbral blending.

\subsection{Notation as Hermeneutic Opening}

The notational strategies developed for Bertoncini's works reveal notation's 
dual function:
\begin{enumerate}
  \item \textbf{Prescriptive}: Transmitting composer's intentions to performers
  \item \textbf{Descriptive}: Documenting what occurred in previous realizations
\end{enumerate}

Traditional Western notation emphasizes the prescriptive function: the score 
as command, the performer as executor. Bertoncini's works suggest alternative 
model: notation as \emph{hermeneutic opening}, as question posed to material 
conditions. The score asks: ``What voices can emerge when air encounters these 
metal strings under these conditions?'' Each performance constitutes a response, 
never definitive, always provisional.

This understanding has profound implications for preservation. The ``authentic'' 
performance cannot mean reproduction of some ur-text but rather faithful 
engagement with the hermeneutic question the work poses. Notation preserves 
not fixed sonic surface but the \emph{condition of questioning}.

%-------------------------------------------------------------------------------
\section{Documentation as Archaeology of the Future}
%-------------------------------------------------------------------------------

\subsection{Multi-Modal Recording Strategies}

LAZZARO's 2025 performances generated comprehensive documentation:

\paragraph{Audio Recording}
Multiple simultaneous recordings:
\begin{itemize}
  \item \textbf{Pickup feeds}: Direct recordings from each instrument's 
  transduction system (preserving pure signal before amplification)
  \item \textbf{Near-field microphones}: Capturing acoustic radiation from 
  each sculpture
  \item \textbf{Ambience microphones}: Room acoustics and acoustic interactions 
  between instruments
  \item \textbf{Binaural recording}: Simulating listener perspective from 
  specific audience position
\end{itemize}

This multi-track approach enables:
\begin{itemize}
  \item Post-performance analysis of individual instrument contributions
  \item Alternative mixing strategies for different presentation contexts
  \item Comparison between transduced and acoustic sound
  \item Spatial reconstruction through ambisonic rendering
\end{itemize}

\paragraph{Video Documentation}
Multi-camera setup capturing:
\begin{itemize}
  \item \textbf{Performer actions}: Gesture, body position, air source 
  manipulation
  \item \textbf{Instrument response}: String vibration patterns, frame 
  resonances (rendered visible through high-speed capture and macro 
  cinematography)
  \item \textbf{Air dynamics}: Using smoke or light scattering to visualize 
  airflow patterns
  \item \textbf{Audience response}: Listening behaviors, spatial positioning
\end{itemize}

The video archive serves multiple purposes:
\begin{itemize}
  \item Pedagogical: Teaching future performers gestural techniques
  \item Analytical: Correlating actions with sonic results
  \item Artistic: Creating standalone video works
  \item Archaeological: Preserving evidence of 2025 performance practice
\end{itemize}

\paragraph{Parametric Data Logging}
Beyond audio/video, systematic logging of:
\begin{itemize}
  \item Air pressure readings (logged via digital pressure sensors)
  \item Environmental conditions (temperature, humidity, barometric pressure---
  all affecting acoustic behavior)
  \item Performer locations (via motion capture or manual notation)
  \item Technical issues and solutions (troubleshooting log)
\end{itemize}

This data constitutes meta-documentation: information about the conditions of 
documentation itself, enabling future users to interpret the primary audio/video 
materials with full contextual understanding.

\subsection{Open Formats and Long-Term Accessibility}

The documentation adheres to open standards:

\paragraph{Audio Formats}
\begin{itemize}
  \item Master recordings: FLAC (lossless compression) at 96kHz/24bit
  \item Distribution versions: Multiple formats (FLAC, WAV, MP3) for different 
  use cases
  \item Metadata: Embedded BWF (Broadcast Wave Format) chunks containing full 
  technical and descriptive information
\end{itemize}

\paragraph{Video Formats}
\begin{itemize}
  \item Master: Open-source codec (FFV1 in Matroska container)
  \item Distribution: H.264/H.265 with open-source encoding
  \item Metadata: Embedded technical, descriptive, and rights information
\end{itemize}

\paragraph{Documentation Text and Graphics}
\begin{itemize}
  \item Text: Markdown and LaTeX sources (plain text, version-controllable)
  \item Graphics: SVG for vector graphics, PNG for raster images
  \item Data: CSV and JSON for tabular and structured data
  \item Code: All signal processing algorithms documented in open-source 
  languages (Python, Faust, Csound)
\end{itemize}

\paragraph{Repository Structure}
All materials organized in Git repository with clear directory structure:

\begin{verbatim}
bertoncini-archive/
├── instruments/
│   ├── circular-harp/
│   │   ├── specifications/
│   │   ├── maintenance-logs/
│   │   └── photos/
│   ├── kathedrale/
│   └── spiral/
├── performances/
│   ├── 2025-09-22-lazzaro-rome/
│   │   ├── audio/
│   │   │   ├── masters/
│   │   │   └── distribution/
│   │   ├── video/
│   │   ├── scores/
│   │   ├── parametric-data/
│   │   └── documentation/
├── research/
│   ├── electroacoustic-chain/
│   ├── notation-systems/
│   └── interviews/
└── README.md
\end{verbatim}

\subsection{Creative Commons Licensing}

The documentation employs tiered licensing:
\begin{itemize}
  \item \textbf{Technical documentation}: CC BY-SA 4.0 (requiring attribution 
  and share-alike)
  \item \textbf{Performance recordings}: CC BY-NC-SA 4.0 (non-commercial, 
  requiring attribution and share-alike)
  \item \textbf{Musical scores}: CC BY-NC-ND 4.0 (preserving performance 
  practice integrity while allowing distribution)
\end{itemize}

This licensing strategy balances multiple concerns:
\begin{itemize}
  \item Honoring Bertoncini's estate and legacy
  \item Enabling scholarly and educational use
  \item Preventing commercial exploitation
  \item Ensuring future accessibility
\end{itemize}

\subsection{The Archive as Score}

The comprehensive documentation constitutes not merely preservation of past 
performances but \emph{score for future realizations}. Future performers in 
2075 or 2125 will encounter this archive as primary text. They will observe:
\begin{itemize}
  \item How we held the air nozzles in 2025
  \item What sounds we privileged through our pickup choices
  \item Which gestural vocabularies we developed
  \item What compromises we made with degraded instruments
\end{itemize}

They will then make their own decisions, continuing the ``unfinished research'' 
that Bertoncini acknowledged and that we have inherited. The archive becomes 
not endpoint but relay point in multigenerational dialogue.

This understanding transforms documentation from bureaucratic necessity to 
artistic practice. Each photograph, each audio file, each technical note 
constitutes a message to unknown future interpreters. We curate this archive 
not for ourselves but for archaeologists who will excavate our sonic world 
long after compressed air has perhaps become obsolete, replaced by technologies 
we cannot yet imagine.

%-------------------------------------------------------------------------------
\section{The ``Toward the Concert'' as Methodology}
%-------------------------------------------------------------------------------

\subsection{Interdisciplinary Convergence}

The preparation for LAZZARO's September 2025 concert constituted research 
process in itself. The ensemble's name encodes this position: LAZZARO, the 
last student of Bertoncini, ``the one from the back row,'' awakens the dead 
instruments not through resurrection but through studious interrogation.

The preparation involved multiple disciplinary perspectives:

\paragraph{Historical Research}
\begin{itemize}
  \item Studying Bertoncini's writings, particularly the dialogic structure 
  of \emph{Arpe Eolie e altre cose inutili}
  \item Analyzing recorded performances from 1995-2008
  \item Interviewing musicians who worked with Bertoncini (Giorgio Nottoli, 
  etc.)
  \item Examining archival photographs of instrument construction
\end{itemize}

\paragraph{Technical Investigation}
\begin{itemize}
  \item Systematic acoustic analysis of the three sculptures
  \item Experimentation with pickup placement and types
  \item Development of air delivery systems
  \item Testing spatial configurations in performance venue
\end{itemize}

\paragraph{Interpretative Exploration}
\begin{itemize}
  \item Developing gestural vocabularies for air manipulation
  \item Discovering sonic possibilities through systematic exploration
  \item Negotiating performance roles among five performers
  \item Creating notation adequate to discoveries
\end{itemize}

\paragraph{Philosophical Reflection}
\begin{itemize}
  \item Interrogating concepts of authenticity and fidelity
  \item Considering the work's ontological status (object vs. process)
  \item Analyzing Bertoncini's dialectical methodology (Bremonte/Menippo)
  \item Situating the project within broader questions of repertoire sustainability
\end{itemize}

These perspectives did not operate sequentially but in continuous dialogue. A 
technical discovery (e.g., electromagnetic pickups revealing previously 
inaudible low frequencies) generated historical questions (Did Bertoncini want 
these frequencies? If technology had permitted, would he have included them?) 
which informed interpretative decisions (Should we feature or minimize these 
frequencies?) which required philosophical consideration (What does ``fidelity'' 
mean for unfinished research?).

\subsection{Rehearsal as Research}

The ensemble conducted rehearsals not as preparation for predetermined 
performance but as structured exploration. Each session followed protocol:

\begin{enumerate}
  \item \textbf{Documentation of starting conditions}: Instrument state, 
  environmental conditions, technical setup
  \item \textbf{Focused experimentation}: Testing specific parameters (air 
  pressure, pickup placement, performer positioning)
  \item \textbf{Systematic recording}: Capturing all trials for subsequent 
  analysis
  \item \textbf{Collective reflection}: Discussion of discoveries, problems, 
  questions
  \item \textbf{Documentation of decisions and open questions}: Maintaining 
  research journal
\end{enumerate}

This methodology yields multiple outputs:
\begin{itemize}
  \item Performance-ready interpretative strategies
  \item Technical documentation of instrument behavior
  \item Comparative recordings for analytical study
  \item Pedagogical materials for future performers
\end{itemize}

\subsection{The Concert as Provisional Synthesis}

The September 22, 2025 performance at Goethe-Institut Rome represented not 
conclusion but \emph{provisional synthesis} of research process. The concert 
fixed---temporarily, for that evening---certain decisions:
\begin{itemize}
  \item These pickup configurations
  \item These air pressures
  \item These gestural approaches
  \item This temporal structure
\end{itemize}

But the research continues. Post-concert analysis revealed:
\begin{itemize}
  \item Acoustic phenomena recorded but not perceived during performance
  \item Alternative strategies that might have been more effective
  \item Technical problems requiring solution
  \item New questions for investigation
\end{itemize}

This cycle---research, synthesis (performance), analysis, renewed research---
mirrors Bertoncini's own practice. His ``repeated attempts'' and ``unfinished 
research'' establish precedent for understanding the work as iterative process 
rather than fixed object.

\subsection{Transmission Beyond Documentation}

The ``toward the concert'' methodology generates knowledge that exceeds 
documentation's capacity. Certain understandings emerge only through embodied 
practice:
\begin{itemize}
  \item The tactile sense of air pressure at different settings
  \item The visual reading of string vibration patterns
  \item The acoustic judgment of when instrument reaches resonance
  \item The ensemble coordination through listening rather than visual cues
\end{itemize}

This \emph{tacit knowledge} \cite{polanyi1966} cannot be fully textualized. It 
requires direct transmission, apprenticeship, the workshop model that Nottoli 
invoked. LAZZARO's practice thus becomes not merely documentation project but 
\emph{living pedagogy}---the ensemble as bottega, maintaining not just 
instruments but practices, not just objects but ways of knowing.

Future performers will need both the archive (documentation) and the lineage 
(embodied transmission). The archive provides information; the lineage provides 
\emph{formation}. Together they constitute preservation adequate to works whose 
essence lies not in fixed score but in ongoing inquiry.

%-------------------------------------------------------------------------------
\section{Sustainability and the Ecology of Practice}
%-------------------------------------------------------------------------------

\subsection{Material Sustainability}

Bertoncini's aeolian sculptures present specific sustainability challenges:

\paragraph{Strings}
\begin{itemize}
  \item Gradual oxidation requiring periodic replacement
  \item Availability of specific gauges and materials
  \item Environmental impact of metal production
\end{itemize}

\paragraph{Compressed Air System}
\begin{itemize}
  \item Energy consumption of air compressor
  \item Noise pollution during performance
  \item Carbon footprint of pneumatic infrastructure
\end{itemize}

\paragraph{Electronic Components}
\begin{itemize}
  \item Pickup longevity and replaceability
  \item Amplification system power requirements
  \item Electronic waste from obsolete equipment
\end{itemize}

The ensemble confronts these challenges through:
\begin{itemize}
  \item Documenting specifications to enable future sourcing
  \item Exploring alternative air sources (manual fans, breath)
  \item Prioritizing repairable over replaceable electronics
  \item Developing skills in lutherie and electronics repair
\end{itemize}

\subsection{Economic Sustainability}

Technology-mediated opera faces economic pressures:
\begin{itemize}
  \item Specialist technical knowledge required
  \item Equipment costs (instruments, amplification, documentation)
  \item Venue requirements (space, acoustic treatment)
  \item Limited audience compared to standard repertoire
\end{itemize}

LAZZARO's model addresses these through:
\begin{itemize}
  \item \textbf{Multi-functionality}: Performers as technicians, researchers, 
  documentarians
  \item \textbf{Collaborative funding}: Combining institutional support 
  (Fondazione Isabella Scelsi, Festival ArteScienza) with grassroots fundraising
  \item \textbf{Open documentation}: Sharing knowledge to reduce redundant 
  research costs for future projects
  \item \textbf{Pedagogical integration}: Embedding research in educational 
  contexts (conservatory partnerships)
\end{itemize}

\subsection{Epistemological Sustainability}

Beyond material and economic concerns lies question of \emph{knowledge 
sustainability}: How to maintain understanding of these works across 
generational gaps?

The challenge intensifies with time:
\begin{itemize}
  \item 2025: Direct students of Bertoncini still active
  \item 2045: Students of students, oral tradition weakening
  \item 2075: Only documentation remains, no direct lineage
  \item 2125: Historical distance comparable to our relation to 1925 avant-garde
\end{itemize}

Each temporal remove requires different preservation strategies:

\paragraph{Near-term (2025-2045)}
\begin{itemize}
  \item Intensive documentation of tacit knowledge
  \item Oral history interviews with all direct collaborators
  \item Establishment of performance traditions and lineages
\end{itemize}

\paragraph{Medium-term (2045-2075)}
\begin{itemize}
  \item Maintenance of instruments and documentation archives
  \item Periodic revival performances to test documentation adequacy
  \item Development of scholarly literature and critical traditions
\end{itemize}

\paragraph{Long-term (2075-2125+)}
\begin{itemize}
  \item Archaeological approach: reconstructing practice from documentation
  \item Acceptance of historical distance and interpretative freedom
  \item Potential for radical re-invention as historical understanding evolves
\end{itemize}

\subsection{Ecological Model: The Workshop as Ecosystem}

The bottega provides model for sustainable practice. Like ecological system, 
the workshop maintains itself through:

\paragraph{Cyclic Processes}
\begin{itemize}
  \item Performance → analysis → modification → rehearsal → performance
  \item Knowledge → practice → discovery → documentation → knowledge
\end{itemize}

\paragraph{Diversity}
\begin{itemize}
  \item Multiple interpretative approaches coexisting
  \item Various technical solutions to similar problems
  \item Different performance contexts and purposes
\end{itemize}

\paragraph{Adaptation}
\begin{itemize}
  \item Responding to changing technological landscape
  \item Incorporating new performance practices
  \item Maintaining core identity while evolving
\end{itemize}

\paragraph{Resilience}
\begin{itemize}
  \item Distributed knowledge (not dependent on single master)
  \item Documented practices (surviving loss of tacit knowledge)
  \item Flexible works (accommodating multiple realizations)
\end{itemize}

This ecological understanding suggests preservation strategy fundamentally 
different from museum model. Rather than attempting to freeze works in 
``original'' state, we cultivate conditions for their continued evolution. The 
goal becomes not crystallization but \emph{sustained vitality}.

%-------------------------------------------------------------------------------
\section{Conclusions: The Duty of the Living}
%-------------------------------------------------------------------------------

\begin{quote}
  \begin{sf}
    \small
    \begin{verse}
      How to honor music\\
      to listen without acting\\
      without sequences of comments\\
      how to be silent not before\\
      but inside music\\
      how to let it undo\\
      the heart and scramble the thought.\\
      A sonic landscape\\
      confronts us everywhere and invites:\\
      to be music is flight from noise\\
      is symphonic ear\\
      toward the voice of silence\\
      toward the count of what remains. \cite{silvi2025botteghe}
    \end{verse}
  \end{sf}
\end{quote}

\subsection{Summary of Findings}

This study has examined preservation and sustainability of Mario Bertoncini's 
technology-mediated works through LAZZARO's 2025 performance project. Key 
findings include:

\paragraph{Technical}
\begin{itemize}
  \item The electroacoustic chain represents unfinished research requiring 
  ongoing development
  \item Maintenance practices constitute interpretative decisions rather than 
  neutral conservation
  \item Documentation must be comprehensive, multi-modal, and format-agnostic 
  for long-term accessibility
\end{itemize}

\paragraph{Methodological}
\begin{itemize}
  \item The ``toward the concert'' process generates knowledge through 
  interdisciplinary convergence
  \item Rehearsal as research produces multiple valuable outputs beyond 
  performance preparation
  \item The workshop model enables transmission of both explicit and tacit 
  knowledge
\end{itemize}

\paragraph{Philosophical}
\begin{itemize}
  \item Fidelity to unfinished works requires evolution rather than 
  crystallization
  \item Notation functions as hermeneutic opening rather than prescriptive 
  command
  \item Preservation means maintaining conditions for questioning, not fixing 
  answers
\end{itemize}

\subsection{Implications for Contemporary Opera}

The Bertoncini case study suggests broader principles for technology-mediated 
opera:

\paragraph{Against Fetishization of Original Technologies}
Historical authenticity cannot mean slavish reproduction of obsolete 
technologies. When original equipment becomes unavailable, contemporary 
alternatives that preserve essential sonic characteristics prove more faithful 
than archaeological reconstruction that produces degraded results.

\paragraph{For Documentation as Creative Practice}
Documentation should not be afterthought but integral component of artistic 
process. The archive constitutes not merely preservation of past but score for 
future.

\paragraph{For Open Knowledge Commons}
Proprietary systems and closed documentation threaten long-term sustainability. 
Open formats, Creative Commons licensing, and transparent technical 
specifications enable broader communities of practice to sustain works beyond 
original creators' lifetimes.

\paragraph{For Workshop Pedagogy}
Conservatory education should integrate practical workshops where students 
engage directly with instruments, electronics, and maintenance practices. 
Technical knowledge cannot remain separated from interpretative knowledge.

\subsection{Future Research Directions}

This study opens multiple avenues for investigation:

\begin{itemize}
  \item \textbf{Comparative studies}: How do preservation strategies differ 
  across technology-mediated works (live electronics, interactive systems, 
  networked performances)?
  \item \textbf{Perceptual studies}: How do audiences experience works 
  performed with updated technologies vs. period equipment?
  \item \textbf{Economic modeling}: What organizational and funding structures 
  best support sustained engagement with challenging repertoire?
  \item \textbf{Notation theory}: Can formal systems be developed for notating 
  complex electro-acoustic interactions?
  \item \textbf{Software archaeology}: How can obsolete software environments 
  be preserved and emulated?
\end{itemize}

\subsection{The Radical Dream}

Mario Bertoncini's legacy resides not in fixed objects but in \emph{radical 
dream}: that wind could become music, that breath could be sculpted, that the 
inaudible could be revealed through technological mediation. This dream remains 
incomplete---as Bertoncini acknowledged, the research continues.

Our duty as inheritors involves neither museum preservation nor abandonment 
but \emph{active stewardship}. We must inhabit the workshop, dialogue with the 
instruments, keep the tools in order. We must ask the questions the works pose, 
knowing answers remain provisional. We must document our attempts so future 
archaeologists can reconstruct not just what we did but why we did it, what we 
hoped for, what remained beyond our reach.

The botteghe di suono await their next inhabitants. The air continues to blow. 
The strings still await excitation. The work---unfinished, as all authentic 
works must be---calls to those willing to take up its questions.

\begin{quote}
  \begin{sf}
    \small
    Born too late to become a Workshop Master, deceased too soon to have 
    LAZZARO in the Workshop. Mario, by bending words, writing interweavings of 
    metal, concealed potential musical forms throughout his life. 
    \cite{silvi2025lazzaro}
  \end{sf}
\end{quote}

It falls to us, now, to unfold those potentials.

%-------------------------------------------------------------------------------
% BIBLIOGRAPHY
%-------------------------------------------------------------------------------

% Note: Bibliography entries to be added based on your existing BibTeX database
% Key references to include:
% - Bertoncini, Mario (2007). Arpe eolie e altre cose inutili
% - Nottoli, Giorgio (2019). Breve riflessione su Mario Bertoncini
% - Blanchot, Maurice. The Space of Literature
% - Adorno, Theodor W. Aesthetic Theory
% - Your own writings on Bertoncini
% - Polanyi, Michael (1966). The Tacit Dimension
% - McLuhan, Marshall. Letters
% - Tarkovskij, Andrej. Sculpting in Time
% - Thom, René. Structural Stability and Morphogenesis

%\bibliographystyle{plain}
%\bibliography{references}


\clearpage

% Nel documento
%\nocite{*}
\printbibliography

\end{document}
