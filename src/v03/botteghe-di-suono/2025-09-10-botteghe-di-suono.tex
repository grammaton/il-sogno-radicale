%-------------------------------------------------------------------------------
\section*{del soffio sconosciuto}
%-------------------------------------------------------------------------------

\begin{quote}
  \begin{sf}
    \small
    \begin{description}
      \item[Bremonte:] Mi viene in mente, dicevo, una confidenza che X. m’ha
      fatto su quell’epoca eroica di ricerca sonora.
    \end{description}
  \end{sf}
\end{quote}

\emph{In che modo la scrittura di Mario Bertoncini partecipa alla comprensione
di ciò che è Bottega, Laboratorio e Opera? In che modo la lettura dei suoi
scritti può svelare un principio di operazione?}

La metodologia dialettica di Bertoncini impone un momento di riflessione
sull'equilibrio \emph{opera-operazione} messo in scena. La sua scrittura è un
teatro dell'intelletto, un'immaginario e, insieme, uno scenario, quindi uno
spazio scenico e un'ambientazione, dove il pensiero si rappresenta attraverso
persone multiple, in una forma possibile di \emph{sympòsion} emergente. Una
vetta di comunicazione indiretta che svetta su una pianura di informazione.

\emph{Bremonte}: è una \emph{parola-strumento}. {\phonfont Bre-}, come brezza,
come breath, come l'espansione in un gonfiare\footnote{Pokorny IEW 146, 167-68,
  169, 170-71, 171-72. {\phonfont bhes-} (soffiare, respirare): radice imitativa
  fondamentale per il vocabolario respiratorio nelle lingue indo-europee;
  {\phonfont bhreu-} (gonfiare, germogliare): radice collegata ai processi di
  espansione e crescita {\phonfont bhreus-} (bollire, suonare): radice con campo
  semantico che include fenomeni acustici e movimento turbolento}.
\emph{Bremonte} è il monte che soffia, il monte che respira. La brezza è il vento
catabatico, fresco, ciclico, il vento soffiato dalla montagna sulla valle.
Bremonte è, come ogni opera di Bertoncini, un enigma naturale, un momento
di realtà. L'esplorazione umana è mossa dalla volontà di decodifica di enigmi,
la realtà del mondo necessità volontà di soluzione, che poi chiamiamo verità,
verità supposte\ldots

%L'uso di \emph{X.} per l'io autobiografico rivela un tratto fondamentale: nel
Nel momento stesso in cui l'artista cerca di catturare la propria esperienza
attraverso la scrittura, diventa \emph{altro} da sé. \emph{X.} non è
semplicemente un espediente stilistico, ma l'incarnazione di quello che Blanchot \cite{fare_ref}
chiama «il neutro» — quella dimensione anonima in cui l'autore si dissolve
nell'atto stesso della creazione. \emph{X.} materializza questa dismissione:
l'io più intimo diventa l'incognita più radicale perché la scrittura trasforma
il soggetto scrivente in oggetto scritto. \emph{X.} come \emph{chi} (χ), “cosa
sconosciuta”, mostra come l'atto autobiografico, anziché rivelare il sé, ne
sveli l'irraggiungibilità.

%Bertoncini non racconta di sé attraverso \emph{X.}, ma piuttosto \emph{diventa X.}.

%-------------------------------------------------------------------------------
\section*{fascino}
%-------------------------------------------------------------------------------

\begin{quote}
  \begin{sf}
    \small
    \begin{description}
      \item[Bremonte:] %Non è esatto.
      Un impulso determinante per il mio amico – quasi traumatico, a detta sua!
    \end{description}
  \end{sf}
\end{quote}

Traumatico è ciò che irrompe nell'ordinario flusso dell'esperienza, creando una discontinuità che il soggetto non riesce immediatamente a integrare. E un evento che eccede la capacità di elaborazione simbolica, lasciando una sorta di "corpo estraneo" nell'economia psichica. Il ferire è un penetrare, l'attraversare, spezzando, una superficie continua.

Verbalizzare la ferita, significa verbalizzare quella dimensione dell'esperienza poetica che fora, penetra la
comprensione concettuale e determina una trasformazione del soggetto.

«Perché il fascino?» si chiede Blanchot? \cite{blanchot82}

\begin{quote}
  \begin{sf}
    \small
    Vedere presuppone la distanza, la decisione separatrice, la facoltà di non essere in contatto e di evitare la confusione del contatto. Vedere significa che tale separazione è, tuttavia, divenuta incontro. Ma cosa succede quando quel che si vede, per quanto distante, sembra toccarvi con un contatto sconvolgente, cosa succede quando il modo di vedere è una sorta di tocco, quando vedere è un contatto a distanza? Quando quel che viene visto si impone allo sguardo, come se lo guardo fosse colto, toccato, messo in contatto con l’apparenza? Non si tratta di un contatto attivo, ciò che resta ancora dell’iniziativa e dell’azione in un contatto vero e proprio, ma lo sguardo è trascinato, come assorbito in un movimento immobile e in un fondo senza profondità. Quel che ci viene dato in un contatto a distanza è l’immagine, e il fascino è la passione dell’immagine.
  \end{sf}
\end{quote}

Il fascino descrive quell'incontro con l'opera dove il soggetto viene come
“destituito” dalla propria posizione di controllo.

\begin{quote}
  \begin{sf}
    \small
    \begin{description}
      \item[Bremonte:] %Non è esatto.
      Un impulso determinante per il mio amico – quasi traumatico, a detta sua! –
      occorse durante una visita ad un piccolo studio romano di ricerca
      elettroacustica il cui tecnico, Guido Guiducci, persona in possesso di
      grande inventiva artigianale,% aveva costruito un oggetto simile alla
      % “chitarra abruzzese”, ma dotato d’un numero infinitamente maggiore di
      % corde metalliche sottilissime. Le corde, vicinissime l’una all’altra,
      % avrebbero dovuto avere una funzione analoga a quella dell’\emph{eco a
      % piastra}, di cui Guiducci voleva dotare lo studio, ma che avrebbe
      % rappresentato uno sforzo notevole per le non floride condizioni economiche
      % di esso. Il tentativo fallì; l’oggetto, prima di finire nel mucchio della
      % spazzatura, servì al Guiducci per una curiosa applicazione: soffiando
      % lievemente sulle corde e registrando l’azione per mezzo d’un microfono
      % sensibilissimo, il tecnico produsse casualmente una sequenza di suoni
      % eolici, una fascia microintervallica di grande complessità spettrale,
      [\ldots] e inconsapevolmente diede l’avvio ad una trentennale ricerca che il mio
      amico misterioso – come ama chiamarlo lei – perseguì con costante
      applicazione esplorando un mondo sonoro inedito le cui possibilità egli, a
      tutt’oggi, pensa di non aver ancora esaurito.
    \end{description}
  \end{sf}
\end{quote}

L'impulso determinante, quasi traumatico, descrive la dimensione del fascino
come evento che destituisce il soggetto dalla sua posizione ordinaria.

L'inconsapevolezza dell'evento fondativo richiama la struttura come esperienza
che precede la coscienza\footnote{Fascino, Blanchot, APT: fascino(esperienza,coscienza).}.

La ricerca che deriva da questo trauma suggerisce come l'enigma non si esaurisca
mai\footnote{Approfondire: l'enigma non si esaurisc mai}. Caratteristica
essenziale del \emph{Rätselcharakter} adorniano \cite{adorno}, l'opera-operazione
autentica contiene possibilità poetiche senza mai consegnarsi continuamente.

Il mondo sonoro  inedito indica una prospettiva ontologica, un'apertura verso
possibilità di esistenza precedentemente inaccessibili. Questo risuona con
l'idea blanchotiana che l'arte apra spazi di alterità radicale.

Con le possibilità\ldots di non aver ancora esaurito Bertoncini mantiene aperto
l'orizzonte dell'inesplorato, preservando quel carattere enigmatico che continua
ad orientare la ricerca.

L'opera esercita una forza di attrazinoe che precede e condiziona ogni possibile
comprensione. È l'esperienza di essere afferrati, prima ancora di afferrare.

Una volta in operazione, l'opera (autentica) mantiene un nucleo non riducibile
al concetto, un “di più” che resiste alla spiegazione. Questo carattere
enigmatico è condizione poetica dell'autenticità poetica.

Questo “trauma fondativo” del primo incontro con l'enigma sonoro è un momento di
vocazione, una chiamata (\emph{vocatio}) che determina un orientamento
(esistenziale). Il fascino \cite{blanchot82} e l'enigma \cite{adorno} convergono
nel descrivere dove la soglia dell'esperienza poetica si fa esperienza di
trasformazione.

%-------------------------------------------------------------------------------
\section*{menippo}
%-------------------------------------------------------------------------------

\begin{quote}
  \begin{sf}
    \small
    \begin{description}
      \item[Bremonte:] Il volume sonoro sviluppato da un fascio di corde che
      vengano eccitate con la lieve pressione dell’aria (dal fiato o dal vento)
      è minimo. Sia l’esperimento del Guiducci che i successivi tentativi
      compiuti da X. al riguardo, forniscono soltanto un’indicazione sulla
      natura e sulla qualità sonora del fenomeno, senza potere in alcun modo
      renderlo adatto a sostenere il peso d’un discorso musicale\ldots
      \item[Menippo:] A non parlare delle esigenze d’una sala da concerto!
      \item[Bremonte:] Appunto. Egli costruì quindi un sistema basato sul punto
      nevralgico di tutti gli strumenti acustici tradizionali: il ponticello.
      Fabbricò dei ponticelli mobili di vario spessore e di materiali diversi
      (legno, plastica, cartone, metallo)\ldots
      [\ldots] di modo che le corde assolvono acusticamente la doppia funzione
      di \underline{\emph{produzione} del suono} e di \underline{\emph{amplificazione} di esso}, cioè costituiscono in
      pari tempo anche una cassa di risonanza. L’amplificazione elettronica
      avviene dal ponticello, su cui X. adatta dei cristalli microfonici
      piezoelettrici.
      \item[Menippo:] Che io sappia, però, il microfono a cristallo offre una
      risposta di frequenza limitata, nelle regioni gravi addirittura manchevole.
      \item[Bremonte:] Giusto. È appunto questa la ragione dei ripetuti
      tentativi compiuti dal mio amico, della sua lunga ricerca sonora non
      ancora conclusa.
    \end{description}
  \end{sf}
\end{quote}

Avanzando in questa fantasiosa analisi dell'architettura dell'enunciazione
bertonciniana, Bremonte è il “monte che freme” ed evoca l'inquietudine creativa,
Menippo\footnote{%
  Menippo innovò la forma prosimetrica, mescolando prosa e versi,
  per discutere argomenti filosofici seri in spirito di ridicolo e satira. Per
  Bakhtin la menippea è un genere “carnevalizzato” che, combinando serietà
  filosofica e comicità, alto e basso, testa idee attraverso situazioni
  estreme, mantenendo sempre un legame con l'attualità del proprio tempo. Esso
  consente l'attivazione di «scene di scandalo, comportamenti eccentrici,
  discorsi e performance inappropriati, cioè tutti i tipi di violazioni del
  corso generalmente accettato e consueto degli eventi e delle norme stabilite
  di comportamento» e «la preoccupazione per questioni attuali e d'attualità»\cite{bakhtin84}\cite{bakhtin84}
}%
è l'indagine classica, richiama la satira menippea, genere dialogico che mescola
serio e giocoso, filosofia e ironia: il dialogo per esplorare contraddizioni
filosofiche. Menippo è la tradizione che attraversa i secoli.\footnote{%
  Marshall McLuhan nella lettera a Michael Hornyansky del  3 febbraio 1976 scrive:

  Caro Professor Hornyansky,
  Uno studente in un seminario ha recentemente citato alcuni suoi commenti
  riguardo ai miei presunti punti di vista sul libro stampato. Ho mandato di
  recente un articolo su questa questione a Newsweek, di cui allego una copia.
  Partendo dagli studi simbolisti, molto tempo fa iniziai a studiare gli effetti
  degli artefatti umani. Il lavoro di Flaubert, per esempio, è interamente
  interessato a questi effetti nel modo in cui essi plasmano la psiche e la
  società. Dato che gli effetti di qualsiasi forma sono sempre subliminali,
  molte persone di lettere sembrano risentirsi dell'idea di essere state
  manipolate inconsciamente dal loro medium preferito. Tuttavia, dato che oggi
  ci sono molti media in competizione con la letteratura per la manipolazione
  della coscienza, diventa obbligatorio distinguere tra amici e nemici del
  libro. Ho scoperto, nel fare questo, che il solo suggerire che il libro abbia
  qualsiasi effetto subliminale viene considerato come essere nemici del libro.
  (Il cosiddetto contenuto del libro è il lettore, come Baudelaire sottolineò
  nel suo “hypocrite lecteur, mon semblable, mon frère.”) Quelle persone che
  pensano che io sia un nemico del libro semplicemente non hanno letto il mio
  lavoro, né riflettuto sul problema. La maggior parte della mia scrittura è
  satira menippea,* che presenta la superficie attuale del mondo in cui viviamo
  come un'immagine ridicola.

  *La Satire Ménippée (rif. al satirico greco Menippo) era un libello politico
  francese che circolava a Parigi negli anni 1590.
}» \cite{mcluhan87}

Questo \emph{polemos} articolativo e non distruttivo tra \emph{Bremonte} (la
\emph{parola-strumento} d'invenzione, il “monte che freme”) e Menippo (la
tradizione che frena) svela «la solitudine essenziale» in cui Bertoncini
costruisce e abita, che Blanchot \cite{blanchot82} impagina come condizione specifica
del lavoro artistico, e che Engelberg \cite{engelberg01}
analizza come discorso interno «tra il Sé e il (Sé come) Altro».

Se seguissimo «il cosiddetto contenuto del libro è il lettore» ci troveremmo alla
questione del \emph{medium}: il libro, il testo, il dialogo, l'oggetto sonoro,
il concerto: \emph{media}.

Definiti i media bertonciniani, non resta che definire il lettore, l'interprete,
il musicista, il dialogante\ldots la separazione dal medium attivata
dall'incantesimo del fascino innesca la possibilita di un enigma che prende
forma nel contenuto del lettore destinatario.

%-------------------------------------------------------------------------------
\section*{ricerca}
%-------------------------------------------------------------------------------

\begin{quote}
  \begin{sf}
    \small
    \begin{description}
      \item[Bremonte:] Mi viene in mente, dicevo, una confidenza che X. m’ha
      fatto su quell’epoca eroica di ricerca sonora.

      Era l’inverno del 1973; egli si trovava a Berlino ospite del
      Künstlerprogramm del D.A.A.D. Per la prima volta nella sua vita il mio
      amico ebbe l’agio di allestire un vero e proprio laboratorio artigiano,
      adatto alla realizzazione delle idee che già da alcuni anni lo spingevano
      verso un certo tipo di costruttivismo musicale, ma che per ragioni
      pratiche, a Roma, non potevano avere adeguato sviluppo.

      [\ldots]

      % Tornato dunque nel proprio studio dopo la visita romana al Guiducci,
      % ancora sotto l’impressione vivissima di quelle sonorità particolari anche
      % se appena udibili, sonorità che non avevano come modo d’attacco niente in
      % comune con i mezzi di produzione sonora corrente, egli si accinse alla
      % costruzione d’una prima arpa eolia dalla cornice metallica rettangolare di
      % cm. 80 x 60, cui seguì immediatamente una seconda di dimensioni poco
      % maggiori e una serie di \emph{gong eolici} (così li chiamò in un primo
      % tempo e quel nome rimase); cioè un’installazione basata su spirali e barre
      % metalliche di acciaio, di bronzo e di ottone, messe in vibrazione
      % esclusivamente da getti d’aria compressa.

      Ma io non sono in grado di rendere nemmeno approssimativamente la tensione
      che egli riesce a comunicare quando evoca quei momenti. Nella descrizione
      della fase finale nella costruzione della prima arpa eolia, e delle prove
      ripetute d’un sistema d’amplificazione da lui escogitato per rendere
      chiaramente percettibili quelle sonorità ricchissime ma fondamentalmente
      al limite della della audibilità, egli è capace di ricreare e di evocare
      nell’immaginazione dell’ascoltatore la stessa atmosfera gravida di attesa
      che circonda il lavoro febbrile di costruzione della campana nel film di
      A. Tarkowskij, ANDREIJ RUBLOV.
    \end{description}
  \end{sf}
\end{quote}

% \subsection*{appunti da wikipedia}
%
% La campana, 1423: Un Duca vuol far realizzare una grande campana, ma tutti i fonditori di campane sono morti per via della peste. Boriska (Nikolaj Burljaev), il figlio di uno di questi, convince gli uomini del duca di essere a conoscenza del segreto della fusione delle campane, confidatogli dal padre prima della morte. Il compito viene affidato al ragazzo, che dirige i lavori con severità e determinazione. Alla preparazione della campana assiste anche Andrej Rublëv. Il buffone che anni fa era stato arrestato riconosce Andrej e minaccia di ucciderlo perché pensa che sia stato lui a denunciarlo in quell'occasione. Kirill riesce a calmare il buffone, e poi confida ad Andrej di essere stato lui anni prima a denunciare il buffone e lo rimprovera di non usare il talento che Dio gli ha dato, nonostante molti gli chiedano di dipingere. Il giorno dell'inaugurazione della campana Boriska è estremamente teso e inquieto. È presente tutta la popolazione e molti nobili e la campana suona perfettamente. Quando tutti se ne sono andati Boriska piange disperato. Andrej gli si avvicina e, rompendo il voto di silenzio, gli chiede perché piange; il ragazzo allora gli rivela che suo padre non gli ha mai voluto rivelare il segreto per la costruzione delle campane. Egli è riuscito nell'impresa solo grazie alla sua determinazione e alle sue capacità. Andrej allora gli propone di partire insieme: “Tu fonderai campane, io dipingerò icone”
%
%
% \subsection*{appunti da Il suono del kairos e il segreto della campana}
% Più che un film su un pittore di icone, si potrebbe dire che Andrej Rublëv è, esso stesso, un'icona. L'unità è soltanto suggerita dall'andamento paratattico fatto di episodi che, proprio come i dettagli di un'immagine sacra, derivano le dimensioni dal valore che è loro attribuito, anziché da una forzatura prospettica. Nel continuo entrare e uscire dallo sguardo dei personaggi, il momento in cui la campana inizia a suonare risulta  un punto decisivo.
%
% Anche se nella Russia sovietica quella delle campane viene considerata una voce di dissenso e il suono del bronzo finisce sopraffatto da quello dell'acciaio, tutto il film, così come la vita russa della tradizione, è scandito da rintocchi. Il rintocco della campana di Boriška, però, assume un significato nettamente centrale, già dalle premesse narrative: se la campana non suona, Boriška morirà (anche perché una campana ortodossa, una volta fusa non si può più ritoccare: essendo un oggetto sacro, accordarla sarebbe una sorta di sacrilegio).
%
% Nell'ottavo capitolo si concentra il senso della creazione artistica secondo Tarkovskij. Rublëv, Tarkovskij stesso e Boriška sono tre ipostasi della medesima idea: un'arte che tragga origine dalla collettività e che a questa ritorni. Anche se il lavoro nella fucina e il picchiare dei martelli riecheggiano l'ossessione sovietica per l'industria pesante, Boriška non è affatto un eroe del lavoro. C'è una sostanziale deviazione rispetto ai  film di argomento storico sfuggiti alle forbici negli anni precedenti: secondo Tarkovskij l'artista deve farsi servo dell'assoluto e condividerne la rappresentazione, senza alcuna glorificazione finale. Teurgia a parte, la riflessione del regista si lega alla prima avanguardia, che nelle sue varie formulazioni mirava a stimolare una sorta di completamento del film da parte dello spettatore. Per indurre una reazione autentica da parte di chi osserva, per Tarkovskij bisogna prescindere dalle convenzioni stilistiche, che equivalgono a pregiudizi, proprio come avviene in Andrej Rublëv, in cui etica ed estetica si compenetrano. La specificità del mezzo cinema deve consistere in una vera e propria scultura del tempo (in un certo senso nella sua cattura fattografica all'interno dell'inquadratura), con tutte le implicazioni filosofiche che ne derivano.
%
% Si potrebbe dire che con l'avvio della campana, a sua volta metafora di un'icona, Andrej Rublëv arriva persino a simulare l'esperienza della rivelazione divina: è come se Tarkovskij proponesse, nel corso del film, una guida alla lettura delle immagini sacre poi mostrate a colori nella parte finale. Si pensi alla tensione temporale dell'inquadratura in cui il pesantissimo battaglio (in russo jazyk, 'lingua') prende ad oscillare dallo sfondo al primo piano: ricalca quasi il movimento suggerito dalla cosiddetta 'prospettiva rovesciata' delle icone teorizzata da Pavel Florenskij, in base alla quale le figure sembrano avanzare in direzione dell'osservatore. Mentre si attende la manifestazione della voce divina, immagine e suono sono ancora saldate nelle inquadrature dei lentissimi cigolii. Poi, quando la campana finalmente suona, Tarkovskij cela la fonte sonora, per spostarsi su Boriška e su Rublëv. Inevitabilmente lo stacco ha l'effetto di spezzare la prevedibilità non soltanto visiva ma anche temporale di una distanza progressivamente ridotta e di creare un nuovo senso di attesa. L'agognato e imprevedibile rintocco arriva a suggerire una trasformazione del chronos, il tempo quantitativo, in kairos, un tempo qualitativo. Per Rublëv è il tempo della rivelazione e dell'incontro con Dio.
%
% Osservando il tutto senza tener conto di suggestioni teoriche si pone una questione: non può essere una coincidenza il fatto che il frammento del film in cui si delinea una svolta spirituale sia anche quello che più si avvicina a uno schema da cinema classico. Tarkovskij scrive che il montaggio verticale è una dittatura, eppure, nella scena delle oscillazioni, in fondo il momento di massima rigidità ideologica, conserva una precisa saldatura tra suono e immagine e ricorre a una soluzione di montaggio, con una raffinata elaborazione in campo-controcampo. Magari era questo il segreto della campana.
%
%
% \subsection*{appunti bibliografici}
%
% Deleuze, Gilles, Cinema 1: The Movement-ImageLondon: Continuum, 2005a.
%
% Deleuze, Gilles, Cinema 2: The Time ImageLondon: Continuum, 2005b.
%
% Tarkovsky, Andrei, Sculpting in Time: Reflections on the CinemaLondon: University of Texas Press, 1988.

%-------------------------------------------------------------------------------
\section*{botteghe di suono}
%-------------------------------------------------------------------------------


Mario Bertoncini sta all'aria compressa come Prometeo sta al fuoco. Mario era un
uomo di Bottega. Il Laboratorio è il momento di catastrofe in cui egli ha
addomesticato l'aria. Il laboratorio di Mario è il pensiero, come in Schoenberg,
come in Nono.

Le opere di Mario, oggi, dopo Mario, appaiono come botteghe di suono. Il
progetto bottega che mosso la sua intera esistenza è presente, oggi, in ogni
traccia della sua eredità. È nostro compito vivere quei luoghi, comprendere le
sue parole per curare la sua costruzione.

\begin{quote}
\verbatim
Come onorare la musica\\
ascoltarla senza agire\\
senza sequenze di commenti\\
come tacere non di fronte\\
ma dentro la musica\\
come lasciare che mi disfi\\
il cuore e mi scombini il pensiero.
Un paesaggio sonoro\\
ci affronta ovunque e ci invita:\\
essere musica è fuga dal rumore\\
è orecchio sinfonico\\
verso la voce del silenzio\\
verso la conta di quello che resta.
\end{quote}
