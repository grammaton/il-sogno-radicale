E quindi, \emph{che cos'è ricerca?}\footnote{%
  La ricerca, dal latino \emph{re-circare} (letteralmente "girare intorno",
  "cercare di nuovo"), porta in sé l'idea del movimento circolare, del ritorno
  su qualcosa per approfondirlo. Il prefisso \emph{re-} suggerisce iterazione,
  mentre \emph{circare} deriva da \emph{circus}, il cerchio - un'immagine che evoca tanto il metodo quanto la condizione esistenziale del ricercatore.
  Nell'antichità classica, la \emph{quaestio} era l'atto del domandare,
  dell'investigare, ma senza la sistematicità che caratterizzerà la ricerca
  moderna. I filosofi presocratici erano \emph{physiologoi}, studiosi della
  natura, mentre Aristotele introduce il concetto di \emph{episteme} -
  conoscenza dimostrativa che si distingue dalla \emph{doxa}, l'opinione
  comune. Questi sono dati indicativi sull'origine del termine che, passando
  per il Medioevo giunge all'accezione moderna che emerge nel Rinascimento, quando la \emph{scientia} si emancipa dalla filosofia naturale e si
  introduce con l'induzione sperimentale. La ricerca diventa anche \emph{experimentum}, prova controllata della realtà.

  «\emph{research}, noun [mass noun] (also researches) the systematic
  investigation into and study of materials and sources in order to establish
  facts and reach new conclusions\ldots»}

Il ricercatore di professione, \emph{professionale}, è un'invenzione del XIX
secolo. La figura si delinea definitivamente con la rivoluzione industriale, a
seguito della quale i laboratori industriali incarneranno la “ricerca applicata”,
mentre le università continueranno a coltivare quella “pura”. %La dicotomia tra theoría e praxis si istituzionalizza.
Poco dopo, due guerre mondiali dimostreranno il potere della scienza applicata:
dal \emph{Progetto Manhattan} alla penicillina, la ricerca diventa questione di sopravvivenza\footnote{Vannevar Bush, nel rapporto \emph{Science: The Endless
Frontier} (1945), teorizza il \emph{linear model}: ricerca di base: ricerca
applicata: sviluppo tecnologico: benessere sociale.}.

Poi, un giorno, la dicotomia università/industria è venuta a mancare. Oggi il
ricercatore di professione si trova in una posizione paradossale. Da un lato,
gode di prestigio sociale teorico - la società della conoscenza celebra
retoricamente la ricerca. Dall'altro, sperimenta una condizione di precarietà
economica e strutturale. Il \emph{publish or perish} diventa imperativo
categorico e la ricerca si sottomette alla logica della \emph{performance},
misurata attraverso metriche quantitative: \emph{impact factor, h-index, citation count}.

Il ricercatore contemporaneo è quindi una figura tragica, nel senso greco, che
potremmo fissare su pietra: consapevole dell'importanza della propria missione,
chinato a ri-cercare, qualcosa, è schiacciato da un grande globo in putrefazione.
Ai suoi piedi, come un \emph{Gulliver}, tanti piccolo uomini: la società odierna
lo celebra quando produce risultati immediatamente applicabili - vaccini,
tecnologie, soluzioni - ma lo emargina quando si staglia enorme, dedito alla
comprensione pura.

Questo teatro, ha degli angoli nascosti da descrivere. La neonata condizione
della \emph{ricerca artistica} fa pensare, più che a una nascita, a una
ri-nascita. La ricerca artistica è più un Lazzaro che un Gesù. Originariamente
integrata nel sapere umano, oggi l’arte può partecipare alla ricerca, di nuovo,
dopo le sue morti. E questa metafora cristologica non dovrebbe risultare casuale:
dovrebbe, di nuovo, rivelare la natura sacrificale del processo.

L'arte, nel mondo antico, era \emph{techne} - sapere operativo indistinto dalla
conoscenza. L'architetto, il musico, il poeta partecipavano dello stesso
\emph{logos} del matematico e del filosofo. Vitruvio non distingueva tra
ingegneria e estetica; le arti del \emph{trivium} e \emph{quadrivium} formavano
un corpus unitario. L'artista medievale era \emph{artifex}, artigiano del sacro,
depositario di saperi trasmessi attraverso corporazioni che erano insieme
botteghe e accademie. Il Rinascimento porta all'apogeo questa integrazione.
Leonardo incarna l'uomo universale - la sua anatomia è insieme arte e scienza,
i suoi studi idraulici sono pittura in movimento. L'arte diventa \emph{ars
inveniendi}, strumento di scoperta del reale. La prospettiva brunelleschiana è
geometria applicata; la botanica di Dürer è tassonomia ante litteram. Ma già qui
si annuncia la frattura. Leon Battista Alberti teorizza l'arte come
\emph{imitatio naturae}, introducendo una gerarchia che subordina l'arte alla
natura - e quindi alla scienza che la studia. L'artista inizia a perdere la
propria autonomia epistemologica. La modernità consuma il divorzio. Cartesio
distingue \emph{res extensa} e \emph{res cogitans}, ma l'arte resta orfana - né
pura materia né puro pensiero. La rivoluzione scientifica galileiana,
matematizzando la natura, espelle l'arte dal regno della conoscenza “seria”.
Kant, nella \emph{Critica del Giudizio}, tenta una mediazione, ma finisce per
relegare l'estetico in una zona franca, autonoma ma marginale. Il Romanticismo
compie il parricidio definitivo. Proclamando l'arte “religione dell'assoluto”
(Schelling), la sottrae al mondo della conoscenza razionale. L'artista romantico
è \emph{vates}, profeta ispirato che accede a verità inattingibili alla ragione.
Ma questa sacralizzazione è anche una condanna a morte: l'arte diventa puro
soggettivismo, esperienza incomunicabile, \emph{Erlebnis} irriducibile al
concetto.

Hegel diagnostica lucidamente questa fine: l'arte, nella modernità, ha esaurito
la propria funzione conoscitiva. È “cosa del passato” (\emph{Sache der Vergangenheit}).
La bellezza naturale cede il passo a quella artistica, ma questa, a sua volta,
viene superata dalla filosofia come forma suprema dello spirito assoluto.
Il positivismo ottocentesco suggella la separazione. Auguste Comte espelle
l'arte dalla gerarchia delle scienze. Spencer la riduce a “gioco” evolutivo.
L'artista diventa \emph{bohémien}, figura liminale che la società borghese
tollera ma non riconosce come produttore di sapere.

Le avanguardie novecentesche tentano la resurrezione con tentativi disperati di
riconquistare dignità epistemologica, ma attraverso la negazione di sé.

Oggi assistiamo davvero a una resurrezione lazarina. La \emph{research-based art},
la \emph{practice-based research}, l'\emph{artistic research} rivendicano per
l'arte lo statuto di produttrice di conoscenza. Ma è una rinascita problematica,
perché avviene in un mondo accademico già compromesso dalla logica performativa
denunciata. L'arte, per rientrare nell'università, deve sottomettersi ai suoi
protocolli: \emph{peer review}, pubblicazioni, metriche. Il paradosso è stridente:
l'arte, che per natura eccede ogni misura, deve ora misurarsi secondo parametri
che le sono estranei. L'\emph{artistic research risk} di diventare un ossimoro:
arte che rinuncia alla propria specificità per ottenere rispettabilità accademica.

Eppure, in questa resurrezione può esserci qualcosa di autentico. Come Lazzaro
che torna dal regno dei morti portando con sé l'esperienza dell'oltre, l'arte
contemporanea può rientrare nella ricerca con una consapevolezza, con un corpo,
con la coscienza del limite, del non-dicibile, dell'aporetico, con il dialogo
filosofico, con la cura di sé, un processo, e delle sue peculiarità. L'arte sa
che ogni rappresentazione è sempre anche tradimento del rappresentato. Sa che
la verità non è possesso ma tensione. Sa che la conoscenza più profonda spesso
si dà per via negativa, attraverso ciò che sottrae piuttosto che ciò che aggiunge.

La resurrezione dell'arte nella ricerca potrebbe essere la redenzione della
ricerca stessa - se saprà resistere alla tentazione di normalizzarsi secondo
logiche che le sono estranee, mantenendo viva quella dimensione di stupore che è
all'origine di ogni autentico sapere.

\bigskip

L'arte è già sempre \emph{res publica} - cosa pubblica - non perché distribuita
attraverso canali istituzionali, ma perché è la realtà resa comune, condivisibile,
esperibile. L'opera d'arte non rappresenta il mondo: fa mondo.

Quando Cézanne dipinge la Montagne Sainte-Victoire, non sta “pubblicando” una
ricerca sulla montagna - sta facendo esistere quella montagna nella coscienza
collettiva in modo precedentemente impossibile. L'opera è già compiutamente
pubblica nell'istante della sua realizzazione, perché apre un mondo comune di
significati. Non ha bisogno di validazione esterna: la sua verità si manifesta
nell'esperienza diretta, nel rapporto immediato tra l'opera e l'altro.

Il tempo della condivisione artistica è infatti qualitativo dell'opportunità,
della maturazione, del riconoscimento graduale. Le \emph{Variazioni Goldberg} di
Bach non hanno avuto bisogno di peer review per diventare patrimonio dell'umanità.
Hanno avuto bisogno di tempo per sedimentarsi nella coscienza collettiva, per
essere ri-scoperte, re-interpretate, ri-vissute attraverso generazioni interpreti
e ascoltatori.

C'è anche una “produzione artistica” che macina prime esecuzioni, con precisione
devastante. Essa replica esattamente la logica accademica del \emph{publish or
perish}: quantità di output come metro di valore, originalità formale come
feticcio, velocità di produzione come virtù. Ma così facendo tradisce l'essenza
stessa dell'arte. L'arte autentica sa che la ripetizione è spesso più rivelatrice
dell'invenzione. Morandi ha dipinto le stesse bottiglie per decenni, ogni volta
scoprendo nuove verità in quella configurazione apparentemente immutabile.
Glenn Gould ha registrato le Variazioni Goldberg due volte a distanza di
vent'anni - e la seconda volta non era una “ripubblicazione”, ma una nuova
ricerca sullo stesso territorio inesauribile. Karajan ha inciso quattro volte
l'integrale sinfonico di Beethoven, tre di queste con la stessa orchestra, nello
stesso auditorium.

Il sistema delle prime esecuzioni assolute è infatti una perversione della
temporalità artistica. Impone all'arte la logica dell'événement, dell'accadimento
spettacolare, quando invece l'arte lavora per durata, per sedimentazione,
per attraversamenti ripetuti che rivelano sempre nuove profondità.

La ricerca lazarina ha un tempo diverso: quello della fermentazione, della lenta
acquisizione di senso, del riconoscimento progressivo. Come Lazzaro che torna
dai morti, porta con sé una verità che non può essere consumata immediatamente,
che richiede elaborazione, metabolizzazione, comprensione graduale.

Forse la ricerca artistica dovrebbe rivendicare il diritto alla lentezza, alla
ripetizione, alla non-originalità. Dovrebbe opporre alla fretta della prima
esecuzione assoluta la pazienza dell'approfondimento infinito. Dovrebbe ricordare
che l'arte, a differenza della scienza, non “supera” le proprie acquisizioni
precedenti ma le attraversa sempre di nuovo, ogni volta con sguardo diverso.
Il valore non sta nell'accumulo di pubblicazioni, ma nella capacità di fare mondo,
di aprire spazi di senso condivisi, di creare quella \emph{res publica} del
significato di cui ogni autentica comunità umana ha bisogno per riconoscersi e
crescere.

La ricerca autentica è \emph{perigegesis} - circumnavigazione, circumambulazione
- non teleologia. È movimento orbitale intorno a un mistero che non va risolto
ma abitato. \emph{Re-circare}: il prefisso iterativo \emph{re-} unito a
\emph{circus}, cerchio. Non un “andare verso” ma un “girare intorno”, non una
conquista lineare ma una danza circolare. La ricerca è movimento che non consuma
il proprio oggetto ma lo onora attraverso l'approssimazione infinita.

Celan scrive: "La poesia non s'impone, si espone". La ricerca autentica non si impone sulla realtà per piegarla ai propri fini, ma si espone ad essa, si lascia trasformare dall'incontro. È passio prima che actio, ricezione prima che conquista.

Questo movimento circolare ha una sua economia diversa da quella industriale. Non consuma risorse per produrre beni, ma trasforma continuamente le stesse risorse - attenzione, tempo, sensibilità - in comprensione sempre più raffinata. È economia dell'intensificazione, non dell'accumulo.

La circolarità della ricerca è anche temporale. Essa non procede linearmente
dal passato verso il futuro, ma riattiva continuamente il passato nel presente,
scopre nel già-noto il non-ancora-pensato. In questo senso, la ricerca è sempre
anche \emph{anamnesis} platonica - non apprendimento di nuovo, ma ri-cordanza di
ciò che già si sapeva senza saperlo. È riconoscimento, non conquista. È tornare
a casa attraverso una strada mai percorsa prima, ma che era sempre stata lì ad
aspettarci.

\emph{Iventare}, dal latino \emph{invèntus}, participio passato di \emph{in-venire}:
trovare, scoprire cercando, giungere a qualche meta: giungere.
