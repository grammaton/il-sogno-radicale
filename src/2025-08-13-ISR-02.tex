%!TEX TS-program = xelatex
%!TEX encoding = UTF-8 Unicode

\documentclass{gs}

% Configurazione per URL su nuova linea
\DeclareFieldFormat{url}{\newline\url{#1}}
\DeclareFieldFormat{howpublished}{\newline\footnotesize#1}
\DeclareFieldFormat{urldate}{\mkbibparens{#1}}

\usepackage{../lib/apf}

\title{il sogno radicale}
\subtitle{autobiografia di un eretico. appunti.}
\author{Giuseppe Silvi}
\date{\today}

% Definizione delle parole chiave per i metadati PDF
\kuno{filosofia}
\kdue{greco antico}
\ktre{Platone}
\kquattro{etimologia}
\kcinque{XeLaTeX}

% Nel preambolo del .tex
\addbibresource{bibliografia.bib}

\begin{document}

\maketitle

\section*{premessa: \emph{perché siamo qui?}}

Sto operando quella che Ronchi, nella prefazione de \emph{il pensiero bastardo} \cite{ronchi2001} ha definito una \emph{torsione}, una «preliminare torsione filosofica» dell'oggetto.

\begin{quote}
\begin{sf}
\small
  Le altre pratiche (pittura, fotografia, cinema, ecc.) sono chiamate in causa solo al prezzo di una preliminare torsione filosofica del loro oggetto. \cite{ronchi2001}
  \end{sf}
\end{quote}

La forma latina \textit{torsiō} deriva dal verbo \textit{torquēre}, riconducibile alla radice indoeuropea *terkʷ- “torcere, girare”\footnote{Pokorny IEW 1077-1078; de Vaan EDLIL 621-622.}, che designa il movimento rotatorio applicato mediante forza. La torsione implica una resistenza\footnote{Il materiale oppone una resistenza caratterizzata dal modulo di elasticità tangenziale e dal momento d'inerzia polare della sezione. Dal latino \emph{re-sistere}: uno \emph{stare contro} che è inevitabilmente anche \emph{stare di nuovo} - un ristabilirsi, un tornare a stare.} che rende possibile una tensione\footnote{La torsione genera tensioni di taglio (\emph{shear stress}) nel materiale. Queste tensioni sono distribuite trasversalmente alla sezione e aumentano linearmente dalla fibra neutra verso la periferia. Dal latino \emph{tensio}, da \emph{tendere}, pro-tendere qualcosa verso; un orientamento attivo.}. Un \emph{ob-ponere} che dura in un \emph{ob-sidere} che genera forma, movimento. Un doppio moto – di deformazione sotto l'azione del momento torcente applicato; di ritorno ad un altro momento sotto l'azione di resistenza elastica\footnote{Il termine momento indica sia l'istante temporale che la grandezza fisica (momento torcente), suggerendo come ogni deformazione sia sempre un dialogo tra forza e tempo, tra imposizione e memoria del materiale. Ritorno ad un altro momento perché, dopo una torsione, si è altrove.}.

\begin{figure}[htbp]
\begin{center}
\begin{tikzpicture}
  \tikzstyle{every node}=[font=\small]
  \ciclodiagram{torsione}{resistenza}{tensione}
\end{tikzpicture}
\caption{Il grafico simboleggia la torsione come nodo concettuale stretto tra un moto di tensione distorcente e un moto di ritorno sotto la forza di resistenza.}
\label{torsione}
\end{center}
\end{figure}

Mi è evidente che c'è un passo della torsione, quello successivo, che conduce alla \emph{dis}-torsione. %Magari della realtà.
Tuttavia, seppur con questo breve pensiero cerco di tenermene a distanza, un passo prima, mi è altresì evidente che, in torsione, non si può avere paura delle distorsioni poiché questa impedirebbe possibili \emph{con}-torsioni. È se anche il Prof. Ronchi intendesse indicare una qualsiasi semplice deviazione rispetto a una condizione di normalità o linearità (la torsione di un ragionamento, di una interpretazione, di una situazione), cosa che io dubito, ormai mi son mosso.

Sento un'assenza. Mi accorgo, quotidianamente, da un certo tempo, a seguito di alcune singole riflessioni \cite{silvi2023a, silvi2023b, silvi2023c, silvi2024}, seppur non isolate, di desiderare, di avere necessità di \emph{abitare} una teoria della musica. Non è un bisogno, momentaneo e transitorio, bensì l'inevitabile presenza di un'assenza, costante. Molte delle singole riflessioni, delle risonanze singolari, vorrebbero appartenere a un riverbero, a riverberi generali di un luogo più ampio che chiamerei \emph{teoria della musica}. 

\begin{quote}
\begin{sf}
\small
  La teoria è infatti chiamata in causa, dopotutto, solo per far luce su di un'esperienza di cui non si riesce a venire a capo. \cite{ronchi2001}
  \end{sf}
\end{quote}

Abitare una teoria. Tuttavia, \emph{che cos'è una teoria? Che cos'è una teoria musicale? E santo cielo! Che cos'è Musica? Che cos'è un'esperienza? Che cos'è un'esperienza musicale?}

Heidegger, che mi pare amasse le torsioni, in un ragionamento sull'abitare \cite{heidegger1991} si muove da un primo passo «All'abitare, così sembra, perveniamo solo attraverso il costruire». Ma noi, in musica, oggi, siamo in piena «crisi di alloggi». Osservo l'ambiente che mi circonda e vedo pochissime costruzioni che «albergano l'uomo», il musicista. Per imparare ad avere rispetto della mia ineluttabile necessità scorro la storia delle teorie della musica, un indice che si muove più volte avanti e dietro, non in cerca di regole, definizioni di una o più problematiche specifiche, ma in cerca di divergenze, di attitudini, di problematiche generali affinché io possa, con semplicità, trovare radici per le mie problematiche specifiche, per le utopie che muovono le mani. «Il costruire non è soltanto mezzo e via per l'abitare, il costruire è già in se stesso un abitare.» \cite{heidegger1991} %Per questo motivo festeggio la mia torsione, festeggio con una danza tra necessità e creazione, per sentirmi a casa. 

\subsection*{\emph{théatron}: luogo riverberante}

\begin{quote}
\begin{sf}
\small
\ldots spazi vuoti pronti a venirsi riempiendo uno alla volta, spazi della voce nei quali si apprenderà con l'udito\ldots~\cite{zambrano1991}
\end{sf}
\end{quote}

Il \emph{théatron} messo in piedi per (\emph{theáomai}) questa torsione è un teatro acustico, poiché la regione in cui si sta organizzando l'assedio è la conoscenza dell'udire. È sì che si torce: \emph{theōría}: guardare, osservare e il risultato di tutto ciò: ma a occhi chiusi, spegnendo l'immagine. \emph{Theáomai} sì, ma spettatore di altra natura, di altro \emph{théatron}. Della visione attenta, della contemplazione, tratteniamo la postura, la partecipazione intellettuale. Come \emph{theōroi}, osservatori ufficiali inviati ad assistere alla festa del suono, a consultare l'oracolo che chiamiamo musica, a riferirne ciò ch'è stato udito, sentito, appreso.  L'osservare con le orecchie è a sua volta una torsione, una con-torsione (dialettica) dell'ascoltare nelle due direzioni opposte e convergenti dell'udire e del sentire.

\begin{figure}[htbp]
\begin{center}
\begin{tikzpicture}
  \tikzstyle{every node}=[font=\small]
  \ciclodiagram{ascoltare}{udire}{sentire}
\end{tikzpicture}
\caption{L'udire prende il posto della resistenza, la resistenza meccanica si traduce in una resitenza acustica (impedenza): il materiale risponde alle sollecitazioni vibrazionali accogliendo, respingendo, ignorando. L'udito è già selettivo di questa selezione: ha le sue soglie. Resiste e amplifica, è un sistema di resistenze differenziali. Udire (\emph{Audire}) e anche un obedire (\emph{Ob-audire}, udire verso). Sentire, connesso a \emph{sensus} e alla radice indoeuropea \emph{sent-} (“andare, dirigersi”) il sentire è movimento orientato, direzionalità dell'esperienza. Come \emph{tendere} è “pro-tendersi verso”, sentire è “dirigersi verso” - entrambi esprimono tensionalità, orientamento intenzionale.}
\label{ascoltare}
\end{center}
\end{figure}

Lo spazio agonistico \cite{ronchi2001} di questo ascoltare (\emph{theōría}) è uno spazio della vibrazione acustica udibile, un tempo della vibrazione, uno spazio dei suoni: un \concetto{sonario}. 

\subsection*{sonario}

\begin{quote}
\begin{sf}
\small
Gli uomini medievali non provavano fiducia per i loro sensi; l'orecchio musicale non offriva sufficiente sicurezza. Il metodo scolastico degli studiosi non includeva la ricerca e l'esperimento, ma consisteva nel trovare una autorità classica e nell'adottare le sue conclusioni riguardo a problemi contemporanei. Certamente dal greco Aristosseno i musicisti medievali avrebbero potuto apprendere il rispetto per l'orecchio musicale, ma proprio a causa di questa fiducia nelle impressioni sensoriali egli era rifiutato dai monaci; uno scrittore anonimo, probabilmente dell'XI secolo, affermò ingenuamente che Boezio non sarebbe mai andato d'accordo con Aristosseno... \cite{sachs1996}
\end{sf}
\end{quote}

Si vuole raccontare una storia. Al fianco del \emph{vedere-comprendere} che riflette la concezione classica della conoscenza come visione intellegibile che pervade tutto il vocabolario epistemologico occidentale (idee, evidenza, intuizione, immaginazione\ldots) la storia che si vuole raccontare ha origini antichissime, come la visione, con lo stesso dualismo dialettico agonistico tra teorico e pratico, con lo stesso eterno desiderio di giungere alla \emph{theōría} con quel significato tecnico scientifico di sistema coerente di principi che spiegano un insieme di fenomeni. Così, si vuole raccontare una storia %dell'esperienza musicale.

\begin{quote}
\begin{sf}
\small
we have to perceive what is coming to be and remember what has come to be. There is no other way of following the contents of music\footnote{Aristosseno, Elementa Harmonica, book II, 38–39. Traduzione in \emph{The Stanford Encyclopedia of Philosophy}}
\end{sf}
\end{quote}

Mi pare che Aristosseno attivi, come il suo Maestro Aristotele, quell'\emph{agōn} \cite{ronchi2001} tra sensazione (volpe) e memoria (cane), in quel \emph{per sempre avere luogo} della rappresentazione istantanea della coesistenza tra sensazione e memoria, che chiamiamo \emph{esperienza musicale}.

La storia, quindi, sottolinea la dimensione temporale della coscienza musicale, ovvero la percezione di ciò che \emph{è diventato} (orientata al passato) e il ricordo di ciò che \emph{sta per diventare} (orientato al futuro).

\begin{figure}[htbp]
\begin{center}
\begin{tikzpicture}
  \tikzstyle{every node}=[font=\small]
  \ciclodiagram{coscienza musicale}{sensazione}{memoria}
\end{tikzpicture}
\caption{Ciò che in Aristosseno è \emph{pre}-disposizione per l'esperienza musicale, si ri-organizza nella coscienza  di quella esperienza con la sensazione uditiva di una vibrazione che per definizione è già passata; la memoria di quel passato è protesa ad un nuovo futuro.}
\label{coscienza}
\end{center}
\end{figure}

\begin{description}

  \item[Glossario] Dal latino tardo \emph{glossarium}, derivato da \emph{glossa} (parola rara o straniera che necessita spiegazione), a sua volta dal greco \textgreek{γλῶσσα} (glōssa) che significa "lingua" o "linguaggio".
Raccolta di termini specifici di una disciplina, ordinati alfabeticamente e corredati di definizione. Originariamente era una raccolta di termini difficili o arcaici (\emph{glossae}) con relative spiegazioni.

  \item[Immaginario] Dal latino \emph{imaginarius} (che esiste solo nell'immaginazione), derivato da \emph{imago, imaginis} (immagine, rappresentazione), connesso alla radice indoeuropea \emph{im-} (copiare, imitare).
Come sostantivo, indica l'insieme delle immagini, dei simboli e delle rappresentazioni mentali condivise da una cultura o da un individuo. È il luogo concettuale dove risiedono le immagini che costituiscono il nostro modo di vedere e interpretare il mondo.

  \item[Scenario] Dal tardo latino \emph{scenarium}, derivato da \emph{scena} (palcoscenico, scena teatrale), che proviene dal greco \textgreek{σκηνή} (skēnē), originariamente "tenda, riparo" poi "palcoscenico".
Inizialmente indicava l'insieme delle scene di una rappresentazione teatrale. Per estensione, oggi denota la descrizione di una possibile sequenza di eventi o sviluppi futuri, o l'ambientazione in cui si svolge un'azione.

\end{description}

Come il glossario raccoglie e definisce le parole, l'immaginario accoglie e struttura le immagini, lo scenario esemplifica lo spazio delle scene, sia esso verticale (coesistenza) che orizzontale (successione) così il \emph{sonario} diventa il ricettacolo ontologico dei fenomeni sonori, in relazione all'ascolto, alla creazione e alla riflessione umana.

\begin{description}

  \item[Sonario] Dal latino \emph{sonus} (suono) con il suffisso \emph{-arium} che indica raccolta, contenitore o luogo dedicato. Il \emph{Sonario} è lo spazio agonistico della ragione acustica che si fa metodo nella coscienza musicale, il \emph{théatron} dove il suono si fa interfaccia teorico-pratica – dove \textgreek{θεωρία}, \textgreek{πρᾶξις} e \textgreek{ποίησις} si dispiegano.

\end{description}

\emph{Theōría}, l'attività d'osservazione attenta e contemplativa (in musica: \emph{ascolto}) che si relaziona con la \emph{praxis} condotta strumento alla mano, come questa penna stilografica, e che adduce alla \emph{poiesis}: la musica.

Per Aristotele questo spazio agonistico è tripartito in \emph{bios theoretikos, bios praktikos, bios poietikos}. All'interno della visione di questa \emph{theōría} diviene

\begin{figure}[htbp]
\begin{center}
\begin{tikzpicture}
  \tikzstyle{every node}=[font=\small]
  \ciclodiagram{ποίησις}{πράξις}{θεωρία}
\end{tikzpicture}
\caption{Triade aristotelica come ciclo dialettico}
\label{poiesis}
\end{center}
\end{figure}

\subsection*{\emph{en archei}}

\begin{quote}
\begin{sf}
\small
Dove ha origine – ci si chiede in queste pagine – il fascino che accompagna l'apparizione di qualcosa quando questo si staglia dal suo contesto abituale per lasciare traccia del suo passaggio?
\end{sf}
\end{quote}

\emph{dove ha origine?}

È un dove nasce e, insieme, un quando inizia. Ma è anche un quando nasce, insieme dove inizia. Origins, punto di partenza temporale oriente oriens, che sorge. dove ha origine è una corrispondenza spazio – temporale.ARCHĒ


\begin{quote}
\begin{sf}
\small
L'archeologia è la ricerca di un archē, usa il termine greco archē ha due significati: significa tanto “origine”, principio, punto “comando ordine” così il verbo archo significa “imitare, essere, il primo a fare qualcosa” ma significa anche “comandare, essere il capo”. Senza dimensione che l'arconte (letteralmente “colui che comincia”) era in Atene la suprema magistratura. (AGAMBER CREAZIONE E ANARCHIA p 91) 
\end{sf}
\end{quote}

\emph{en archei}, in principio, nel comando, «cioè nella forma di un comando», «l'inizio è sempre anche il principio che governa e comanda».

Archeologia del fascino che accompagna l'apparizione. e nell'apparire si lega insieme all'apparizione, come una fase che vincola e attrae irresistibilmente. l'archeologia del fascino mantiene l'idea di un potere che opera al di là della volontà razionale, en archei, il potere di un comando.

«qualcosa appare a partire da se stesso» GAITP

Apparire, mostrarsi, divenire visibile a partire da se stesso. È un ad – parēre, un movimento verso.

\begin{quote}
\begin{sf}
\small
Il presente, l'Anwesend, deve avermi già rivolto la parola nel suo essere presente.
\end{sf}
\end{quote}

C'è un qualcosa. C'è un movimento. C'è un qualcosa che si muove.

C'è un fascino che accompagna il qualcosa in movimento.

Ha tutti i tratti per essere definito un segnale: mezzo – movimento – informazione\footnote{attenzione: Ronchi dedica il libro al passaggio della cometa halle Bopp. APPROFONDIRE}. 


%Il termine \etimologia{filosofia}{greco φιλοσοφία, “amore per la sapienza”} rappresenta la ricerca fondamentale dell'essere umano verso la comprensione del mondo e di se stesso.
%
%\section{Platone e le idee}
%
%Il \concetto{mondo delle idee} platonico si esprime attraverso il greco \greco{κόσμος νοητός} (kosmos noetos), ovvero il mondo intelligibile contrapposto a quello sensibile.
%
%\begin{citazionefilosofica}{Platone, Repubblica VII, 514a}
%Εἰκάσαι τοιούτῳ πάθει τὴν ἡμετέραν φύσιν παιδείας τε πέρι καὶ ἀπαιδευσίας.
%\end{citazionefilosofica}
%
%Questa celebre allegoria della caverna illustra la condizione dell'uomo nei confronti della conoscenza e dell'educazione.
%
%\subsection{L'episteme e la doxa}
%
%La distinzione tra \concetto{episteme} (ἐπιστήμη, conoscenza scientifica) e \concetto{doxa} (δόξα, opinione) rappresenta uno dei pilastri del pensiero platonico.
%
%\section{Conclusioni}
%
%L'approccio filosofico che considera l'\etimologia{etimologia}{greco ἐτυμολογία, "studio del vero significato"} come strumento di comprensione concettuale si rivela particolarmente fruttuoso nello studio del pensiero antico.

\clearpage

% Nel documento
\printbibliography

\end{document}
