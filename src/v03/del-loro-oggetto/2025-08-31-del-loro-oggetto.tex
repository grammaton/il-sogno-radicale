\section{«\ldots del loro oggetto»: \emph{per quale musica?}}

\emph{Theōría}, l'attività d'osservazione attenta e contemplativa (in musica:
\emph{ascolto}) che si relaziona con la \emph{praxis} condotta strumento alla
mano, come questa penna stilografica, e che adduce alla \emph{poiesis}: la musica.

Per Aristotele questo spazio agonistico è tripartito in \emph{bios theoretikos,
bios praktikos, bios poietikos}. All'interno della visione di questa
\emph{theōría} diviene

\begin{figure}[htbp]
\begin{center}
\begin{tikzpicture}
  \tikzstyle{every node}=[font=\small]
  \ciclodiagram{ποίησις}{πράξις}{θεωρία}
\end{tikzpicture}
\caption{Il processo aristotelico della ποίησις emerge dalla tensione dinamica
tra pratica (πρᾶξις) e teoria (θεωρία), realizzando un ciclo temporale dove ogni
elemento si alimenta degli altri senza perdita di energia com- plessiva. Questo
movimento circolare rappresenta l’isomorfismo fondamentale con il filtro AllPass:
conservazione dell’energia informazionale attraverso trasformazione delle
relazioni temporali.}
\label{poiesis}
\end{center}
\end{figure}

In questo spazio vorrei approfondire alcune problematiche che emergono dalla lettura del primo capitolo del Manuale di Armonia di Schoenberg (riportato in appendice) \cite{schonberg1970} in relazione a ciò che è emerso nella seconda metà del secolo (evangelisti).

\begin{quote}
\begin{sf}
\small
  Le altre pratiche (pittura, fotografia, cinema, ecc.) sono chiamate in causa
  solo al prezzo di una preliminare torsione filosofica del loro oggetto.
  \cite{ronchi2001}
  \end{sf}
\end{quote}

Qual è l'oggetto della musica? Mi viene alla memoria il domandare, domandandosi,
di Deleuze in \emph{Che cos'è l'atto di creazione?} \cite{deleuze2009} Ci-si
chiede «cosa faccio io quando faccio o spero di fare filosofia?». È una
conferenza, si rivolge agli studenti «che cosa fate voi che fate cinema?» E poi
«che cos'è avere un'idea al cinema?»

\begin{quote}
\begin{sf}
\small
  Le idee bisogna considerarle come dei potenziali già impegnati in questo o in
  quel modo d'espressione e inseparabili dal modo d'espressione\ldots In funzione
  delle tecniche che conosco posso avere un'idea in un certo ambito\ldots
  \cite{deleuze2009}
  \end{sf}
\end{quote}

Per avvicinarmi al «loro oggetto» e distinguerlo dalla torsione filosofica devo
passare per «che cos'è il contenuto della filosofia?»

\begin{quote}
\begin{sf}
\small
  \ldots la filosofia è una disciplina che crea e inventa come le altre. Crea o
  inventa concetti. [\ldots] Ci vuole una necssità, in filosofia come altrove, altrimenti non c'è proprio niente.
  \cite{deleuze2009}
  \end{sf}
\end{quote}

La musica crea dei blocchi di suoni/timbri.

\begin{quote}
\begin{sf}
\small
  Se tutte le discipline comunicano fra di loro è sul piano di ciò che non si
  libera mai per se stesso, ma che è come \emph{impegnato} in ogni disciplina
  creatrice, cioè la costituzione degli spazio-tempo.
  \cite{deleuze2009}
  \end{sf}
\end{quote}

\begin{quote}
\begin{sf}
\small
  Il sogno è una terribile volontà di potenza.
  \cite{deleuze2009}
  \end{sf}
\end{quote}

\begin{quote}
\begin{sf}
\small
  Mi dico che avere un'idea non è comunque dell'ordine della comunicazione.
  [\ldots] Non tutto ciò che di cui si parla è riducibile alla comunicazione.
  [\ldots] la comunicazione è la trasmissione e la propagazione di
  un'informazione. [\ldots] un'informazione è un'insieme di parole d'ordine.
  [\ldots] Non ci viene chiesto di credere, ma di comportarci come se credessimo.
  [\ldots] Ciò significa che l'informazione è proprio il sistema del controllo.
  \cite{deleuze2009}
  \end{sf}
\end{quote}

\begin{quote}
\begin{sf}
\small
  La contro-informazione è effettiva solo quando diventa un atto di resistenza.
  \cite{deleuze2009}
  \end{sf}
\end{quote}

\apf{atto di resistenza}{informazione}{contro-informazione}


\begin{quote}
\begin{sf}
\small
  Che rapporto ha l'opera d'arte con la comunicazione?

  Nessuno. L'opera d'arte non è uno strumento di comuinicazione. L'opera d'arte
  non ha niente a che fare con la comunicazione. L'opera d'arte non contiene
  letteralmente la minima informazione. Cìè invece un'affinità fondamentale tra
  l'opera d'arte e l'atto di resistenza.
  \cite{deleuze2009}
  \end{sf}
\end{quote}

l'opera d'arte

\begin{quote}
\begin{sf}
\small
  Malraux\ldots dice che è la sola cosa che resiste alla morte\ldots l'arte
  è ciò che resiste\ldots
  \cite{deleuze2009}
  \end{sf}
\end{quote}

\begin{quote}
\begin{sf}
\small
  Mpm ogni atto di resistenza è un'opera d'arte, benché, in un certo senso, lo
  sia. Non ogni opera d'arte è un atto di resistenza e tuttavia, in un certo
  senso, lo è.
  \cite{deleuze2009}
  \end{sf}
\end{quote}

\begin{quote}
\begin{sf}
\small
  Non c'è opera d'arte che non faccia appello a un popolo che non esiste ancora.
  \cite{deleuze2009}
  \end{sf}
\end{quote}

\begin{quote}
\begin{sf}
\small
  Un'arte produce invece necessariamente un che di inatteso, di non riconosciuto,
  di non-riconoscibile.
  \cite{deleuze2009}
  \end{sf}
\end{quote}

\begin{quote}
\begin{sf}
\small
  Un'opera è sempre la creazione di un nuovo spazio-tempo\ldots deve far scaturire
  problemi e questioni in cui veniamo presi, piuttosto che dare risposte. Un'opera
  è una nuova sintassi, cosa che è molto più importante del vocabolario, e scava
  una lingua straniera nella lingua.
  \cite{deleuze2009}
  \end{sf}
\end{quote}

\begin{quote}
\begin{sf}
\small
  La novità è il solo criterio dell'opera. Se non si pensa di aver visto
  qualcosa di nuovo o di aver qualcosa di nuovo da dire, perché mai scrivere,
  dipingere o prendere una cinepresa? [\ldots] Il nuovo, in questo senso, è
  sempre l'inatteso, ma anche ciò che diventa immediatamente eterno e necessario.

  \cite{deleuze2009}
  \end{sf}
\end{quote}
