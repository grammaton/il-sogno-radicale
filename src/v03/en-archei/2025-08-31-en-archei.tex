\section{\emph{en archei}}

\begin{quote}
\begin{sf}
\small
Dove ha origine – ci si chiede in queste pagine – il fascino che accompagna l'apparizione di qualcosa quando questo si staglia dal suo contesto abituale per lasciare traccia del suo passaggio?
\end{sf}
\end{quote}

\emph{dove ha origine?}

È un dove nasce e, insieme, un quando inizia. Ma è anche un quando nasce e, insieme, un dove inizia. \emph{Origins}, punto di partenza temporale oriente, \emph{oriens}, che sorge. Dove ha origine è una corrispondenza spazio–temporale. Un \emph{archē}.


\begin{quote}
\begin{sf}
\small
L'archeologia è la ricerca di un \emph{archē}, ma il termine greco \emph{archē} ha due significati: significa tanto “origine, principio”, quanto “comando ordine”. Così, il verbo \emph{archo} significa “iniziare, essere il primo a fare qualcosa” ma significa anche “comandare, essere il capo”. Senza dimensione che l'arconte (letteralmente “colui che comincia”) era in Atene la suprema magistratura. \cite{agamben2017}
\end{sf}
\end{quote}

\emph{En archei}, in principio, nel comando, «cioè nella forma di un comando», «l'inizio è sempre anche il principio che governa e comanda».

Archeologia del fascino che accompagna l'apparizione. Qualcosa appare e nell'apparire ammalia. Qualcosa appare e nell'apparire si lega insieme all'apparizione, come una fase che vincola e attrae irresistibilmente. L'archeologia del fascino mantiene l'idea di un potere che opera al di là della volontà razionale, \emph{en archei}, il potere di un comando. «Qualcosa appare a partire da se stesso» \cite{agamben2019} Apparire, mostrarsi, divenire visibile a partire da se stesso. È un \emph{ad – parēre}, un movimento verso.

\begin{quote}
\begin{sf}
\small
Il presente, l'\emph{Anwesend}, deve avermi già rivolto la parola nel suo essere presente. \cite{agamben2019}
\end{sf}
\end{quote}

C'è un qualcosa. C'è un movimento. C'è un qualcosa che si muove. C'è un fascino che accompagna il qualcosa in movimento. Ha tutti i tratti per essere definito un segnale: mezzo – movimento – informazione\footnote{attenzione: Ronchi dedica il libro al passaggio della cometa halle Bopp. APPROFONDIRE}.

\begin{quote}
\begin{sf}
\small
  Ciò che è al presente è quello che l'immagine “rappresenta”, ma non l'immagine
  stessa. L'immagine è un insieme di rapporti di tempo, da cui scaturisce il
  presente come comune multiplo o come minimo divisore. I rapporti di tmepo non
  sono mai visibili nella percezione ordinaria, ma lo sono nell'immagine, non
  appena essa diventa creatrice. L'immagine rende sensibili, visibili, i rapporti
  di tmepo irriducibli al presente.
  \cite{deleuze2009}
  \end{sf}
\end{quote}

\begin{quote}
\begin{sf}
\small
  Non c'è un inizio così come non c'è una fine. Si arriva sempre nel mezzo di
  qualcosa, si crea solo nel mezzo, dando nuove direzioni o biforcazini a delle
  linee che preesistono.
  \cite{deleuze2009}
  \end{sf}
\end{quote}

%-------------------------------------------------------------------------------
%----------------------------------------------------------------- SUB-SECTION -
%-------------------------------------------------------------------------------

\subsection{\emph{ichnos}}

Il termine \textit{ichnos} (ἴχνος) affonda le sue radici nella proto-lingua indoeuropea attraverso la radice \textit{*seik-} o \textit{*seig-}, che veicola l'idea fondamentale di "raggiungere", "estendersi" o "toccare" \cite{chantraine1968}. Questa radice si ritrova in diverse lingue della famiglia indoeuropea: nel sanscrito \textit{siṣakti} (egli cerca), nel germanico antico \textit{*saikjan-} (cercare), nel latino \textit{signum} (segno, marca).

La peculiarità del greco \textit{ichnos} risiede nella specificazione semantica verso il concetto di “impronta del piede”, ma con un'estensione metaforica verso qualsiasi tipo di traccia o segno lasciato da un passaggio. Il verbo correlato \textit{ἰχνεύω} (ichneuo) significa “seguire le tracce”, “investigare”, “rintracciare”, introducendo una dimensione dinamica e processuale.

Nei testi omerici, \textit{ichnos} compare principalmente in contesti venatori o militari, designando le tracce che permettono di seguire una preda o un nemico. Tuttavia, già in Pindaro troviamo un uso metaforico del termine per indicare la “via” o il “percorso” \cite{pindar1997}.

Nei frammenti di Parmenide, l'idea di traccia si lega alla questione dell'essere e del non-essere, mentre in Platone la nozione di \textit{typos} (τύπος) - intimamente connessa a \textit{ichnos} - diventa centrale nella teoria delle idee come "impronte" impresse sulla materia.

Il termine “traccia”, dal latino medievale \textit{tractiare} (frequentativo di \textit{trahere}), introduce una dimensione temporale specifica: quella della successione e della direzione. A differenza dell'\textit{ichnos} greco, che può indicare anche un segno statico, “traccia” implica sempre movimento e processualità, il segno di un passaggio che orienta verso una direzione futura.

%La traccia non è mai un semplice residuo passivo, ma il segno di un passaggio che orienta verso una direzione futura. In Levinas, la traccia dell'Altro è precisamente ciò che eccede ogni presente, ogni presenza possibile, orientando verso un'alterità irriducibile \cite{levinas1974}.

“Impronta”, da \textit{imprimere} (premere sopra), evoca la dinamica del contatto diretto, della forza che si esercita su una superficie. Questo termine porta con sé una dimensione di necessità fisica che manca negli altri due: l'impronta testimonia un incontro reale, una resistenza vinta, una forma imposta. L'impronta è sempre il risultato di una relazione, di un contatto che trasforma simultaneamente il soggetto imprimente e l'oggetto che riceve l'impressione \cite{merleau-ponty1945}.

“Vestigio”, dal latino \textit{vestigium}, introduce la dimensione della perdita e della frammentarietà. Il vestigio è ciò che resta \textit{in quanto} resto, ciò che porta in sé il segno della propria incompletezza essenziale, è forma specifica di apparizione del tempo storico. Il vestigio rende visibile la dialettica tra natura e storia, mostrando come ogni costruzione culturale porti in sé i semi della propria dissoluzione \cite{benjamin1928}.

%\subsection{Archeologia della coscienza temporale}

L'analisi husserliana della coscienza interna del tempo \cite{husserl1928} offre uno strumento concettuale per comprendere come la triplice articolazione traccia-impronta-vestigio operi nella costituzione dell'esperienza temporale soggettiva.

La \textit{ritenzione} primaria corrisponde alla modalità della traccia: è il processo attraverso cui il presente appena trascorso viene "trattenuto" nella coscienza, creando quella continuità temporale che Husserl chiama "flusso". La ritenzione non è memoria nel senso forte, ma quella forma minima di passato che è ancora presente, che "traccia" la direzione del flusso temporale.

L'\textit{impressione originaria} corrisponde alla modalità dell'impronta: è il punto-limite in cui l'oggetto temporale si dà alla coscienza nel suo massimo di presenza. L'impressione è sempre evanescente, ma nella sua fugacità lascia un'impronta duratura nella strutura della coscienza.

La \textit{ritenzione secondaria} o memoria propriamente detta opera secondo la modalità del vestigio: richiama il passato non nella sua pienezza originaria, ma come resto, come frammento che deve essere ricostruito attraverso un atto specifico della coscienza \cite{husserl1913}.

%\subsubsection*{Durata e interpenetrazione}

La filosofia bergsoniana della durata \cite{bergson1896} offre un modello alternativo in cui la distinzione tra traccia, impronta e vestigio si complica produttivamente. Nella durata pura, passato e presente si interpenetrano in modo tale che ogni momento presente contiene virtualmente tutto il passato.

In questa prospettiva, la traccia mnonica non è più il semplice collegamento tra momenti temporali discreti, ma l'espressione della continuità qualitativa della coscienza. L'impronta non è più l'effetto meccanico di una causa esterna, ma la cristallizzazione momentanea di un flusso continuo. Il vestigio non è più il residuo di una pienezza perduta, ma la virtualità che si attualizza in ogni momento presente.

%\section{Temporalità dell'esperienza: \textit{chronos}, \textit{kairos}, \textit{aion}}
%
%\subsection{Tempo quantitativo e tempo qualitativo}
%
%La distinzione greca tra \textit{chronos} e \textit{kairos} offre un quadro concettuale per comprendere come traccia, impronta e vestigio si rapportino diversamente alla temporalità.
%
%Il \textit{chronos} è il tempo della successione, della misura, della quantità. In questa dimensione temporale, la traccia opera come elemento di connessione tra istanti discreti, l'impronta come fissazione momentanea, il vestigio come resto temporalmente situato in un "prima" determinabile.
%
%Il \textit{kairos} è il tempo opportuno, qualitativo, dell'intensità. Qui la traccia diventa orientamento esistenziale, l'impronta si trasforma in evento trasformativo, il vestigio si rivela come apertura di senso che eccede la sua collocazione temporale specifica \cite{agamben2000}.

%\subsection{L'eternità nell'istante: la dimensione dell'\textit{aion}}
%
%La nozione di \textit{aion} (eternità) introduce una terza dimensione temporale che complica ulteriormente il quadro. L'\textit{aion} non è né il tempo successivo del \textit{chronos} né il tempo intensivo del \textit{kairos}, ma quella dimensione di eternità che può manifestarsi nell'istante stesso.
%
%In questa prospettiva, il vestigio assume una valenza particolare: non è semplicemente resto del passato nel presente, ma apertura dell'eternità nel tempo. La rovina, il frammento, il vestigio diventano paradossalmente i luoghi privilegiati in cui l'eternità si manifesta, non nonostante ma proprio attraverso la loro incompletezza \cite{jankelvitch1957}.

%\section{Ermeneutica dei segni: indice, icona, simbolo}
%
%\subsection{La semiotica peirciana applicata}

La tipologia segnica di Charles Sanders Peirce \cite{peirce1931} offre un quadro teorico per comprendere come traccia, impronta e vestigio operino differentemente nel processo interpretativo.

La traccia funziona primariamente come \textit{indice}: intrattiene con il suo oggetto una relazione di contiguità fisica o causale. La traccia nel bosco indica il passaggio di un animale non per somiglianza ma per connessione reale. Tuttavia, nella sua dimensione ermeneutica, la traccia tende a trasformarsi in simbolo: diventa segno di un sistema di significati che eccede la sua materialità immediata.

L'impronta opera prevalentemente secondo la modalità \textit{iconica}: intrattiene con il suo oggetto una relazione di somiglianza strutturale. L'impronta del piede somiglia al piede, l'impronta della mano conserva la forma della mano. Questa somiglianza non è però mai perfetta: nell'impronta avviene sempre una trasformazione, una traduzione che introduce elementi di interpretazione.

Il vestigio funziona principalmente come \textit{simbolo}: la sua relazione con l'oggetto è mediata da un sistema convenzionale di significati. La rovina romana "significa" l'impero non per somiglianza né per connessione causale diretta, ma attraverso una rete complessa di mediazioni culturali e storiche.

%\subsection{Dialettica dell'interpretazione}

Ogni segno concreto può funzionare simultaneamente secondo tutte e tre le modalità, e il processo interpretativo consiste proprio nel cogliere questa complessità semiotica.

Il lavoro ermeneutico consiste nel seguire la traccia (dimensione indicale), riconoscere l'impronta (dimensione iconica), e ricostruire il vestigio (dimensione simbolica). Ma questa sequenza non è lineare: l'interpretazione procede per circoli ermeneutici in cui ogni livello illumina retroattivamente gli altri \cite{gadamer1960}.

%\section{Ontologie del decadimento: tra dissoluzione e rivelazione}
%
%\subsection{Il decadimento come struttura esistenziale}

L'analisi heideggeriana del \textit{Verfallen} (decadimento) \cite{heidegger1927} è struttura ontologica fondamentale dell'essere-nel-mondo. Il decadimento è il modo in cui l'Esserci (Dasein) tende spontaneamente a disperdersi nella quotidianità, a perdere se stesso nelle relazioni con gli enti intramondani.

In questa prospettiva, traccia, impronta e vestigio sono le forme attraverso cui il decadimento stesso si manifesta e, paradossalmente, si rivela. %La traccia testimonia la dispersione dell'Esserci nel mondo; l'impronta fissa momentaneamente questa dispersione; il vestigio conserva la memoria di autenticità perdute.

%\subsection{La rovina come forma di verità}

La rovina nella concezione benjaminiana \cite{benjamin1928} non è decadimento di una forma originaria, ma forma originaria di verità. Il vestigio è intensificazione di senso. Nella sua incompletezza, nel suo carattere frammentario, il vestigio rivela verità che la forma integra nascondeva. La rovina è più eloquente dell'edificio intatto perché mostra la temporalità come struttura costitutiva di ogni costruzione umana.

%\subsection{Estetica dell'impermanenza}

L'estetica giapponese del \textit{wabi-sabi} \cite{kuki1930} offre un modello alternativo per pensare il rapporto tra bellezza e decadimento. Il \textit{wabi} (bellezza dell'imperfezione) e il \textit{sabi} (patina del tempo) non sono semplici categorie estetiche ma modalità di comprensione dell'esistenza.

In questa tradizione, il vestigio è rivelazione della \textit{mono no aware} (la tristezza delle cose), quella consapevolezza dell'impermanenza che costituisce il fondo dell'esperienza estetica. La traccia rivela la bellezza del transitorio; l'impronta non fissa una presenza ma mostra la grazia del passaggio.

%\section{Dialettica della presenza: tra fenomenologia e decostruzione}
%
%\subsection{La traccia derridiana}

La riflessione derridiana sulla traccia \cite{derrida1967} costituisce un punto di svolta nella comprensione del rapporto tra segno e presenza. La traccia non è più semplicemente ciò che resta di una presenza passata, ma la condizione di possibilità di ogni presenza.

In \textit{Della grammatologia}, Derrida mostra come ogni presenza sia sempre già segnata dall'assenza, sempre già "tracciata" dalla differenza. La traccia non è fenomeno tra altri, ma la struttura generale della significazione. Non c'è presenza pura che non sia sempre già traccia di altro da sé.

Questa concezione modifica radicalmente lo statuto dell'impronta e del vestigio. L'impronta non testimonia più un contatto originario tra presenza e presenza, ma il processo stesso attraverso cui ogni presenza si costituisce come differita. Il vestigio non è più resto di una pienezza originaria, ma la forma originaria attraverso cui ogni pienezza si dà come sempre già perduta.

%\subsection{Fenomenologia della latenza}

La fenomenologia di Merleau-Ponty \cite{merleau-ponty1945} sviluppa una concezione della percezione che riabilita il ruolo dell'assente nella costituzione del presente. Il "campo di presenza" non è fatto di elementi positivamente dati, ma di un intreccio complesso di presenza e latenza, di attuale e virtuale. Ogni percezione presente porta con sé un “orizzonte” di tracce del passato e di anticipazioni del futuro. L'impronta non è più semplicemente impressa dall'esterno, ma risulta dall'intreccio tra attività e passività che costituisce la struttura chiasmatica della percezione.

%\section{Verso un'ontologia dell'intermittenza}
%
%\subsection{Il regime dell'apparire}

La questione della traccia, dell'impronta e del vestigio ci conduce verso quella che potremmo chiamare un'ontologia dell'intermittenza: un pensiero dell'essere che non si fonda sulla stabilità della presenza ma sull'alternarsi di apparizione e sparizione, di manifestazione e latenza.

Didi-Huberman \cite{didi-huberman2002} ha mostrato come l'immagine operi secondo questa logica dell'intermittenza: non è mai semplicemente presente né semplicemente assente, ma intermittente. Allo stesso modo, traccia, impronta e vestigio sono modalità dell'intermittenza ontologica: modi in cui l'essere si dà nel ritirarsi, si manifesta nel nascondersi.

%\subsection{Temporalità anacronistica}

Questa ontologia dell'intermittenza implica una concezione anacronistica del tempo: un tempo che non procede linearmente dal passato verso il futuro, ma in cui presente, passato e futuro si intrecciano in configurazioni sempre nuove.

%Il vestigio non appartiene semplicemente al passato ma può "tornare" nel presente, riattualizzarsi in modi inattesi. La traccia non orienta semplicemente verso il futuro ma può "retroagire" sul passato, modificandone retroattivamente il senso. L'impronta non fissa semplicemente il presente ma può "perdurare" in modi che eccedono la sua durata cronologica.

%\section{Conclusioni: per un pensiero della soglia}

La soglia è lo spazio intermedio in cui si decide il passaggio dall'uno all'altro. Traccia, impronta e vestigio sono soglie temporali (tra passato e presente), ontologiche (tra essere e non-essere), ermeneutiche (tra senso e non-senso).

Pensare a partire da queste soglie significa abbandonare le ontologie della presenza piena per avventurarsi in un pensiero dell'intermittenza, dell'anacronismo, della differanza. Significa riconoscere che l'esperienza umana non si dà mai nella forma della presenza immediata ma sempre attraverso mediazioni temporali, semiotiche, corporee che portano in sé la marca dell'alterità.
