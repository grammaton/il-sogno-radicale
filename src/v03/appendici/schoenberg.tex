\section{Arnold Schoenberg: Teoria o metodologia?}

Colui che insegna composizione musicale è chiamato insegnante di teoria; chi poi ha scritto un trattato di armonia è chiamato addirittura teorico. Eppure a un falegname, che anch'egli deve insegnare il mestiere al suo apprendista, non verrà mai in mente di spacciarsi per «insegnante di teoria»: semmai si definisce mastro falegname, che è però più designazione di condizione che titolo; e in nessun caso si ritiene qualcosa come un dotto, anche se alla fin fine conosce il suo mestiere. Se esiste una differenza, essa può consistere solo nel fatto che la tecnica della musica è più «teorica» di quella del falegname: il che non è poi cosi ovvio. Il fatto che il falegname sappia come si uniscono solidamente tra loro i diversi pezzi di legno si basa sull'osservazione acuta e sull'esperienza, così come avviene quando il «teorico» musicale sa collegare tra loro con efficacia gli accordi. Se il falegname sa che qualità di legname usare per determinate esigenze è perché valuta i rapporti naturali e del materiale, così come fa il «teorico» musicale quando stabilisce, valutando le possibilità dei temi, la lunghezza che può avere un pezzo di musica. Quando il falegname guarnisce con scannellature una superficie piatta per ravvivarla, anch'egli denota cattivo gusto e poca fantasia al pari di quasi tutti gli artisti, ma sempre in misura non inferiore a quella di un «teorico» della musica. Dunque, l'insegnamento del falegname si basa, come quello dell'insegnante di teoria, sull'osservazione, sull'esperienza, sulla riflessione e il gusto, sulla conoscenza delle leggi naturali e delle proprietà del materiale; ma allora, esiste proprio una differenza sostanziale?

Perché dunque un mastro. falegname non si chiama «teorico», o un teorico della musica «mastro musico»? Perché c'è una piccola differenza: il falegname non dovrebbe mai intendere il suo mestiere da un punto di vista esclusivamente teorico, mentre il teorico musicale in genere non ha nessuna capacità pratica; insomma non è un «mastro». Altra differenza: il vero teorico della musica si vergogna del mestiere, perché non è cosa \emph{sua} ma \emph{di altri}; e non gli basta nascondere questo fatto, senza far di necessità virtù, cosicché il titolo di «maestro» è svalutato, perché potrebbe generare confusione con altri mestieri. Di qui la terza differenza: a una Professione più nobile deve corrispondere un più nobile titolo, e per questo la musica - benché ancor oggi il grande musicista venga chiamato «maestro» - non ha, come persino la pittura, un tirocinio artigianale ma l'insegnamento della teoria.

Conseguenza è che nessun'arte è stata tanto ostacolata nel suo sviluppo dai suoi insegnanti quanto la musica, perché non esiste custode più geloso dei suoi averi di chi sa che essi, in fondo, non gli appartengono: quanto più è difficile dimostrare la legittimità del proprio possesso, tanto maggiore è lo sforzo di procurarsela. Il teorico, che di solito non è artista o è un cattivo artista (che è quanto dire che artista non è), ha tutte le ragioni per darsi da fare a consolidare la sua posizione innaturale. Sa che l'allievo imparerà soprattutto dai modelli che i veri Maestri gli indicano nei loro capolavori; e se gli allievi potessero assistere alla composizione della musica come possono assistere alla nascita di un quadro, se ci fossero studi di composizione come ci sono di pittura, sarebbe allora chiaro che l'insegnante di teoria è superfluo, anzi nocivo, come le accademie d'arte.

Questo il teorico lo avverte, e si cerca un surrogato ponendo la teoria, il sistema, al posto del modello vivo.
Non voglio polemizzare con i tentativi onesti di chi cerca di trovare le presunte leggi dell'arte: essi sono necessari soprattutto per l'insaziabile cervello dell'uomo. L'impulso più nobile, quello della conoscenza, cl impone il dovere della ricerca; e un'erronea dottrina che sia frutto di un'onesta ricerca sta sempre più in alto della sicurezza contemplativa di chi la rinnega, perché crede di sapere senza aver cercato di persona. È addirittura nostro dovere meditare continuamente sulle cause misteriose di ogni risultato artistico, senza mai stancarci di cominciare da principio, sempre osservando e sempre cercando un nostro ordine, considerando come elementi dati solo i fenomeni, i quali possono essere ritenuti eterni a maggior diritto che non le leggi che si crede di trovare. Poiché li conosciamo con certezza, avremmo più diritto a chiamar «scienza» quel che sappiamo di loro che non tutte le supposizioni con cui cerchiamo di spiegarli.

Ma anche queste supposizioni sono giustificate, in quanto sono tentativi, risultati di applicazioni del pensiero, ginnastica spirituale, forse qualche volta addirittura i primi passi verso la verità.

Se la teoria dell'arte si appagasse di questo e si accontentasse della ricompensa garantita da un'onesta ricerca, non ci sarebbe nulla da obiet-tarle. Il fatto è che essa vuol essere qualcosa di più: non vuol essere il tentativo di trovar delle leggi, ma pretende di aver trovato le leggi \emph{eterne}. Essa osserva una serie di fenomeni, li classifica secondo alcune caratteristiche comuni e ne deduce le leggi. Questo è giusto, non foss'altro per il fatto che purtroppo non vi sono quasi altre possibilità. Ma qui incomincia l'errore, perché si trae la falsa conclusione che queste leggi, dal momento che sono apparentemente vere per i fenomeni osservati fino a quel momento, dovrebbero esser valide d'ora innanzi anche per tutti i fenomeni futuri. Ecco il più nefasto risultato: si crede di aver trovato una \emph{norma} per determinare il valore d'arte anche delle opere future. Per quanto spesso i teorici siano stati sconfessati dalla realtà quando dichiaravano non-artistico «ciò che non seguiva le loro regole», tuttavia non fanno che «insistere nell'errore». Che sarebbero infatti se non avessero almeno il monopolio della bellezza, dal momento che l'arte non appartiene loro? Che sarebbero se a tutti e per tutti i tempi fosse chiaro ciò che ancora una volta uno di loro mostra? Che sarebbero, dal momento che in realtà l'arte si trasmette con i capolavori e non con le leggi della bellezza? Che cosa li porrebbe al disopra di un mastro falegname?

Qualcuno potrà sostenere che io esagero, e che ognuno sa oggidi che l'estetica non prescrive «leggi di bellezza», ma cerca di dedurne l'esistenza dai risultati delle opere d'arte. Esatto: questo oggi lo sanno quasi tutti; ma nessuno ne tien conto, ed è questo che importa. Faccio un esempio. Credo di essere riuscito in questo libro a confutare alcuni vecchi pregiudizi dell'estetica musicale, e il fatto stesso che essi siano esistiti finora sarebbe già una dimostrazione sufficiente della mia tesi. Ma quando affermo che esiste qualcosa che per me non è requisito necessario per conseguire un buon risultato artistico, quando dico che la tonalità non è una legge naturale ed eterna della musica, tutti vedono che i teorici saltano su indignati e mettono in dubbio la mia onorabilità: chi mai sarebbe oggi disposto ad ammettere una cosa simile, anche se lo dimostrassi ancor più rigorosamente di quanto non farò in queste pagine?

Per consolidare la sua posizione insostenibile il teorico ha bisogno di una forza che gli viene dall'alleanza con l'estetica, la quale si occupa solo di cose eterne e nella vita arriva dunque sempre troppo tardi. Questa posizione è detta conservativa, il che è ridicolo quanto un treno direttissimo «conservativo»: ma l'estetica garantisce troppi vantaggi al teotico perché questi se ne possa dar pena. «Uno dei mezzi più efficaci per ottenere in musica un buon risultato formale è la tonalità»: ecco una frase che, detta dal maestro all'allievo, non ha nulla di grandioso. Ma com'è diverso quando egli parla del principio della tonalità come di una legge («tu devi1\ldots»), una legge che è indispensabile seguire per ottenere dei buoni risultati! Indispensabile\ldots Si sente un soffio di eternità! Prova a sentire diversamente, giovane artista, e li avrai tutti contro, quelli che sanno da gran pezza quello che dico qui. Ti chiameranno «giovinastro spargi-zizzania», «imbroglione», ti calunnieranno («hai voluto turlupinarci, bluffare»); e quando ti avranno ben bene insozzato con la loro volgarità, se ne staranno come quegli eroi che hanno ritenuto vile non rischiare per le loro idee qualcosa che nuoce solo all'avversario; e il briccone, naturalmente, sei tu.

Al diavolo tutte queste teorie, se servono solo a incatenare l'evoluzione dell'arte, e se tutto quello. che hanno di positivo consiste, al massimo, nell'aiutare coloro che saranno comunque dei cattivi compositori a diventar presto tali.

Tutti questi «teorici» non realizzano quello che si potrebbe esigere da loro: la forma con cui fanno dell'estetica è estremamente primitiva. Non vanno molto più in là di qualche bella frase, e dall'estetica hanno preso soprattutto il metodo di spiccare affermazioni e giudizi apodittici. Asseriscono per esempio: «Questo passaggio è buono, è cattivo» (mentre sarebbe più giusto e onesto dire «bello» o «brutto»); il che è, in primo luogo, una presunzione bell'e buona, e secondariamente è anche un giudizio estetico. E poi, perché si dovrebbe prestar fede a tale giudizio se non è dimostrato? Bisogna prestar fede all'autorità del teorico? E perché? Costui, quando non dà dimostrazioni precise, dice ciò che sa (cioè quello che non ha trovato per conto suo e che ha dovuto imparare), oppure dice quello che tutti credono, in quanto costituisce l'esperienza generale. Ma la bellezza non ha nulla da spartire con l'esperienza comune, semmai solo con l'esperienza di singoli individui. Innanzi tutto, se un giudizio del genere potesse essere valido senza ulteriore dimostrazione, quest'ultima dovrebbe derivare dal sistema con tale necessità da essere sottintesa. Ma qui abbiamo messo il dito sulla piaga più dolorosa dei teorici: le loro teorie vogliono aver la funzione di un'estetica pratica, vogliono influenzare il sentimento del bello, nel senso che esso dovrebbe, per esempio, pro-durre, per mezzo dei concatenamenti armonici, effetti considerati «belli»; vogliono avere il diritto di escludere quelle sonorità e quei concatenamenti che passano per brutti. Tuttavia queste teorie non sono costruite in modo tale che dai loro principi e dal loro coerente sviluppo nasca da sé la valutazione estetica; al contrario, non esiste nessuna relazione, assolutamente nessuna relazione. Questi giudizi di bello e di brutto sono escursioni del tutto arbitrarie nel campo dell'estetica, che non hanno nulla a che vedere con la struttura dell'insieme. Le «quinte» parallele sono cattive (e perché?), questa nota di passaggio è dura (e perché?), gli accordi di «nona» non esistono, oppure sono duri (e perché?). Dove si trova nel sistema dell'armonia la risposta comune a questi tre «perché»? Il sentimento del bello? E che cos'è il sentimento del bello? E poi, in che rapporto sta il sentimento del bello con questo sistema? Sì, proprio, con il sistema!

Questi sistemi! Mostrerò, in altra occasione, che essi a ben guardare non sono nemmeno quello che in fondo potrebbero essere, vale a dire sistemi di esposizione, metodi che distribuiscano unitariamente la materia, la suddividano con chiarezza e partano da principi che garantiscano una successione ininterrotta. Mostrerò che il sistema dei teorici si esaurisce assai presto e deve spesso essere infranto per poi, riassestato da un secondo sistema (che non è nemmeno tale), far posto in qualche modo almeno ai fenomeni più noti. Eppure la cosa dovrebbe andare ben diversamente. Un vero sistema dovrebbe avere innanzi tutto dei principi che abbraccino tutti i fenomeni; e l'ideale sarebbe che ne comprendessero tanti quanti ce ne sono in realtà, non uno di più né uno di meno. Questi principi sono leggi di natura, e solo quei principi che non sono obbligati ad ammettere eccezioni dovrebbero pretendere di essere considerati universalmente validi, in quanto avrebbero in comune con le leggi di natura tale proprietà: mentre le leggi dell'arte abbondano soprattutto di eccezioni!

Anch'io non ho potuto scoprire questi principi e credo che non si arriverà a scoprirli tanto presto. I tentativi di ricondurre incondizionatamente i fenomeni dell'arte a quelli della natura continueranno a naufragare per un bel pezzo; e quindi gli sforzi tendenti a scoprire le leggi dell'arte potranno tutt'al più raggiungere i risultati che ottiene un buon paragone, nel senso di acquistare influenza sul modo in cui l'organo del soggetto osservante si adegua alle peculiarità dell'oggetto osservato. Il paragone avvicina ciò che è troppo lontano, ingrandendo taluni particolari; allontana ciò che è troppo vicino, permettendo una visione d'assieme. Non è oggi possibile attribuire alle leggi dell'arte un valore superiore a questo; ma è già molto. Il tentativo di costruire leggi artistiche in base a peculiarità comuni non dovrebbe mancare in un trattato d'arte, così come non dovrebbe venir meno il principio del paragone. Tuttavia non si dovrebbe pretendere che il lettore prenda tali miseri risultati per leggi eterne o per qualcosa di simile alle leggi naturali. Infatti, come si è già detto, le leggi della natura non conoscono eccezioni, mentre le teorie dell'arte constano sostanzialmente di eccezioni. Quello che resta al di fuori di queste eccezioni può bastare, purché venga svolto come metodo di insegnamento, come sistema di esposizione la cui organizzazione può essere sensata e opportuna per raggiungere il fine dell'insegnamento, e la cui chiarezza sia chiarezza di esposizione, senza pretendere di far luce sui principi che stanno alla base di quello che forma la materia stessa dell'esposizione.

Ho cercato tutt'al più di creare un sistema del genere e non so se ho raggiunto il mio scopo. Mi sembra però di essere riuscito almeno a sfuggire a quella situazione forzosa che obbliga ad ammettere delle eccezioni.
I principi del sistema determinano un'eccedenza di casi possibili su quelli che si incontrano in realtà, errore questo comune anche ai sistemi che non riescono ugualmente a includere tutti i fenomeni musicali. Anch'io dovrò fare delle eccezioni: ma quelli le fanno in base a criteri estetici, e u-sano termini come «cattivo», «duro», «brutto» ecc., mentre non si servono di un mezzo assai meno pretenzioso e assai più veritiero, che è quello di constatare che si tratta solo di aspetti \emph{inusitati}. Ciò che è veramente brutto non è certo adatto a risultar bello, almeno nel senso di questi esteti, mentre ciò che è semplicemente inusitato può benissimo, anche se non necessariamente, entrare nell'uso. Si toglie così alla teoria della composizione una responsabilità che non ha mai potuto sostenere, ed essa può allora limitarsi a ciò che è veramente il suo assunto: ottenere dal-l'allievo una tale prontezza da metterlo in grado di creare qualcosa di \emph{sperimentata efficacia}. Non deve essa garantire che si tratti di cosa nuova, interessante o addirittura bella, ma può assicurare che, osservando i suoi precetti, si può raggiungere qualcosa che assomigli alle condizioni artigianali degli antichi capolavori, almeno fino al punto in cui la creazione si sottrae ad ogni controllo anche dal lato tecnico-meccanico.

Per quanta teoria si faccia in questo volume - e vi sono per lo più costretto per confutare false teorie e per allargare concezioni ristrette fino a comprendervi tutti i fenomeni - questo avverrà però sempre con la piena coscienza che io faccio solo paragoni nel senso sopra descritto, creo simboli, aspiro solo a collegare tra loro idee apparentemente lontane, a sollecitare la comprensibilità mediante una esposizione unitaria, a gettar semi e infondere stimoli mediante la ricchezza di interrelazioni che tutti i fenomeni hanno con un'idea, ma non a stabilire nuove leggi eterne. Se riuscirò ad insegnare a un solo allievo l'artigianato della nostra arte così a fondo come lo può fare qualsiasi falegname, allora sarò contento. E sarei orgoglioso di poter dire, variando un detto ben noto: « Ho \emph{tolto} ai miei allievi di composizione una cattiva \emph{stetica}, ma ho \emph{dato} loro in cambio un buon \emph{mestiere}.»
