%!TEX TS-program = xelatex
%!TEX encoding = UTF-8 Unicode

\documentclass[border=0.0cm]{standalone}
% Pacchetti per font e colori
\usepackage{fontspec}

\usepackage{tikz}

\usepackage{Alegreya}
\usepackage{newpxmath}
%\usepackage{fourier}

\usetikzlibrary{
  decorations.text
}

% Definizione dello stile per le curve con testo
\tikzset{
  text curve/.style={
    postaction={
      decorate,
      decoration={
        raise=-2.4pt,
        text effects along path,
        text align=center,
        text={~#1~},
        text effects/.cd,
          character count=\i,
          character total=\n,
          characters={text along path, fill=white, font=\Large},
      }
    }
  },
  % Stile per la linea base (orizzontale)
  base line/.style={
    postaction={
      decorate,
      decoration={
        raise=-2.4pt,
        text effects along path,
        text align=center,
        text={~#1~},
        text effects/.cd,
          character count=\i,
          character total=\n,
          characters={text along path, fill=white, font=\Large},
      }
    }
  },
  % Stile per l'arco superiore (da sinistra a destra)
  upper arc/.style={text curve={#1}, ->, thick},
  lower arc/.style={
    ->, thick,
    postaction={
      decorate,
      decoration={
        raise=-2.4pt,
        text effects along path,
        reverse path,
        text align=center,
        text={~#1~},
        text effects/.cd,
          character count=\i,
          character total=\n,
          characters={text along path, fill=white, font=\Large}
      }
    }
  }
}

\begin{document}
\begin{tikzpicture}
  \draw[color=white] (-14.4,-8.2) rectangle (14.4,8.2);
  \tikzstyle{every node}=[font=\small]

%  \node[anchor=north west, font=\Huge] at (-13,7) {$y[n] = x[n-m] - g \cdot x[n]$};
 % \node[anchor=north west, font=\Huge] at (-13,6) {\texttt{interprete(storia,senso)}};
%  \node[anchor=west, font=\Huge] at (-13,-6.5) {\emph{che cos'è un'interprete?}};

  \begin{scope}[rotate=3]
     % \draw[base line=grazie, very thick, ->] (-7,0) -- (7,0);
     % \draw[upper arc=una storia, very thick, ->] (-3,0) arc (180:2:3cm);
     % \draw[lower arc=il senso, very thick, ->] (3,0) arc (0:-178:3cm);
     \node[anchor=center, font=\Huge] at (0,0) {\emph{grazie!}};

    % Pallino centrale con etichetta
%    \draw[fill=black] (0,0) circle (0.1cm);
%    \node[above=3pt, font=\Large, rotate=3] at (0,0) {$m(t)$};

    % Pallini superiore e inferiore con etichette
%    \draw[fill=black] (0,3) circle (0.1cm);
%    \node[above=3pt, font=\Large, rotate=3] at (0,3) {$-g$};

%    \draw[fill=black] (0,-3) circle (0.1cm);
%    \node[below=3pt, font=\Large] at (0,-3) {IIR};
  \end{scope}

  \node[anchor=south east] at (13.9,-7) {\includegraphics[width=3cm]{2023-11-12-logo-kern.pdf}};
  \node[anchor=south east] at (13.9,-8) {\includegraphics[width=3cm]{gs-signature_a}};
\end{tikzpicture}
\end{document}
