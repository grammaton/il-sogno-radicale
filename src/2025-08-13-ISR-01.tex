%!TEX TS-program = xelatex
%!TEX encoding = UTF-8 Unicode

\documentclass{gs}

\title{il sogno radicale}
\subtitle{autobiografia di un eretico. appunti.}
\author{Giuseppe Silvi}
\date{\today}

% Definizione delle parole chiave per i metadati PDF
\kuno{filosofia}
\kdue{greco antico}
\ktre{Platone}
\kquattro{etimologia}
\kcinque{XeLaTeX}

% Nel preambolo del .tex
\addbibresource{bibliografia.bib}

\begin{document}

\maketitle

\section*{premessa}

Sto operando quella che Ronchi, nella prefazione de \emph{il pensiero bastardo} \cite{ronchi2001} ha definito una \emph{«torsione»}, una \emph{«preliminare torsione filosofica»} dell'oggetto.

\begin{quote}
Le altre pratiche (pittura, fotografia, cinema, ecc.) sono chiamate in causa solo al prezzo di una preliminare torsione filosofica del loro oggetto.
\end{quote}

La forma latina \textit{torsiō} deriva dal verbo \textit{torquēre}, riconducibile alla radice indoeuropea *terkʷ- “torcere, girare”\footnote{Pokorny IEW 1077-1078; de Vaan EDLIL 621-622.}, che designa il movimento rotatorio applicato mediante forza. La torsione implica una tensione, una resistenza, un'opposizione interna che genera forma. Un doppio moto. 

Mi è evidente che c'è un passo della torsione, quello successivo, che conduce alla \emph{distorsione}. %Magari della realtà.
Tuttavia, seppur con questo breve pensiero sto cercando di tenermene a distanza, un passo prima, in torsione non si può avere paura delle distorsioni poiché questa impedirebbe possibili \emph{contorsioni} (dialettiche). È se anche il Prof. Ronchi intendesse indicare una qualsiasi deviazione rispetto a una condizione di normalità o linearità (la torsione di un ragionamento, di una interpretazione, di una situazione), cosa che io dubito, ormai mi son mosso.

Sento un'assenza. Mi accorgo, quotidianamente, da un certo tempo, a seguito di alcune singole riflessioni, seppur non isolate, di desiderare, di avere necessità di \emph{abitare} una teoria della musica. Non è un bisogno, momentaneo e transitorio, bensì l'inevitabile presenza di un'assenza, costante. Molte delle singole riflessioni, delle risonanze singolari, vorrebbero appartenere a un riverbero, a riverberi generali: che chiamerei \emph{teoria della musica}. 

\begin{quote}
La teoria è infatti chiamata in causa, dopotutto, solo per far luce su di un'esperienza di cui non si riesce a venire a capo. 
\end{quote}

Abitare una teoria. Tuttavia, \emph{che cos'è una teoria? Che cos'è una teoria musicale? E santo cielo! Che cos'è Musica?}

Heidegger, che amava le torsioni, in un ragionamento sull'abitare \cite{heidegger1991} si muove da un primo passo «All'abitare, così sembra, perveniamo solo attraverso il costruire». Ma noi siamo in piena «crisi di alloggi». Attorno, mi guardo intorno, osservo il mio ambiente e vedo pochissime costruzioni che «albergano l'uomo», il musicista. «Il costruire non è soltanto mezzo e via per l'abitare, il costruire è già in se stesso un abitare.» Per questo motivo festeggio la mia torsione, festeggio con una danza tra necessità e creazione, per sentirmi a casa. 

Questo è il mio luogo riverberante.

\begin{quote}
Spazi vuoti pronti a venirsi\\
riempiendo uno alla volta, spazi della\\
voce nei quali si apprenderà con l'udito\ldots
\end{quote}

Il \emph{théatron} messo in piedi per (\emph{theáomai}) questa torsione è un teatro acustico, poiché la regione in cui si sta organizzando l'assedio non è, se non in contingenza, la conoscenza della visione, ma la conoscenza dell'udito. È sì che si torce: \emph{theōría}: guardare, osservare e il risultato di tutto ciò: ma a occhi chiusi, spegnendo l'immagine. \emph{Theáomai} sì, spettatore di altra natura, di altro \emph{théatron}. Della visione attenta, della contemplazione, tratteniamo la postura, la partecipazione intellettuale. Osservatori ufficiali inviati ad assistere alla festa del suono, a consultare l'oracolo che chiamiamo musica, a a riferirne ciò ch'è stato udito, sentito, appreso.  L'osservare con le orecchie è a sua volta una torsione, una con-torsione (dialettica) dell'ascoltare nelle due direzioni opposte e convergenti dell'udire e del sentire. 

Lo spazio agonistico di questo guardare (\emph{theōría}) è uno spazio della vibrazione, un tempo della vibrazione, uno spazio dei suoni: un \concetto{sonario}. 

Così si vuole raccontare una storia parallela. Al fianco del "vedere"-"comprendere" che riflette la concezione classica della conoscenza come visione intellegibile che pervade tutto il vocabolario epistemologico occidentale (idee, evidenza, intuizione\ldots) la storia che si vuole raccontare ha origini antichissime, come la visione, con lo stesso dualismo dialettico agonismo, tra teorico e pratico, con lo stesso desiderio di attraversare la scolastica medievale e giungere fino alla modernità musicale dove \emph{theōría} - ha acquisito o acquisì - significato tecnico scientifico di sistema coerente di principi che spiegano un insieme di fenomeni.

\emph{Theōría}, l'attività d'osservazione attenta e contemplativa (in musica: \emph{ascolto}) che si relaziona con la \emph{praxis} condotta strumento alla mano, come questa penna stilografica, e che adduce alla \emph{poiesis}: la musica.

Per Aristotele questo spazio agonistico è tripartito in \emph{bios theoretikos, bios praktikos, bios poietikos}. All'interno della visione di questa \emph{theōría} diviene

poiesis(praxis,theoria)


%Il termine \etimologia{filosofia}{greco φιλοσοφία, “amore per la sapienza”} rappresenta la ricerca fondamentale dell'essere umano verso la comprensione del mondo e di se stesso.
%
%\section{Platone e le idee}
%
%Il \concetto{mondo delle idee} platonico si esprime attraverso il greco \greco{κόσμος νοητός} (kosmos noetos), ovvero il mondo intelligibile contrapposto a quello sensibile.
%
%\begin{citazionefilosofica}{Platone, Repubblica VII, 514a}
%Εἰκάσαι τοιούτῳ πάθει τὴν ἡμετέραν φύσιν παιδείας τε πέρι καὶ ἀπαιδευσίας.
%\end{citazionefilosofica}
%
%Questa celebre allegoria della caverna illustra la condizione dell'uomo nei confronti della conoscenza e dell'educazione.
%
%\subsection{L'episteme e la doxa}
%
%La distinzione tra \concetto{episteme} (ἐπιστήμη, conoscenza scientifica) e \concetto{doxa} (δόξα, opinione) rappresenta uno dei pilastri del pensiero platonico.
%
%\section{Conclusioni}
%
%L'approccio filosofico che considera l'\etimologia{etimologia}{greco ἐτυμολογία, "studio del vero significato"} come strumento di comprensione concettuale si rivela particolarmente fruttuoso nello studio del pensiero antico.

% Nel documento
\printbibliography

\end{document}
