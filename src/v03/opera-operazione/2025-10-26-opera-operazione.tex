%-------------------------------------------------------------------------------
\section*{che cos'è opera?}
%-------------------------------------------------------------------------------
Ogni opera è segno. Ogni segno, una volta inciso, è nel passato. L'opera in
quanto segno è discretizzazione di un continuo. Il continuo è granulare. Ogni
opera è segno di segni: segni trapassati in un segno passato, in un segno del
passato. Opera è una relazione di allontanamento.

\begin{figure}[htbp]
\begin{center}
  \begin{tikzpicture}
  %  \draw[color=white] (-14.4,-8.2) rectangle (14.4,8.2);
    %\tikzstyle{every node}=[font=\small]

    \begin{scope}[rotate=3]
       \draw[very thick, >->] (-5,0) -- (5,0);
       \draw[upper arc=fir, very thick, ->] (-3,0) arc (180:2:3cm);
  %    \draw[very thick, ->] (3,0) arc (0:-178:3cm);

      % Pallino centrale con etichetta
      \draw[fill=black] (0,0) circle (0.1cm);
      %\node[above=3pt, font=\Large, rotate=3] at (0,0) {$m(t)$};

      % Pallini superiore e inferiore con etichette
  %    \draw[fill=black] (0,3) circle (0.1cm);
  %    \node[above=3pt, font=\Large] at (0,3) {FIR};

  %    \draw[fill=black] (0,-3) circle (0.1cm);
  %    \node[below=3pt, font=\Large] at (0,-3) {IIR};

  \node[anchor=east, rotate=3] at (-5,0) {futuro};
  \node[anchor=west, rotate=3] at (5,0) {passato};
    \end{scope}
  \end{tikzpicture}
%\caption{opera}
\label{opera1}
\end{center}
\end{figure}

L'opera è dunque un momento di crisi del tempo continuo, con determinate
condizioni di risonanza.

Opera è segno.

\section*{che cos'è operazione?}

L'operazione si avvia al primo sintomo di opera, mai prima, mai senza.
L'operazione senza opera è autoscillazione, chiusura al mondo esterno.

L'operazione è un segnale poetico, tendente al futuro. La riattivazione
dell'opera interpretata è un segnale estetico, tendente al passato: è un
trascinare risuonando. L'estetica è nel passato; la poetica preme al futuro.

\section*{che cos'è la loro tensione?}

\begin{figure}[htbp]
\begin{center}
  \begin{tikzpicture}
  %  \draw[color=white] (-14.4,-8.2) rectangle (14.4,8.2);
    %\tikzstyle{every node}=[font=\small]

    \begin{scope}[rotate=3]
       \draw[very thick, >->] (-5,0) -- (5,0);
       \draw[upper arc=opera, very thick, ->] (-3,0) arc (180:2:3cm);
      \draw[lower arc=operazione, very thick, ->] (3,0) arc (0:-178:3cm);

      % Pallino centrale con etichetta
      \draw[fill=black] (0,0) circle (0.1cm);
      %\node[above=3pt, font=\Large, rotate=3] at (0,0) {$m(t)$};

      % Pallini superiore e inferiore con etichette
  %    \draw[fill=black] (0,3) circle (0.1cm);
  %    \node[above=3pt, font=\Large] at (0,3) {FIR};

  %    \draw[fill=black] (0,-3) circle (0.1cm);
  %    \node[below=3pt, font=\Large] at (0,-3) {IIR};

  \node[anchor=east, rotate=3] at (-5,0) {futuro};
  \node[anchor=west, rotate=3] at (5,0) {passato};
    \end{scope}
  \end{tikzpicture}
%\caption{opera}
\label{opera1}
\end{center}
\end{figure}

È il senso dell'arte, nei due sensi: $\leftarrow$ poetico; $\rightarrow$ estetico.
Il futuro dell'arte è il non ancora processato. L'operazione tende al futuro. La
loro tensione è creazione, la durata dell'arte. L'arte è l'assoluto dell'opera,
di cui l'opera è discretizzazione, di cui l'operazione è integrazione.
