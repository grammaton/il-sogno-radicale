\section{\emph{théatron}: luogo riverberante}

Il greco antico ha diversi termini che esprimono concetti analoghi al
\textit{torquēre} come \textgreek{τρέπω} (trepō), “girare, volgere, cambiare
direzione”. Tuttavia, per il concetto specifico di \emph{torcere con una forza},
quindi “torcere, girare, rivolgere” si usa il verbo \textgreek{στρέφω}
(strephō) da cui deriva il sostantivo \textgreek{στροφή} (strophē), “rotazione,
torsione, volta”, da cui deriva il nostro “strofa”.

La radice di στρέφω è indoeuropea **streb(h)- "torcere", diversa da **terkʷ- ma
semanticamente affine\footnote{La distribuzione del campo semantico su più radici, ciascuna con sfumature
specifiche potrebbe arricchire la riflessione sulla “torsione” come gesto
ermeneutico che si articola in modalità differenti.}.

Quindi “strofa” deriva da “torcere, girare” e designava originariamente il
movimento fisico del coro nella tragedia greca: la “volta” che i coreuti
compivano spostandosi da destra a sinistra nell'orchestra.

La sequenza rituale era: \textgreek{στροφή} (strofe) – movimento verso sinistra;
\textgreek{ἀντιστροφή} (antistrofe) - movimento di ritorno verso destra;
\emph{ἐπῳδός} (epodo): stasi al centro.

Il termine poetico "strofa" conserva questa memoria cinetica: ogni strofa è una
“volta”, un movimento che torna su se stesso, una torsione del discorso che si
riavvolge prima di procedere. La poesia non procede linearmente come la prosa,
ma avanza per spirali, per ritorni, per riprese tematiche. La strofa è la
cellula di questo movimento tortile del senso. In termini filosofici, la
torsione poetica opera quella che Heidegger chiamava \emph{Kehre} - la svolta
che non è semplice cambio di direzione, ma ripiegamento che apre nuove
possibilità di senso. La poesia “torce” il linguaggio ordinario, lo sottopone a
una tensione che ne rivela potenzialità latenti. Così la “torsione filosofica”
dell'oggetto musicale condivide con la poesia questo gesto fondamentale di
deviazione creatrice.

\begin{quote}
\begin{sf}
\small
\ldots spazi vuoti pronti a venirsi riempiendo uno alla volta, spazi della voce
nei quali si apprenderà con l'udito\ldots~\cite{zambrano1991}
\end{sf}
\end{quote}

Il \emph{théatron} messo in piedi per (\emph{theáomai}) questa torsione è un
teatro acustico, poiché la regione in cui si sta organizzando l'assedio è la
conoscenza dell'udire. È sì che si torce: \emph{theōría}: guardare, osservare e,
anche, il risultato di tutto ciò: ma a occhi chiusi, spegnendo l'immagine.
\emph{Theáomai} sì, ma spettatore di altra natura, di altro \emph{théatron}.
Della visione attenta, della contemplazione, tratteniamo la postura, la
partecipazione intellettuale. Come \emph{theōroi}, osservatori ufficiali inviati
ad assistere alla festa del suono, a consultare l'oracolo che chiamiamo musica,
a riferirne ciò ch'è stato udito, sentito, appreso.

L'osservare con le orecchie è a sua volta una torsione, una con-torsione
(dialettica) dell'ascoltare nelle due direzioni opposte e convergenti dell'udire
e del sentire.

\begin{figure}[htbp]
\begin{center}
\begin{tikzpicture}
  \tikzstyle{every node}=[font=\small]
  \ciclodiagram{ascoltare}{udire}{sentire}
\end{tikzpicture}
\caption{L'udire prende il posto della resistenza, la resistenza meccanica si
traduce in una resitenza acustica (impedenza): il materiale risponde alle
sollecitazioni vibrazionali accogliendo, respingendo, ignorando. L'udito è già
selettivo di questa selezione: ha le sue soglie. Resiste e amplifica, è un
sistema di resistenze differenziali. Udire (\emph{Audire}) e anche un obedire
(\emph{Ob-audire}, udire verso). Sentire, connesso a \emph{sensus} e alla radice
indoeuropea \emph{sent-} (“andare, dirigersi”) il sentire è movimento orientato,
direzionalità dell'esperienza. Come \emph{tendere} è “pro-tendersi verso”,
sentire è “dirigersi verso” - entrambi esprimono tensionalità, orientamento
intenzionale.}
\label{ascoltare}
\end{center}
\end{figure}

Lo spazio agonistico \cite{ronchi2001} di questo ascoltare (\emph{theōría}) è
uno spazio della vibrazione acustica udibile, un tempo della vibrazione, uno
spazio dei suoni: un \concetto{sonario}.

\begin{quote}
\begin{sf}
\small
  Io credo che questo doppio movimento di disseminazione e di riunificazione
  semantica sia consustanziale alle nostre lingue e che soltanto attraverso
  questo gesto contraddirorio una parola possa realizzare il suo significato.
  \cite{agamben17}
  \end{sf}
\end{quote}

\apfc{sonario}{impressione}{codice}

Il \emph{sonario} è il nucleo di memoria, di esperienza, in cui si rappresenta
l'im-pressione acustica (della variazione di pressione acustica che si dissemina,
si disperde) e la costruzione di un codice (come sforzo di riunificazione con
la vibrazione perduta).

\begin{figure}[htbp]
\begin{center}
\begin{tikzpicture}
  \tikzstyle{every node}=[font=\small]
  \ciclodiagram{sonario}{impressione}{codice}
\end{tikzpicture}
\caption{}
\label{sonario}
\end{center}
\end{figure}



 il doppio movimento tra
%-------------------------------------------------------------------------------
%----------------------------------------------------------------- SUB-SECTION -
%-------------------------------------------------------------------------------
%\clearpage
\subsection{simbolario intergalattico per autostoppisti}
%\subsection{feedback: creatività, ciclo base}

Percorrere la lunga storia della fantasia creativa occidentale, da Aristotele a
Fagioli \cite{mf:istinto}, con un paio di soste lunghe nei pressi di
Bergson \cite{bergson1896} e Sartre \cite{jps:immaginario}, attraverso le questioni
musicali lucidamente illustrate da Cacciari \cite{Cacciari1995}, ha prodotto un
risultato inatteso, un'assenza: non ho \emph{alternative sonore} alle parole
immaginazione e immaginario; non ho un lessico indipendente dalla visione,
autonomo dal visibile per ciò che conctraddistingue la \emph{fantasia sonora};
l'immaginazione, l'immaginario, spesso si sostituiscono generalizzando fantasia
e creatività, amplificando l'assenza di una parola indipendente e necessaria per
l'attività creatrice sonora.

\begin{quote}
\begin{sf}
\small
Gli uomini medievali non provavano fiducia per i loro sensi; l'orecchio musicale
non offriva sufficiente sicurezza. Il metodo scolastico degli studiosi non
includeva la ricerca e l'esperimento, ma consisteva nel trovare una autorità
classica e nell'adottare le sue conclusioni riguardo a problemi contemporanei.
Certamente dal greco Aristosseno i musicisti medievali avrebbero potuto
apprendere il rispetto per l'orecchio musicale, ma proprio a causa di questa
fiducia nelle impressioni sensoriali egli era rifiutato dai monaci; uno
scrittore anonimo, probabilmente dell'XI secolo, affermò ingenuamente che Boezio
non sarebbe mai andato d'accordo con Aristosseno... \cite{sachs1996}
\end{sf}
\end{quote}

Si vuole raccontare una storia. Al fianco del \emph{vedere-comprendere} che
riflette la concezione classica della conoscenza come visione intellegibile che
pervade tutto il vocabolario epistemologico occidentale (idee, evidenza,
intuizione, immaginazione\ldots) la storia che si vuole raccontare ha origini
antichissime, come la visione, con lo stesso dualismo dialettico agonistico tra
teorico e pratico, con lo stesso eterno desiderio di giungere alla \emph{theōría}
con quel significato tecnico scientifico di sistema coerente di principi che
spiegano un insieme di fenomeni. Così, si vuole raccontare una storia %dell'esperienza musicale.

\begin{quote}
\begin{sf}
\small
we have to perceive what is coming to be and remember what has come to be. There
is no other way of following the contents of music\footnote{Aristosseno, Elementa
Harmonica, book II, 38–39. Traduzione in \emph{The Stanford Encyclopedia of
Philosophy}}
\end{sf}
\end{quote}

Mi pare che Aristosseno attivi, come il suo Maestro Aristotele, quell'\emph{agōn}
\cite{ronchi2001} tra sensazione (volpe – «pura inafferrabilità») e memoria (cane
– «pura potenza dell'afferrare»); costituisce quel «per sempre avere luogo» nella
rappresentazione istantanea, nella coesistenza tra sensazione e memoria che
chiamiamo \emph{esperienza musicale}.

Ascoltare mediante l'ascolto musicale è un passare attraverso il “foro” della
musica: oltrepassa la soglia solo un qualcosa, in un certo modo.

\vspace{1cm}

DIFFRAZIONE

«\ldots perceive what is coming» è una postura, una modalità su cui il modo della
vibrazione acustica si schianta, si riflette, si incide in negativo.

\vspace{1cm}

NEGATIVO SEGNO RONCHI

Si fa traccia dell'assoluto suono. Questo svolgersi è un processo inverso,
specchiato, di ingranaggi che ruotano in sensi opposti, cedendosi informazioni
inverse. Ci si timbra, ascoltando, con l'ascolto.

La storia, quindi, sottolinea la dimensione temporale della coscienza musicale,
ovvero la percezione di ciò che \emph{è diventato} (orientata al passato) e il
ricordo di ciò che \emph{sta per diventare} (orientato al futuro).

\begin{figure}[htbp]
\begin{center}
\begin{tikzpicture}
  \tikzstyle{every node}=[font=\small]
  \ciclodiagram{coscienza musicale}{sensazione}{memoria}
\end{tikzpicture}
\caption{Ciò che in Aristosseno è \emph{pre}-disposizione per l'esperienza
musicale, si ri-organizza all'\emph{inverso} \cite{bergson:segno} nella coscienza
di quella esperienza con la sensazione uditiva di una vibrazione che per
definizione è già passata; la memoria di quel passato è protesa ad un nuovo futuro.}
\label{coscienza}
\end{center}
\end{figure}

Penso suoni, ricordo suoni, (\emph{sogno suoni?}) e più mi impegno nel farlo e
più trovo che l'attività che faccio abbia poco a che vedere con l'immaginazione;
l'attività creativa di \emph{sonazione}, di invenzione timbrica, di messa a
punto di una tecnica d'indagine estesa dentro gli strumenti musicali
\cite{netti23} conduce alla costruzione di un \emph{sonario} rappresentabile sia
nel processo di scrittura (in funzione di riduzione simbolica) sia nella
relazione di dialogo possibile ed esclusivo tra musicisti.

\begin{description}
  \item[Glossario] Dal latino tardo \emph{glossarium}, derivato da \emph{glossa}
  (parola rara o straniera che necessita spiegazione), a sua volta dal greco
  \textgreek{γλῶσσα} (glōssa) che significa "lingua" o "linguaggio". Raccolta di
  termini specifici di una disciplina, ordinati alfabeticamente e corredati di
  definizione. Originariamente era una raccolta di termini difficili o arcaici
  (\emph{glossae}) con relative spiegazioni.
  \item[Immaginario] Dal latino \emph{imaginarius} (che esiste solo
  nell'immaginazione), derivato da \emph{imago, imaginis} (immagine,
  rappresentazione), connesso alla radice indoeuropea \emph{im-} (copiare,
  imitare). Come sostantivo, indica l'insieme delle immagini, dei simboli e
  delle rappresentazioni mentali condivise da una cultura o da un individuo. È
  il luogo concettuale dove risiedono le immagini che costituiscono il nostro
  modo di vedere e interpretare il mondo.
  \item[Scenario] Dal tardo latino \emph{scenarium}, derivato da \emph{scena}
  (palcoscenico, scena teatrale), che proviene dal greco \textgreek{σκηνή}
  (skēnē), originariamente "tenda, riparo" poi "palcoscenico". Inizialmente
  indicava l'insieme delle scene di una rappresentazione teatrale. Per
  estensione, oggi denota la descrizione di una possibile sequenza di eventi o
  sviluppi futuri, o l'ambientazione in cui si svolge un'azione.
\end{description}

Come il glossario raccoglie e definisce le parole, l'immaginario accoglie e
struttura le immagini, lo scenario esemplifica lo spazio delle scene, sia esso
verticale (coesistenza) che orizzontale (successione) così il \emph{sonario}
diventa il ricettacolo ontologico dei fenomeni sonori, in relazione all'ascolto,
alla creazione e alla riflessione umana.

\begin{description}
  \item[Sonario] Dal latino \emph{sonus} (suono) con il suffisso \emph{-arium}
  che indica raccolta, contenitore o luogo dedicato. Il \emph{Sonario} è lo
  spazio agonistico della ragione acustica che si fa metodo nella coscienza
  musicale, il \emph{théatron} dove il suono si fa interfaccia teorico-pratica –
  dove \textgreek{θεωρία}, \textgreek{πρᾶξις} e \textgreek{ποίησις} si dispiegano.
\end{description}

Il termine “simbolario” deriva dal latino “symbolum” (che a sua volta viene dal
greco \textgreek{σύμβολον} (symbolon), letteralmente “segno di riconoscimento”)
e indica una raccolta sistematica di simboli con i loro significati\footnote{Nel
contesto della segnaletica stradale, il simbolario costituisce l'insieme
codificato di tutti i pittogrammi, ideogrammi e simboli grafici utilizzati nella
comunicazione visiva del traffico.}.

L'insieme codificato di tutti i simboli costituisce un codice e, in quanto tale,
opera su due livelli semiotici:
\begin{description}
  \item[il livello simbolico immediato]: il sistema di segni visivi che
  comunicano direttamente attraverso la percezione, sfruttando convenzioni
  iconografiche universali;
  \item[il livello normativo mediato]: l'apparato di regole scritte che
  richiedono interpretazione linguistica e conoscenza giuridica per essere
  applicate.
\end{description}

Questa dualità riflette due modalità cognitive diverse: il riconoscimento
immediato (quasi istintivo) versus la comprensione mediata dal linguaggio e
dalla memoria delle norme.

“Codice” deriva dal latino \emph{codex} (genitivo \emph{codicis}), che
originariamente significava “tronco d'albero” o “pezzo di legno”.

Il passaggio da “tronco” a “sistema di regole” attraversa la materialità del
supporto scrittorio: dal legno come materia prima alla scrittura come
codificazione del sapere normativo. C'è quasi una genealogia che va dalla natura
(albero) alla cultura (diritto), mediata dalla tecnica della scrittura.

La parola mantiene oggi questa tensione tra materialità e astrazione: un codice
è insieme supporto fisico e sistema logico.

\begin{quote}
\begin{sf}
\small
  Una classificazione è sempre una sintomatologia e ciò che viene classificato
  sono dei segni, per tirarne fuori un concetto che si presenta come un evento,
  non come un'essenza astratta. Da questo punto di vista le varie discipline
  sono veramente delle materie segnaletiche.
  \cite{deleuze2009}
  \end{sf}
\end{quote}

\begin{quote}
\begin{sf}
\small
  Il cinema non va compreso come un linguaggio, ma come una materia segnaletica.
  \cite{deleuze2009}
  \end{sf}
\end{quote}


%-------------------------------------------------------------------------------
%----------------------------------------------------------------- SUB-SECTION -
%-------------------------------------------------------------------------------
\subsection{creatività e allucinazione}

Una definizione di allucinazione indica un «\emph{fenomeno} psichico,
provocato da cause diverse, per cui un individuo percepisce come reale ciò che
è solo \emph{immaginario}.» Immaginario, «che è \emph{effetto} d'immaginazione»
in contrapposizione con reale, «che è, che \emph{esiste effettivamente} e
concretamente, non illusorio, immaginario o possibile». Quindi il reale non è
possibile, in quanto reale, mentre l'immaginario può essere possibile.

\begin{quote}
  La possibilità non è la realtà, ma è anch'essa una realtà: che l'uomo possa
  fare una cosa o non possa farla, ha la sua importanza per valutare ciò che
  realmente si fa. Possibilità vuol dire libertà. \cite{ag:matst}
\end{quote}% GRAMSCI P35

L'immaginario può essere possibile; possibilità vuol dire libertà: la creatività,
libera, è un percorso di trasduzione da un'informazione immaginaria a
un'informazione possibile: un processo possibile dall'idea alla cosa.

Ma l'immaginario, quando collettivo, è anche una cosa, un \emph{insieme di
simboli e concetti presenti nella memoria} di una comunità: una memoria
collettiva. L'\emph{insieme di rappresentazioni simboliche}.

Quindi la creatività libera e contemporanea \cite{agamben08} del singolo
(quella possibile, immaginaria, inattuale) si stacca dall'immaginario collettivo
che è fondamentalmente memoria, passato, passato di altri, in una forma di
rappresentazione privata, in attesa di condivisione, in attesa forse di divenire
memoria collettiva di altri, di un altro tempo.

\begin{quote}
  Il punto cioè in cui la concezione del mondo, la contemplazione, la filosofia
  diventano «reali» perché tendono a modificare il mondo, a rovesciare la prassi.
  \cite{ag:matst}
\end{quote}% GRAMSCI P41

L'allucinazione è anche descritta come \emph{turbamento mentale}, come la
comparsa di \emph{immagini sensoriali} dotate di \emph{piena evidenza realistica},
che vengono posizionate, inquadrate, viste, udite, sentite, come sensazioni,
nella realtà esterna, formate, tuttavia, per un processo interno, anche in
assenza o senza che sia presente, o esista, l'oggetto o il fatto corrispondente.

\begin{quote}
  Creativo, occorre intenderlo quindi nel senso «relativo», di pensiero che
  modifica il modo di sentire del maggior numero e quindi la realtà stessa che
  non può essere pensata senza questo maggior numero. Creativo anche nel senso
  che insegna come non esista una «realtà» per se stante, in sé per sé, ma in
  rapporto storico con gli uomini che la modificano, ecc\ldots \cite{ag:matst}
\end{quote}% GRAMSCI P23

Per portare questo quadro, questa diagnosi, all'interno del processo creativo
sonoro, accedere alla rappresentazione, quindi alla \emph{sonazione} nella
costruzione di un \emph{sonario} personale e condivisibile è necessario
soffermarsi sul sensibile, sugli oggetti che dall'ambiente diventano coscienza,
a volte conoscienza.

Come suggerisce Domenico Guaccero \cite{branchi1977tecnologia} riparto
dall'osservazione dei fenomeni cercando la relazione organica con
l'ambiente, nel tentativo di condividere un primo prototipo di \emph{sonario}.
Il fenomeno sotto osservazione è la vibrazione acustica, quella particolare
vibrazione che rientra nell'ambito multidimensionale\footnote{%
  Un fenomeno acustico è udibile se soddisfa tutte e tre le dimensioni: tempo,
  ampiezza, frequenza.
} dell'udibile: la vibrazione è udibile, energia potenziale. Il fenomeno fisico
è esterno, e l'organo dell'udito è l'entrata, verso l'interno dell'uditore.
Fuori da quell'ambito c'è l'inudibile, dentro quell'ambito c'è l'udibile. Il
grado successivo di intimazione è il sentire: non si è più fuori, si è già
dentro l'organo di senso.

\begin{quote}
  Il senso in musica è l'interfaccia uomo/musica. \cite{stefani}
\end{quote}

La vibrazione sensibile, dentro e dopo le soglie di udibilità, entra in
relazione con la sensibilità individuale. È quello che nel senso comune
chimiamo suono, la rappresentazione della vibrazione acustica.

Il suono non è la vibrazione stessa ma il residuo di quella sensazione specifica
di quella vibrazione acustica. Non è fuori, ma dentro. È esperienza, quindi
memoria, quindi coscienza. Mentre la vibrazione è attivazione del meccanismo
uditivo, il suono è un'esperienza della mente. Questo apre ad una soglia
invalicabile: il silenzio è l'esperienza inaccessibile. La non-esperienza che
attrae e che apre ad altre esperienze (sonore). Luigi Nono, John
Cage\ldots~nuovi suoni in cerca di silenzio, verso nuove musiche: il suono
diviene storia del suono, storia del silenzio, storia della musica.

Nella meccanica del reale esperibile la coordinata temporale è inevitabile.
L'esperienza è nel tempo. Il tempo è esperienza. Se è presente la coordinata
temporale non può essereci silenzio. \emph{Il silenzio non esiste} novecentesco
è vero solo se ci si attiene alla meccanica tradizionale e al reale percepibile.
Tuttavia, il silenzio può esistere nelle strutture mentali e quindi nella fisica
quantistica, ovvero là dove la coordinata temporale smette di avere quel
significato. Potrei così giocare con il linguaggio: il silenzio è così
allusivamente attraente, relazionale, perché è quantistico. Il legame tra parola
e tempo ci pone nel dominio meccanico del dire, conducendo la parola, a volte
frammenti di essa, nel luogo senza tempo della relazione, si entra nel dominio
quantistico della poesia.

Tornando al sistema di riferimento musicale, il silenzio è nelle relazioni tra i
suoni, a cui tendiamo prima e dopo di ognuno di questi. La soglia che separa
l'uomo sensibile dal silenzio è una curva di ampiezze dipendenti dalla frequenza
nel tempo: un sistema musicale: in un sistema fisico: in un sistema filosofico.

\emph{E il rumore?} Il rumore è semplicemente la vibrazione di cui non si ha
ancora esperienza. La vita del rumore è la durata, la transizione, tra silenzio
ed esperienza.

Tendere a quella soglia ci porta a scoprire nuovi suoni, perfetti sconosciuti:
rumori. Siamo ascoltatori di uno spazio esteso oltre il conosciuto, come un
Galileo al tubo ottico: cerchiamo cose, per sentirne alcune, per udirne
altre, per dare nomi, per creare relazioni. Pionieri della \emph{sonazione}.

\begin{quote}
  [\ldots] l'unità di scienza e vita è appunto una unità attiva, in cui solo si
  realizza la libertà di pensiero, è un rapporto maestro-scolaro,
  filosofo-ambiente culturale in cui operare, da cui trarre i problemi necessari
  da impostare e risolvere, cioè il rapporto filosofia-storia. \cite{ag:matst}
\end{quote}% GRAMSCI P22

% $vibrazione(acustica) + esperienza(tempo) + spazio = timbro$. L'orecchio:
% strumento millenario, perfezionato per scappare e predare. Utilizzato come luogo
% creativo in cui costruire strumenti artificiali, l'orecchio, apre a nuove
% parole: Timbro. Quando predavamo scappando dai predatori non esisteva il
% timbro. Ci sono voluti secoli di strumenti artificiali, oggetti che hanno
% permesso una storia della musica, di tecnica e pensiero, per avere necessità di
% una nuova parola.
%
% Attività creativa, creazione come luogo particolareggiato del fare umano.
% Lo strumento come strumento di pensiero e la necessità di nuovi strumenti di
% pensiero.

L'allucinazione \emph{sonaria} è quindi una rappresentazione unica, ideale fino
a che non viene concessa e condivisa con una seconda persona: l'idea di un
singolo, privata agli altri, non è un oggetto sociale, mentre un'allucinazione
\emph{sonaria} condivisa e comprensibile è un oggetto sociale: è un possibile.
\cite{ferraris2014}
%-------------------------------------------------------------------------------
%----------------------------------------------------------------- SUB-SECTION -
%-------------------------------------------------------------------------------
\subsection{allucinazione e sogno}

Il motore che anima la ricerca è il sogno. Ogni scienza è guidata da un sogno.
Il sogno è l'anima della scienza, ogni scienza si distingue nella prospettiva di
un sogno: la ricerca di quel sogno specifico è lo spirito di quella scienza.
(Da Bergson) Il pensiero filosofico è tornato co-scienza della scienza. La
materia è memoria: la musica è un'interfaccia tecnologica della memoria.

\begin{quote}
  Non c’è, nella ragione del logos la linea che è creazione pura. La
  simbolizzazione della scrittura non può essere neppure trasformazione di una
  cosa percepita, ovvero immagine onirica, perché la linea, fuori ed oltre la
  mano dell’uomo che la segna, non esiste in natura e non può essere percepita
  e fatta ricordo o memoria. Noi percepiamo la linea che non ha figura ed ha
  forme infinite ed è senza identità manifesta, per la creazione della mano
  dell’uomo che fa i confini e definisce, rendendole visibili, le forme e la
  figura delle cose e delle immagini delle cose. E così crea anche l’identità
  della linea stessa che, ogni volta, è diversa perché infinitamente sottile o
  infinitamente lunga. E penso alla parola tempo che indica un movimento che
  non si ferma mai e non ha figura, né forma né confini. Ma il movimento
  invisibile della materia vivente non possiamo vederlo. Linea, movimento e
  tempo sono ancora tre parole che non indicano e non ricordano la figura
  di una cosa, ma sono... concetti, l’espressione di cose invisibili che si
  possono soltanto pensare. \cite{mf:left2008}
\end{quote}

La spiegazione del processo creativo che \emph{Cobb} condivide con
\emph{Ariadne}, del potenziale mentale nel sogno, è una delle tematiche
emergenti in \emph{Inception}, durante la cui scrittura l'autore, Christopher
Nolan, si è chiesto: «cosa accadrebbe se un gruppo di persone potesse
condividere (realmente) un sogno?».

\begin{figure}[ht]
\centering
%\resizebox{0.81\linewidth}{!}{%
%\includegraphics[scale=1]{tikz/inception/inception.pdf}
\begin{tikzpicture}
  \tikzstyle{every node}=[font=\small]
  \ciclodiagram{}{}{}
\end{tikzpicture}
%\captionsetup{width=.81\linewidth}
\caption{Christopher Nolan, \emph{Inception} (2010).\\
         «Imagine you're designing a building. \textbf{You consciously create
         each aspect}. But sometimes,
         \textbf{it feels like it's almost creating itself}, if you know what I
         mean». «Yeah, like \textbf{I'm discovering it}». «Genuine inspiration,
         right? Now, in a dream, our mind continuously does this.
         \textbf{We create and perceive our world simultaneusly}. And our mind
         does this so well that we don't even know it's happening.
         \textbf{That allows us to get right in the middle of that process}».
         «How?». «\textbf{By taking over the creating part}».
         Il grafico disegnato da Cobb descrive il processo creativo di un sogno
         durante la costruzione di un edificio. Nel sogno, la curva che avanza
         da sinistra verso destra è il lavoro di costruzione che avanza mentre
         la curva che torna indietro da destra verso sinistra è la percezione,
         la scoperta di un mondo che si crea da solo. La riga centrale è «la
         presa in carico della parte creativa», la consapevolezza che si sta
         creando su due piani distinti.}
\label{tikz:inception}
\end{figure}

Il disegno in fig. \ref{tikz:inception} ricalca quello proposto da \emph{Cobb}
durante la spiegazione citata.
Osservandolo, mi chiedo: «non è quello che (simbolicamente) accade nella
condivisione di una ricerca?» L'attività creativa, quella che Fagioli definisce
creazione pura, non ha origini dal percepito, si attiva, è attività di
costruzione, di architettura, di autoprogettazione.

\begin{figure}[ht]
  \centering
  \begin{tikzpicture}
    \tikzstyle{every node}=[font=\small]
    \ciclodiagram{creatività}{invenzione}{ricerca}
  \end{tikzpicture}
  %\includegraphics{tikz/ciclo-base/ciclobase.pdf}
  %\captionsetup{width=.81\linewidth}
  \caption{ciclo base: l'invenzione è il trovato; la ricerca è coscienza protesa
  fuori dal tempo. L'atto creativo è un momento di ebaborazione della memoria che
  modifica l'ambiente.}
  \label{tikz:ciclobase}
\end{figure}

Il processo è trasducibile completamente nel dominio della creazione musicale,
la consapevolezza creativa, l'atto responsabile è l'asse su cui l'invenzione
avanza verso destra, nel dominio del tempo, mentre la ricerca si muove fuori
dal tempo, nello spazio mentale di relazione con la storia e l'ambiente.
Il disegno è simbolico del puro processo creativo, sogno o arte.
%
% \marginpar{%
%   \includegraphics[width=\marginparwidth]{images/apf.png}
%   \captionof{figure}{\raggedright Diagramma di un filtro \emph{all-pass}.}
%   \label{img:apf}
% }
%
Osserviamo ora il diagramma della creazione: è un ciclo base tra l'invenzione,
il trovato nella relazione con l'ambiente e la pulsione di ricerca, con le sue
possibilità fuori del tempo e che modificano l'ambiente ri-alimentandolo con un
apparato di idee e materia completamente nuovi. Di fatto, se contempliamo il
processo creativo come quel flusso temporale in cui ci si può perdere un tempo
di sedimentazione, di accumulazione (come una pausa caffè e sigaretta) il
modello descritto è astraibile ad un filtro \emph{all-pass}. Nella letteratura
classica il filtro \emph{all-pass} viene descritto come unità riverberante. Il
processo creativo di relazioni intime, private, in una rete di relazioni tra
processi creativi può essere descritto come riverbero complesso, assimilabile
ad una \emph{Feedback Delay Network}.

%-------------------------------------------------------------------------------
%----------------------------------------------------------------- SUB-SECTION -
%-------------------------------------------------------------------------------
\subsection{sogno e politica}

\begin{quote}
  Che cos'è l'uomo? %È questa la domanda prima e princicpale della filosofia.
  [\ldots] Diaciamo dunque che l'uomo è un processo e precisamente è il
  processo dei suoi atti. [\ldots] occorre concepire l'uomo come una serie di
  rapporti attivi (un processo) in cui se l'individualità ha la massima
  importanza, non è però il solo elemento da considerare. L'umanità che si
  riflette in ogni individualità è composta di diversi elementi: 1) individuo;
  2) gli altri uomini; 3) la natura. [\ldots] Così l'uomo non entra in rapporti
  con la natura semplicemente per il fatto di essere egli stesso natura, ma
  attivamente, per mezzo del lavoro e della tecnica [\ldots] per tecnica deve
  intendersi [\ldots] anche gli strumenti «mentali», la conoscenza filosofica.
\end{quote}% GRAMSCI P27-28

È questo, innanzi tutto, un problema di linguaggio. Con linguaggio identifichiamo
un nome collettivo, che non presupone una cosa unica, è un \emph{oggetto sociale}.
\cite{ag:matst, ferraris2014} Esiste nella cultura, nella filosofia e nella
scienza, sia pure nel grado di senso comune e, in quanto tale, è una
molteplicità di oggetti, fatti, più o meno organicamente concreti e coordinati:

\begin{quote}
  al limite si può dire che ogni essere parlante ha un proprio linguaggio
  personale, cioè un proprio modo di pensare e sentire. \cite{ag:matst}
\end{quote}

E quindi la cultura unifica, raggruppa, cataloga a sua volta una maggiore o
minore quantità di individui più o meno a contatto espressivo, capaci di gradi
diversi di comprensione. Le differenze si riflettono, si riverberano nella
cultura, nel linguaggio comune e producono «ostacoli» e quelle «cause di errore»
che portano, da un lato all'incomprensione, dall'altro alla separazione, in
ogni caso riverbera in un problema ambientale. La conoscenza non è separata e
sufficiente a sé stessa, ma è coinvolta nella relazione individuo-ambiente e
nel processo culturale, vitale. I sensi non appaiono come ingressi della
conoscenza ma partecipano come stimoli all’azione.

\begin{quote}
  \ldots ognuno cambia se stesso, si modifica, nella misura in cui cambia e
  modifica tutto il complesso di rapporti di cui egli è il centro di annodamento.
  In questo senso il filosofo reale è e non può essere altri che il politico,
  cioè l'uomo attivo che modifica l'ambiente, inteso per ambiente l'insieme dei
  rapporti di cui ogni singolo entra a far parte. %Se la propria individualità
  %è l'insieme di questi rapporti, farsi una personalità significa acquistare
  %coscienza di tali rapporti, modificare la propria personalità significa
  %modificare l'insieme di questi rapporti.
  [\ldots] Ma questi rapporti, come si è detto, non sono semplici. Intanto
  alcuni di essi sono necessari, altri volontari. Inoltre averne coscienza più
  o meno profonda (cioè conoscere più o meno il modo con cui si possono
  modificare) già li modifica. Gli stessi rapporti necessari in quanto
  conosciuti nella loro necessità, cambiano d'aspetto e d'importanza.
  La conoscenza è potere in questo senso. \cite{ag:matst}
\end{quote}% GRAMSCI P29

Se la musica ha avuto ed ha un ruolo culturale fuori dall'intrattenimento, è
in questi \emph{non}-spazi e in questi \emph{non}-tempi sociali. Se il musicista
ha ancora un ruolo sociale fuori dall'intrattenitore è in questi spazi e in
questi tempi culturali. Ridefinire il proprio ambito di intervento culturale e
sociale implica una concertazione, un accordo in maniera che la meta della
ricerca sia il risultato di un’appropriata combinazione di accordo non
costrittivo e tollerante disaccordo.

\begin{quote}
  Secoli e secoli di idealismo non hanno mancato di influire sulla realtà. \cite{jlb:finzioni}
\end{quote}% BORGES

\begin{quote}
\begin{sf}
\small
  “Autore” è una funzione ce rinvia all'opera d'arte (e in altre condizioni al
  crimine).
  \cite{deleuze2009}
  \end{sf}
\end{quote}
